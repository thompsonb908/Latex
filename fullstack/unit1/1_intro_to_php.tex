\documentclass[12pt]{article}
\usepackage[top=1in, bottom=1in, left=1in, right=1in]{geometry}
%\usepackage[margin=1in]{geometry}
\usepackage[onehalfspacing]{setspace}
%\usepackage[doublespacing]{setspace}
\usepackage{amsmath, amssymb, amsthm}
\usepackage{enumerate, enumitem}
\usepackage{fancyhdr, graphicx, proof, comment, multicol}
\usepackage[none]{hyphenat} % This command prevents hyphenation of words
\binoppenalty=\maxdimen % This command and the next prevent in-line equation breaks
\relpenalty=\maxdimen
%    Good website with common symbols
% http://www.artofproblemsolving.com/wiki/index.php/LaTeX%3ASymbols
%    How to change enumeration using enumitem package
% http://tex.stackexchange.com/questions/129951/enumerate-tag-using-the-alphabet-instead-of-numbers
%    Quick post on headers
% http://timmurphy.org/2010/08/07/headers-and-footers-in-latex-using-fancyhdr/
%    Info on alignat
% http://tex.stackexchange.com/questions/229799/align-words-next-to-the-numbering
% http://tex.stackexchange.com/questions/43102/how-to-subtract-two-equations
%    Text align left-center-right
% http://tex.stackexchange.com/questions/55472/how-to-make-text-aligned-left-center-right-in-the-same-line
\usepackage{microtype} % Modifies spacing between letters and words
\usepackage{mathpazo} % Modifies font. Optional package.
\usepackage{mdframed} % Required for boxed problems.
\usepackage{parskip} % Left justifies new paragraphs.
\linespread{1.1} 


%figure support
\usepackage{import}
\usepackage{xifthen}
\pdfminorversion=7
\usepackage{pdfpages}
\usepackage{transparent}
\newcommand{\incfig}[1]{%
	\def\svgwidth{\columnwidth}
	\import{./figures/}{#1.pdf_tex}
}
\graphicspath{ {./figures/} }
\pdfsuppresswarningpagegroup=1

\newenvironment{problem}[1]
{\begin{mdframed}[linewidth=0.8pt]
        \textsc{Problem #1:}

}
    {\end{mdframed}}

\newenvironment{solution}
    {\textsc{Solution:}\\}
    {\newpage}% puts a new page after the solution
    
\newenvironment{statement}[1]
{\begin{mdframed}[linewidth=0.6pt]
        \textsc{Statement #1:}

}
    {\end{mdframed}}

%\newenvironment{prf}
 %   {\textsc{Proof:}\\}
 %   {\newpage}% puts a new page after the solution

\begin{document}
% This is the Header
% Make sure you update this information!!!!
\noindent
\textbf{Full Stack Development} \hfill \textbf{Brandon Thompson} \\
%\normalsize  \hfill Due Date:  \\

% This is where you name your homework
\begin{center}
\textbf{Introduction to PHP}
\end{center}
	\begin{itemize}
		\item What is PHP?
			\begin{itemize}
				\item PHP stands for: PHP HyperText Preprocessor
				\item Server side scripting language generally used for web development that can be embedded into HTML.
		\item How does it work?
			\begin{itemize}
				\item Retrieves data from the server and displays it in the browser by writing renderable text to the client.
			\end{itemize}
		\item What can it do?
			\begin{itemize}
				\item Serve data from server to client.
				\item Cron jobs, windows task scheduler
				\item command line scripting.
				\item Desktop applications using GUI extensions to the language.
			\end{itemize}
		\item What will you need?
			\begin{itemize}
				\item Computer with modern operating system.
				\item XAMPP: \verb|https://www.apachefriends.org/index.html|
				\item Run XAMPP manager application and start the Apache server (verify by typing \verb|localhost| in the address bar of your web browser.
				\item Text based editor (Brackets: \verb|https://brackets.io|)
			\end{itemize}
	\end{itemize}	
\end{document}

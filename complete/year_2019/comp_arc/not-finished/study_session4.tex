\documentclass[a4paper]{article}

\usepackage[utf8]{inputenc}
\usepackage[T1]{fontenc}
\usepackage{textcomp}
\usepackage[english]{babel}
\usepackage{amsmath, amssymb}


%figure support
\usepackage{import}
\usepackage{xifthen}
\pdfminorversion=7
\usepackage{pdfpages}
\usepackage{transparent}
\newcommand{\incfig}[1]{%
	\def\svgwidth{\columnwidth}
	\import{./figures/}{#1.pdf_tex}
}
\graphicspath{ {./figures/} }
\pdfsuppresswarningpagegroup=1

\begin{document}
	\title{Study Session 4: Memory}
	\author{Brandon Thompson}
	\maketitle
	
	\begin{enumerate}

		\item You are building a computer with a hierarchical memory system that consists
			of separate instruction
	and data caches followed by main memory. You are using the MIPS multi-cycle processor from figure 7.41
	running at 1GHz.
	\begin{enumerate}
		\item Suppose the instruction cache is perfect but the data cache has a 5\% miss rate. On a
			cache miss, the processor stalls for 60 ns to access main memory, then resumes normal
			operation. Taking cache misses into account, what is the average memory access time?
			\[
			t=\frac{1ns}{t_{MM}=60ns} MR=5\%\\
			AMAT = t_{hit} + MR_{MM} * t_{MM}\\
			=1ns + 0.05 * 60ns\\
			AMAT = 4ns
			.\] 
			%answer
		\item How many clock cycles per instruction (CPI) on average are required for load and store
			word instructions considering the non-ideal memory system?

			%answer
			CPI_{lw}= 4 + 4 = 8\\
			CPI_{sw}= 3 + 4 = 7
		\item Consider the benchmark application of Example 7.7 that has 25\% loads, 10\% stores, 11\%
			branches, 2\% jumps, and 52\% r-type instructions. Taking the non-ideal memory system
			into account, what is the average CPI for this benchmark?
			Average CPI =
			(0.1) + (0.02) * 3 + 0.52 * 4 + 0,1 * 7 + 0.25 * 8 = 5.17

		\item Now assume that the instruction cache is also non-ideal and has a 7\% miss rate. What
			is the average CPI for the benchmark in part (c)? Take into account both instructions
			and data cache misses.
	\end{enumerate}

		\item 
			\begin{enumerate}
				\item The instruction cache is perfect but the data cache has a 15\% miss
					rate. On a cache miss, the processor stalls for 200ns to access
					main memory, then resumes normal operation. Taking cache misses into
					account, what is the average memory access time?
					%answer

				\item How many clock cycles per instruction (CPI) on average are required
					for load and store word instructions considering the non-ideal
					memory system?
					%answer
				\item Consider the benchmark application of Example 7.7 that has 25\% loads,
					10\% stores, 11\% branches, 2\% jumps, and 52\% r-type instructions.
					Taking the non-ideal memory system into account, what is the average
					CPI for this benchmark?
					%answer
				\item Now assume that the instruction cache is also non-ideal and has a 10\%
					miss rate. What is the average CPI for the benchmark in part (c)?
					Take into account both instruction and data cache misses.
					%answer
			\end{enumerate}

		\item The chapter described the least recently used (LRU) replacement policy for multi-way
			associative caches. Other, less common, replacement policies include
			first-in-first-out (FIFO) and random policies. FIFO replacement evicts the block
			that has been there the longest, regardless of how recently it was accessed.
			Random replacement randomly picks a block to evict.
			\begin{enumerate}
				\item Discuss the advantages and disadvantages of each of these replacement
					policies.\\
					\begin{itemize}
						\item LRU Advantage: Only holds most used data.
						\item LRU Disadvantage: 
						\item FIFO Advantage: Queue, easy to implement
						\item FIFO Disadvantage: Violates spatial and temporal
							locality by evicting data used often
						\item Rand Advantage: simple implementation, less overhead.
						\item Rand Disadvantage: don't know which data is evicted.
					\end{itemize}
				\item Describe a data access pattern for which FIFO would perform better than
					LRU.\\
					access previously accessed data, loops 
			\end{enumerate}
	\end{enumerate}

	
\end{document}

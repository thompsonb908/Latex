\documentclass[a4paper]{article}

\usepackage[utf8]{inputenc}
\usepackage[T1]{fontenc}
%\usepackage{inconsolata}
\usepackage{listings}
\usepackage{textcomp}
\usepackage[english]{babel}
\usepackage{amsmath, amssymb}
\usepackage{ulem}
%\usepackage{xcolor}
\usepackage[left=1in,right=1in,top=1in,bottom=1in]{geometry}

%figure support
\usepackage{import}
\usepackage{xifthen}
\pdfminorversion=7
\usepackage{pdfpages}
\usepackage{transparent}
\newcommand{\incfig}[1]{%
	\def\svgwidth{\columnwidth}
	\import{./figures/}{#1.pdf_tex}
}
\graphicspath{ {./figures/} }
\pdfsuppresswarningpagegroup=1
\usepackage{color}
\definecolor{pblue}{rgb}{0.13,0.13,1}
\definecolor{pgreen}{rgb}{0,0.5,0}
\definecolor{pred}{rgb}{0.9,0,0}
\definecolor{pgrey}{rgb}{0.46,0.45,0.48}

\lstset{language=Java,
	showspaces=false,
	showtabs=false,
	breaklines=true,
	showstringspaces=false,
	breakatwhitespace=true,
	commentstyle=\color{pgreen},
	keywordstyle=\color{pblue},
	stringstyle=\color{pred},
	basicstyle=\ttfamily,
	moredelim=[il][\textcolor{pgrey}]{\$\$},
	moredelim=[is][\textcolor{pgrey}]{\%\%}{\%\%}
}

\begin{document}
	\title{CIS4362.01 Homework 3 Due 11/10/19}
	\author{Brandon Thompson 5517}
	\maketitle

	\begin{enumerate}
		\item Suppose we define a MAC $\text{I}_{\text{RAW}} = \left( S,V \right) $ where $S\left( k,m \right) = \text{rawCBC}\left( k,m \right) $.\\
			\textit{\emph{Explain a scenario that shows $\text{I}_{\text{RAW}}$ can be easily broken
			using a 1-chosen message attack}} (This attack scenario can prove why the
			last encryption step must be included in ECBC-MAC).\\
			If adversary gives a one-block message $m \in X$. Request the tag for $m$,
			get  $t = F\left( k,m \right) $. Output $t$ as MAC forgery for the 2-block
			message $\left( m, t\oplus m \right) $. Then:
			\begin{equation*}
				\text{rawCBC}\left( k,\left( m,t\oplus m \right)  \right) = F\left( k,F\left( k,m \right) \oplus \left( t\oplus m \right)  \right) = F\left( k,t\oplus \left( t\oplus m \right)  \right) = t
			\end{equation*}
		\item Consider the encrypted CBC MAC built from AES. Suppose we compute the tag for a
			a long message $m$ comprising of $n$ AES blocks.\\
			\\
			Let $m'$ be the $n$-block message obtained from $m$ by flipping the last bit
			of $m$ (i.e.\ if the last bit of $m$ is $b$ then the last bit of $m'$ is
			$b\oplus 1$). How many calls to AES would it take to compute the tag for $m'$ 
			from the tag for $m$ and the MAC key? (In this question ignore message
			padding and simply assume that the message length is always a multiple of
			the AES block size).\\
			\\
			Because only the last bit of the message changes, you would need to compute:
			$F\left( k,F\left( k_1,tag \right)  \right) \oplus 1$
			to generate the new message $m'$, and then compute the new tag by:
			$F\left( k_1,F\left( k,m' \right)  \right) $.\\
			So a total of 4 AES calls are made.
		\item Consider the following MAC verification algorithm:
			\begin{verbatim}
				def Verify(key, msg, sig_bytes):
				    return HMAC(key,msg) == sig_bytes
			\end{verbatim}
			\begin{enumerate}
				\item \textit{\emph{Explain how the timing attack on the above MAC
					verification algorithm can occur.}}\\
					\\
					The '\verb|==|' is a byte-by-byte comparison so it will return
					false as soon as it finds an inequality. For a target message
					make a random tag, loop over all possible first bytes until
					the verification takes slightly longer. Continue until all
					bytes in the tag are valid.
				\item \textit{\emph{Write a pseudocode that defends the aforementioned
					verification timing attack.}}
					\begin{verbatim}
						return false if sig_bytes has wrong length
						result = 0
						for x, y in zip( HMAC(key,msg), sig_bytes):
						    result |= ord(x) ^ ord(y)
						return result == 0
					\end{verbatim}
					Function checks if the \verb|sig_bytes| is the correct length,
					XOR's the MAC and the \verb|sig_bytes|, if they are the same
					then result should be 0. This works because result is fully
					calculated first, then checked if it is correct.
			\end{enumerate}
		\item Suppose Alice is broadcasting packets to 6 recipients $B_1,\ldots,B_6$. Privacy
			is not important but integrity is. In other words, each of $B_1,\ldots,B_6$ 
			should be assured that the packets he is receiving were sent by Alice.\\
			\\
			Alice decides to use a MAC. Suppose Alice and $B_1,\ldots,B_6$ all share
			a secret key $k$. Alice computes a tag for every packet she sends using
			key $k$. Each user $B_i$ verifies the tag when receiving the packet and
			drops the packet if the tag is invalid. Alice notices that this scheme is
			insecure because user $B_1$ can use the key $k$ to send packets with a
			valid tag to users $B_2,\ldots,B_6$ and they will all be fooled into
			thinking that these packets are from Alice.\\
			\\
			Instead, Alice sets up a set of 4 secret keys $S=\left\{ k_1,\ldots,k_4 \right\} $.
			She gives each user $B_i$ some subset $S_i \subseteq S$ of the keys. When
			Alice transmits a packet she appends 4 tags to it by computing the tag with
			each of her 4 keys. When user $B_i$ receives a packet he accepts is as valid
			only if all tags corresponding to his keys in $S_i$ are valid. For example,
			if user $B_1$ is given keys $\left\{ k_1,k_2 \right\} $ he will accept an
			incoming packet only if the first and second tags are valid. Note that $B_1$ 
			cannot validate the third and fourth tags because he does not have $k_3$ or $k_4$.\\
			\\
			How should Alice assign keys to the 6 users so that no single user can forge
			packets on behalf of Alice and fool some other user?
			 \begin{center}
			%	\color{red}
				\[
				S_1 = \left\{ k_2,k_3 \right\},
				S_2 = \left\{ k_2,k_4 \right\},
				S_3 = \left\{ k_3,k_4 \right\},
				S_4 = \left\{ k_1,k_2 \right\},
				S_5 = \left\{ k_1,k_3 \right\},
				S_6 = \left\{ k_1,k_4 \right\}
				\] 
			 \end{center}
		\item Use SHA256 as the hash function to hash the content of the below text. By
			completing the assignment you will gain experience using crypto libraries
			such as PyCrypto (Python), Crypto++ (C++), or any other. Write your code
			and the output of your code for hashing the below message to get full credit.\\
			\\
			''Anarchism is a political philosophy that advocates self-governed societies
			based on voluntary institutions. These are often described as stateless
			societies, although several authors have defined them more specifically as
			institutions based on non-hierarchical or free associations. Anarchism holds
			the state to be undesirable, unnecessary and harmful.''\\
			\\
			\begin{lstlisting}
import java.security.MessageDigest;
import java.security.NoSuchAlgorithmException;

public class Main {

	private static final String testMessage = "Hello World!";
	private static final String message = "Anarchism is a political philosophy that advocates self-governed societies based on voluntary institutions. These are often described as stateless societies, although several authors have defined them more specifically as institutions based on non-hierarchical or free associations. Anarchism holds the state to be undesirable, unnecessary and harmful.";

	public static void main(String[] args) {
	    System.out.println("Test message hash:");
	    getHash(testMessage);
	    System.out.println("Message hash:");
	    getHash(message);
	}

	private static void getHash(String msg) {
	    System.out.println(hashHexString(msg));
	}

	private static String hashHexString(String msg) {
	    return convertByteToHex(calculateHashBytes(msg)).toString();
	}

	private static byte[] calculateHashBytes(String msg) {
	    MessageDigest md = createMessageDigest(''SHA-256'');
	    md.update(msg.getBytes());
	    return md.digest();
	}

	public static MessageDigest createMessageDigest(String algorithm) {
	    try {
	        return MessageDigest.getInstance(algorithm);
	    }
	    catch (NoSuchAlgorithmException e) {
		e.printStackTrace();
	    }
	    return null;
	}

	private static StringBuffer convertByteToHex(byte[] bytes) {
	    StringBuffer hex = new StringBuffer();
	    for(int i = 0; i < bytes.length; i++) {
		hex.append(hexRepresentation(bytes[i]));
	    }
	    return hex;
	}

	private static String hexRepresentation(byte aByte) {
	    return Integer.toHexString(0xFF & aByte);
	}
}
\end{lstlisting}
Outputs:
\begin{verbatim}
	Test message hash:
	7f83b1657ff1fc53b92dc18148a1d65dfc2d4b1fa3d677284addd20126d9069
	Message hash:
	51c18b42adfb8a7b3082852ae88b6c2b60b4a5895d2b70efeee06d95b7f5

	Process finished with exit code 0
\end{verbatim}
	\end{enumerate}
\end{document}

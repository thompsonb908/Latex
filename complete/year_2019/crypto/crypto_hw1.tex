\documentclass[a4paper]{article}

\newcommand{\numpy}{{\tt numpy}}	% tt font for numpy

\topmargin -.5in
\textheight 9in
\oddsidemargin -.25in
\evensidemargin -.25in
\textwidth 7in

\begin{document}
	\author{Brandon Thompson 5517}
	\title{Homework Assignment #1: CIS4362.01 Due 9/15/19}
	\maketitle

	\medskip

	\begin{enumerate}
		\item
		The earliest known, and simples, use of a substitution cipher was by Julius Caesar. The Caesar  cipher involves replacing each letter of the alphabet with the letter standing 3 places further down the alphabet.
		
		Shift by 3 ('A' mapped to 'D')
		\begin{enumerate}
			\item
			Plain: DORIAN STRUCK THE NORTHERN BAHAMAS AS A CATEGORY FIVE HURRICANE\\\\
			Cipher: GRULDQ VWUXFN WKH QRUWKHUQ EDKDPDV DV D FDWHJRUB ILYH KXUULFDQH\\

			\item
			If it is known that a given cipher text is following Caesar cipher but with a different shift,
			what is the easiest cryptanalysis solution to decrypt the message?\\\\
			Can test the first couple of words for every possible shift until a correct output is given.
			Once the correct shift pattern is known, decrypt the rest of the message.\\
		\end{enumerate}
		
	\item
		Given the below cipher text which has been encrypted using substitution cipher,
		answer the following questions.
		\begin{center}
			UZQSOVUOHXMOPVGPOZPEVSGZWSZOPFPESXUDBMETSXAIZ
			
			VUEPHZHMDZSHZOWSFPAPPDTSVPQUZWYMXUZUHSX
			
			EPYEPOPDZSZUFPOMBZWPFUPZHMDJUDTMOHMQ
		\end{center}
		\begin{enumerate}
			\item Determine the frequency of letters \textbf{P, Z, S, W} in the above cipher text.
			
				\begin{center}
					P $=13.33\%$ 

					Z $=11.67\%$ 

					S $=8.33\%$

					W $=3.33\%$
				\end{center}
			\item Find the equivalent plain text letters for \textbf{P, Z, S, W} based on their frequency.
				\begin{center}
					P &= E \\Z &= T \\S &= A \\W &= H \\
				\end{center}
		\end{enumerate}
	\item	Briefly explain what a brute-force attack is.\\\\
		A brute-force attack is where the attacker tries every possible key on the cipher text until
		the plain text is revealed.\\
	\item Briefly define the Playfair cipher\\\\
		The Playfair cipher is a multiple-letter encryption technique that uses a 5 by 5 matrix of
		non-repeating letters. Depending on how the plain text letter pairs are related on the matrix
		determines their cipher text equivalent.\\

	\item	Using this Playfair matrix:
	\begin{center}
		\begin{array}{|c|c|c|c|c|}
			\hline
			M & F & H & I/J & K \\
			\hline
			U & N & O & P & Q \\
			\hline
			Z & V & W & X & Y \\
			\hline
			E & L & A & R & G \\
			\hline
			D & S & T & B & C \\
			\hline
		\end{array}
	\end{center}
	Encrypt the message: MUST SEE YOU OVER CADOGAN WEST. COMING AT ONCE.\\\\
	Cipher text: UZTBDLGZPNNWLGTGTUEROVLDBDUHFPERHWQSRZ\\
	
	\item	Briefly explain a transition cipher.\\\\
		A transition cipher performs some sort of permutation on the plain text. This does not
		map a plain text letter to a cipher text letter, it just scrambles the message. Transposition
		ciphers can have multiple stages of transposition, making them more difficult the more
		stages there are.\\
	\item	Briefly explain steganography.\\\\
		Steganography conceals the existence of a message my hiding the message within other, less
		meaningful text. Can be done by somehow marking the appropriate letters of the hidden
		message so the receiver can identify them.\\
	\item	Alice and Bob are trying to securely communicate using a Vigenere Cipher technique with the
		same key. Given the key below answer the following questions.\\\\
		\textbf{Key}: DORIAN
		\begin{enumerate}
			\item Decrypt the following message that Alice has sent to Bob.\\\\
				\textbf{Cipher text}: WVV KRHFWRT PERPCMM VVB'K KRRDHZVG AHK AWBF.
				WVV KRHFWRT PERPCMM VV QIMAGLBX VEJ MCSA TUDH YCMNQG GMRSRFD
				JEGWSI BHNQ OCOOELHYUS.\\
				\textbf{Plain text}: THE CRUCIAL PROBLEM ISN'T CREATING JOBS. THE CRUCIAL
				PROBLEM IS CREATING JOBS THAT HUMANS PERFORM BETTER THAN ALGORITHMS\\
			\item	Encrypt the following message on behalf of Bob.\\\\
				\textbf{Plain text}: ITSTOOLATE\\
				\textbf{Cipher text}: LHJBOBOOKM\\\\\\\\
		\end{enumerate}
	\item	Suppose that the universe is defined as follows:\\
		\begin{equation}
			U = \{0, 1\}^5
		\end{equation}
		If the weight of all elements of this universe is uniformly distributed, calculate the probability
		of $A$ which is a subset of $U$, and  $A$ is defined as follows:
		\begin{equation}
			A &= \{ \textrm{all x} \in \textrm{U s.t. } lsb_2(x) &= 11 \} \subseteq $U$
		\end{equation}
		$A$ is every fourth element of $U$. $U$ has 32 total elements: $\frac{32}{4}=8$, thus there are
		8 elements in $A$.\\ Uniform distribution is given as:
		 \begin{equation}
			 \forall \  x \in U,\ P(x) = \frac{1}{|U|}
		\end{equation}
		Therefor the probability of $A$ is $\frac{8}{32} = \frac{1}{4}$ or $0.25$.
	\end{enumerate}
	
\end{document}

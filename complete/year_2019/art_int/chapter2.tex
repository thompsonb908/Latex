\documentclass[a4paper]{article}

\usepackage[utf8]{inputenc}
\usepackage[T1]{fontenc}
\usepackage{textcomp}
\usepackage[english]{babel}
\usepackage{amsmath, amssymb}


%figure support
\usepackage{import}
\usepackage{xifthen}
\pdfminorversion=7
\usepackage{pdfpages}
\usepackage{transparent}
\newcommand{\incfig}[1]{%
	\def\svgwidth{\columnwidth}
	\import{./figures/}{#1.pdf_tex}
}

\pdfsuppresswarningpagegroup=1

\begin{document}
	\author{Brandon Thompson}
	\title{Chapter 2 Notes}
	\maketitle
	\medskip
	
	The Chinese Room Experiment is as follows:
	\begin{itemize}
		\item An \textit{English speaking} human is placed in a room. The human
			has no comprehension to read, speak or understand Chinese.
		\item A set of cards showing printed Chinese symbols.
		\item A list of instructions written in English.

		\item[$-$] A story in \textit{Chinese} is fed into the room as well as a set of questions.
		\item[$-$] The human has to follow the instructions and cards to answer the questions.
	\end{itemize}
	If this system is set up and followed correctly, one would be able to fool an observer into
	believing that the person in the room knew Chinese, when in fact the human in the room gained
	no knowledge about the language. \\
	A computer program that behaves in an intelligent way does not produce an \textit{understanding}
	of the material, just an imitation.\\
	
	Artificial intelligence is all around us. Fuzzy logic is widely used in washing machines,
	cars and elevators. Intelligent agents are used to solve problems on our computers and
	search the internet for relevant documents.\\
	Robots are the physical embodiment of agents, we use them to explore places humans cannot
	go. Expert systems are used by doctors to diagnose/prescribe treatment. AI is also good
	for problems involving combinatorial solutions, like scheduling and travel planning.\\


\end{document}

\documentclass[a4paper]{article}

\usepackage[utf8]{inputenc}
\usepackage[T1]{fontenc}
\usepackage{textcomp}
\usepackage[english]{babel}
\usepackage{amsmath, amssymb}
\usepackage{tikz}

%figure support
\usepackage{import}
\usepackage{xifthen}
\pdfminorversion=7
\usepackage{pdfpages}
\usepackage{transparent}
\newcommand{\incfig}[1]{%
	\def\svgwidth{\columnwidth}
	\import{./figures/}{#1.pdf_tex}
}

\pdfsuppresswarningpagegroup=1

\begin{document}
	\title{Chapter 5: Advanced Search}
	\maketitle
	\begin{description}
		\item[meta-heuristic] is a heuristic to guide another heuristic, can be used to avoid local
			optimal solution in favor of global optimal solution.
		\item[meta-*] something to describe something else.
	\end{description}
	\section{Constraint Satisfaction Search}
	\subsection{The Eight-Queens Problem}
	\begin{itemize}
		\item 8 queens must be placed on a chess board in a way that no two queens are on the same
			row, column or diagonal.
		\item Columns are labeled $a \to h$.
		\item Rows are labeled $1\to 8$.
	\end{itemize}
	Brute force search is a search tree with 8 levels deep. The first level has a branching factor of 64.
	The second level has a branching factor of 63. The 8th level has a branching factor of 57.
	This is a very wide tree.\\
	The branching can be significantly reduced by knowing that each column and row can only have one queen.
	This reduces the branching factor to 8 for the first level and 7 for the second level, \ldots\\
	If we apply each of the constraints the search tree is further reduced by removing future choices
	that would not fit the constraints. This is known as \textbf{forward checking}.

	\section{Forward Checking}
	Can significantly improve the performance of solutions for CSP's. As each queen is placed on the board
	a forward checking mechanism is used to delete a set of impossible future choices. If placing a queen
	removes all future choices, the system will backtrack.\\
	A problem with chronological backtracking in the 8-queens problem is that the forward checking might
	backtrack moves that are not leading to errors.
	
	\section{Most-Constrained Variables}
	Heuristic for further performance improvement. At each stage of the search, work with the variable
	that has the least amount of possibilities. We can assign a value of 1-8 to variables  $a$-$h$ to
	represent the row (1-8) and column ($a$-$h$) of each queen. If the initial state is
	$a=1, b=3, c=5$, this leaves 3 choices for $d$, 3 choices for $e$, 1 choice for $f$, 3 choices for $g$
	and 4 choices for $h$. Then the next move should be in column  $f$ because it has the least possible
	choices.
	
	% insert image for chess board. Slide 17, chapter 5
	\begin{figure}[ht]
		\centering
		\incfig{cboard}
		\caption{$f$ is the most constrained variable.}
		\label{fig:cboard}
	\end{figure}	


	\section{Heuristic Repair}
	Used to improve the performance of finding solutions by generating a possible solution, either
	randomly or heuristic based, and altering the possible solution until it reaches the goal.\\
	In the eight-queens problem, can be done by using a mini-conflicts heuristic. To move from
	one state to a state that is likely closer to the goal, select a queen that has a conflict with 
	another queen, move this queen to a square where is conflicts with as few queens as possible, repeat
	this process until the goal is found.\\
	A starting configuration with only one queen per row and column.\\
	%include figures between slides 22 and 25
	
	This method can be expanded to solve the $n$-queens problem, the 1,000,000-queens problem can be
	solved in about 50 steps where traditional search trees would involve branching factors of $10^{12}$.
	
	\section{Local Search and Meta-heuristics}
	
	\section{Simulated Annealing}


\end{document}

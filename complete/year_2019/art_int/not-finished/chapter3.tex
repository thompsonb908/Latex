\documentclass[a4paper]{article}

\usepackage[utf8]{inputenc}
\usepackage[T1]{fontenc}
\usepackage{textcomp}
\usepackage[english]{babel}
\usepackage{amsmath, amssymb}


%figure support
\usepackage{import}
\usepackage{xifthen}
\pdfminorversion=7
\usepackage{pdfpages}
\usepackage{transparent}
\newcommand{\incfig}[1]{%
	\def\svgwidth{\columnwidth}
	\import{./figures/}{#1.pdf_tex}
}

\pdfsuppresswarningpagegroup=1

\begin{document}
	\author{Brandon Thompson}
	\title{Chapter 3 Notes}
	\maketitle
	\medskip
	
	For computers to solve problems in the real world, they need a way to
	\textit{represent} the real world internally. Computers represent 
	problems by the variables it uses and the operators applied to
	those variables. Good representation is vital for the search of solutions.\\

	\begin{description}
		\item[Semantic Nets] are graphs consisting of nodes that are
			connected by edges. Nodes represent objects. Links
			between nodes are relationships between objects.
			Links are labeled and directional to indicate the nature
			of the relationship.
		\item[Frames] are object-oriented representations that can be used
			to build expert systems. A set of frames are connected together
			by relations. Each frame describes an instance or a class. Each
			frame has one or more slots that are assigned slot values, each
			relationship is expressed by a value in a slot.
			
	\end{description}
\end{document}

In George Orwell's \textit{1984}, published in 1949, we see a war damaged world, fraught with conflict.
Constant surveillance is used by the government, under the lead of Big Brother, to purge anyone that does not fully conform to their regime.
The people caught by the Thought Police are forcefully \textquote{re-educated} to love Big Brother.
Many people see this as a dystopian novel centered on the surveillance of the population.
In contrast, \textit{The Circle} by Dave Eggers, 2013, takes a much more utopian standpoint.
The Circle company did not come about through force like in \textit{1984}, but by developing the best and most user-friendly tools.
The Circle advocates that \textquote{Everything should be known,} and accomplishes this through the guise of helping people.
My research will explore the ideas of technology, surveillance, privacy and transparency as presented in \textit{The Circle}, and how Eggers responds to the constant enhancements of technology.
I will conduct secondary research to include topics such as panopticism and the dangers of surveillance using other texts such as \textit{1984} by George Orwell and online sources.
This topic is important because at the rate that technology is advancing, we are not far from a situation where \textit{The Circle} could actually happen.
Large tech companies like Google and Facebook are already doing their best to track everything we do online under the guise of efficiency and selling our data for large profits.
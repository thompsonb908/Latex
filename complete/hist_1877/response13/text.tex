On October 29, 1929 over a period of 5 hours, roughly \$10 billion of capital value disappeared.
This was a result of bad trading the days leading up to the economic crash and panic as people tried to sell as many of their stocks as they could.
By 1932, the New York stock exchange has lost 90\% of its value.
Many people moved around to search for work.
Cities could no longer feed or house the poor communities and hunger became a large problem.
Newly homeless people lived in shantytowns called ''Hoovervilles'' on the outskirts of town.

Many people blamed themselves for their economic failure and did not turn to violence.
Confidence in corporations plummeted and some groups turned to protesting, usually uncoordinated and non-violent.
The ''Bonus Marchers'' were an exception to this, made up of unemployed veterans, they marched to Washington and set up a Hooverville to demand early payment of promised bonuses.
On July 29, 1932, after police had failed to remove the veterans, the Army evicted the marchers and burned down their shacks and belongings.

Hoover believed that government should not be a major part of the economy and that private organizations should assist the less fortunate.
Hoover and his advisors believed that the shock was because of regular business downturn and it would self correct.
Even after the situation did not improve, the government remained largely inactive, or even counter-productive.
Imposing tariffs on imported goods reduced international trade and hurt the economy because of the trade war that ensued.
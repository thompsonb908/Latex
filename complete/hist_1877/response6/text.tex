Between the late 1800's and early 1900's, there was an exponential growth of the urban population.
American cities were growing rapidly because of industrial growth.
Huge increases in the American economy were driven by manufacturing goods.
Companies were often clumped together in cities because of better access to shipping points and railroads. 
Also American farms were becoming more productive, and able to feed more people with less workers.
Cities start to become the backbone of the American economy and become more important over time.

Cities were enticing to many people, the country was fairly isolated and the chance for more social interaction was a good incentive for most people.
Cities offered more entertainment, if you wanted to go to the movies or something you would have to go to a city.
Cities had ,ore consumer goods and job opportunities, the concentration of factories and markets meant that there would be a lot of jobs available and it was easier to buy products.
Anonymity was also important for those that moved to cities, they had the chance to remake themselves without anyone knowing who they had been or what they had done before.

Cities were filled from mass immigration of people from Europe, Japan, China, and Mexico.
The majority of these immigrants were poor and uneducated.
Most of these people came with the intention of returning home but only a small amount of them did not.
Some immigrants traveled to flee religious persecution, instability or oppression, which is the idealistic American view.
However most immigrants came to America to make money.
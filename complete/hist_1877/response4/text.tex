The important effects of the Second Industrial Revolution were the shift from farming or artisanal jobs to industrial jobs, increased inequality a shifting to a consumer culture and economic instability.

The second industrial revolution created a new wealthy class of people who were far wealthier than any other before it.
By 1890, 1\% of the population owned more property than the other 99\%. Elitist withdrew from social contact with non elite people, forming social groups and areas poorer people were not allowed in. Their wealth was able to influence politics through lobbying and bribery to pass favorable laws. Lower class people began to think that corporations had to much 

The most important impact of the Second Industrial Revolution was the economic instability of the time.
Because production of to many goods flooded markets with more goods than americans could afford, there was financial panic and then depression as workers are laid off because companies were not making profit.
To ensure that companies made profits they formed three types of groups.
Pools were groups of companies within the same market that would fix the prices in all locations so that every group would make a profit.
Trusts were a combination of competing companies under the direction of a single leader to reduce competition.
The best attempt at creating a stable marketplace was vertical integration, where a company would acquire all of the methods in the production process, and other competing companies to form a monopoly of the industry.
This ensured that the company was able to acquire the resources to make the final product without relying on any other company.

This as important because it increased the nations production levels, and influenced the change to a consumerist culture that we have had ever since.
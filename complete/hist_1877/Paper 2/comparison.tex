Comparing these two sources shows Americas conflicting beliefs about entering the war.
On one hand, many Americans saw entering the war as a waste of manpower and resources that could be used in the defense of the nation, and on the other hand many believed that we should intervene in the atrocities that were being committed at the hands of the Nazis.
The America First Committee believed that the problems of other nations were not our problems and quoted the Monroe Doctrine and George Washington to appeal to traditionalists.
Lindbergh uses the geographical isolation of America in his arguments and says the problems that we face with transportation will be put on the Nazis if we are defending.
With the advancements of naval and air transportation, even the geographical isolation would not be an issue for much longer.

The Atlantic Charter advocates for interventionism, showing that it is our duty to ensure that no nation is threatening others, and that everyone has the right to choose the government they live in.
Even though there was a lot of pressure from isolationists, Roosevelt continued to assist nations targeted by the Axis in non-military ways.
The Atlantic Charter only outlined the ideals that America and Britain hoped to instill in the world after the war, but it had no impact in America joining the war or reducing isolationism in America.
The Conservative Coalition argued that individual freedom should be the primary goal of American life.
The core of Conservative beliefs was the freedom of association, state and local governments knew what was best more so than the federal government and that the federal government was getting too powerful.
These views appealed to a large variety of Americans.

Suburbs were one of the hotbeds of conservatism and advocated for individual liberty.
Urban-working class ethnics who were economically hurt by the economic crisis of the 70's and did not approve of the radicalism of the 60's were also in favor of conservatism.

The other reason for conservatism was to limit the federal government.
Antigovernment crusaders believed that the government was too strong.
Libertarians believed that the government should be limited in all cases were not prominent but still present.
Christian Right group were not opposed to the federal government but believed that they would be able to accomplish more of their goals through state and local governments.

The groups mentioned can fall into two categories, neoconservatives and christian conservatives.
Neoconservatives were intellectuals who had been liberals but had turned against liberalism in the 60's, believing that the radicalism had diminished morality and respect for authority.
Christian conservatives were mainly members of the middle class and they tried to reassert \textquote{traditional} religious values as america became more secular.
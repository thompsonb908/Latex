\documentclass[12pt]{article}
\usepackage[top=1in, bottom=1in, left=1in, right=1in]{geometry}
%\usepackage[margin=1in]{geometry}
\usepackage[onehalfspacing]{setspace}
%\usepackage[doublespacing]{setspace}
\usepackage{amsmath, amssymb, amsthm}
\usepackage{enumerate, enumitem}
\usepackage{fancyhdr, graphicx, proof, comment, multicol}
\usepackage[none]{hyphenat} % This command prevents hyphenation of words
\binoppenalty=\maxdimen % This command and the next prevent in-line equation breaks
\relpenalty=\maxdimen
%    Good website with common symbols
% http://www.artofproblemsolving.com/wiki/index.php/LaTeX%3ASymbols
%    How to change enumeration using enumitem package
% http://tex.stackexchange.com/questions/129951/enumerate-tag-using-the-alphabet-instead-of-numbers
%    Quick post on headers
% http://timmurphy.org/2010/08/07/headers-and-footers-in-latex-using-fancyhdr/
%    Info on alignat
% http://tex.stackexchange.com/questions/229799/align-words-next-to-the-numbering
% http://tex.stackexchange.com/questions/43102/how-to-subtract-two-equations
%    Text align left-center-right
% http://tex.stackexchange.com/questions/55472/how-to-make-text-aligned-left-center-right-in-the-same-line
\usepackage{microtype} % Modifies spacing between letters and words
\usepackage{mathpazo} % Modifies font. Optional package.
\usepackage{mdframed} % Required for boxed problems.
\usepackage{parskip} % Left justifies new paragraphs.
\linespread{1.1} 


%figure support
\usepackage{import}
\usepackage{xifthen}
\pdfminorversion=7
\usepackage{pdfpages}
\usepackage{transparent}
\newcommand{\incfig}[1]{%
	\def\svgwidth{\columnwidth}
	\import{./figures/}{#1.pdf_tex}
}
\graphicspath{ {./figures/} }
\pdfsuppresswarningpagegroup=1

\newenvironment{problem}[1]
{\begin{mdframed}[linewidth=0.8pt]
        \textsc{Problem #1:}

}
    {\end{mdframed}}

\newenvironment{solution}
    {\textsc{Solution:}\\}
    {\newpage}% puts a new page after the solution
    
\newenvironment{statement}[1]
{\begin{mdframed}[linewidth=0.6pt]
        \textsc{Statement #1:}

}
    {\end{mdframed}}

%\newenvironment{prf}
 %   {\textsc{Proof:}\\}
 %   {\newpage}% puts a new page after the solution

\usepackage{listings}
\usepackage{color}
\definecolor{pblue}{rgb}{0.13,0.13,1}
\definecolor{pgreen}{rgb}{0,0.5,0}
\definecolor{pred}{rgb}{0.9,0,0}
\definecolor{pgrey}{rgb}{0.46,0.45,0.48}

\lstset{language=Java,
	showspaces=false,
	showtabs=false,
	breaklines=true,
	showstringspaces=false,
	breakatwhitespace=true,
	commentstyle=\color{pgreen},
	keywordstyle=\color{pblue},
	stringstyle=\color{pred},
	basicstyle=\ttfamily,
	moredelim=[il][\textcolor{pgrey}]{\$\$},
	moredelim=[is][\textcolor{pgrey}]{\%\%}{\%\%}
}


\begin{document}
% This is the Header
% Make sure you update this information!!!!
\noindent
\textbf{COP4656.01} \hfill \textbf{Brandon Thompson} \\
\normalsize Prof. Topsakal \hfill Due Date: 4/5/2020 \\

% This is where you name your homework
\begin{center}
\textbf{Individual Assignment 2 Report}
\end{center}
In this assignment we had to modify the previous project (AboutMe) to retrieve \verb|ListView| data from an SQLite database instead of a string array, and pass user data with a \\ \verb|SharedPreferences| file instead of \verb|Intent.putExtra()|.

Table structure is as follows:\\
\begin{center}
	\verb|CREATE TABLE SONGS(_id INTEGER PRIMARY KEY AUTOINCREMENT, NAME TEXT);|
	\verb|CREATE TABLE COURSES(_id INTERGER PRIMARY KEY AUTOINCREMENT, NAME TEXT);|
\end{center}
The tables \verb|SONG| and \verb|COURSE| have columns \verb|id| and \verb|NAME|.

We store the list of songs in the \verb|SONGS| table and the list of courses in the \verb|COURSES| table. Later we use the rows to fill in the \verb|ListView| in the appropriate activities.

To retrieve data from the \verb|COURSE| and \verb|SONG| tables I implemented a \verb|getAllCourses()| and \verb|getAllSongs()| methods in the \verb|DatabaseHelper| file (code below), this returns a list of the \verb|Course| or \verb|Song| class that holds the \verb|id| and \verb|NAME| values of the items. Then I created an \verb|ArrayList<String>| from the list of Courses and Songs that works well with the \verb|ArrayAdapter| class to populate the \verb|ListView|

To pass the user data between Activities, I initialized a \verb|SharedPreferences| object in all the Activities that need to send/receive data (MainActivity, SongActivity, CourseActivity), then depending on what CheckBoxes are clicked, and what RadioButton is selected, it will store the correct combination of name, favorite song, and favorite course with the \verb|SharedPreferencesEditor| object.

In the receiving activities, you only need a \verb|SharedPrefernces| object and use the\\ \verb|getString(<key>)| method to retrieve the stored data. This method call replaces the \verb|Intent.putExtra(<key>)| method and because they both should retrieve Strings, the code worked without changing anything else.

	\begin{lstlisting}
public class DatabaseHelper extends SQLiteOpenHelper {
    private static DatabaseHelper sInstance;
    public static final int DATABASE_VERSION = 1;
    public static final String DATABASE_NAME = "data_db";
    public static final String TABLE_NAME1 = "course";
    public static final String TABLE_NAME2 = "song";
    public static final String COLUMN_ID = "id";
    public static final String COLUMN_NAME = "name";
    public static final String CREATE_TABLE1 =
        "CREATE TABLE " + TABLE_NAME1 + "("
        + COLUMN_ID + " INTEGER PRIMARY KEY AUTOINCREMENT,"
        + COLUMN_NAME + " TEXT"
        + ")";
    public static final String CREATE_TABLE2 =
        "CREATE TABLE " + TABLE_NAME2 + "("
        + COLUMN_ID + " INTEGER PRIMARY KEY AUTOINCREMENT,"
        + COLUMN_NAME + " TEXT"
        + ")";

    public static synchronized DatabaseHelper 
    getInstance(Context context) {
        if (sInstance == null) {
            sInstance = new DatabaseHelper(
	        context.getApplicationContext());
        }
        return sInstance;
    }

    private DatabaseHelper(Context context) {
        super(context, DATABASE_NAME, null, DATABASE_VERSION);
    }

    @Override
    public void onCreate(SQLiteDatabase db) {
        db.execSQL(CREATE_TABLE1);
        db.execSQL(CREATE_TABLE2);
    }

    @Override
    public void onUpgrade(
        SQLiteDatabase db, int oldVersion, int newVersion) {
	db.execSQL("DROP TABLE IF EXISTS " + TABLE_NAME1);
        db.execSQL("DROP TABLE IF EXISTS " + TABLE_NAME2);
        onCreate(db);
    }

    public long insertCourse(String name) {
        SQLiteDatabase db = this.getWritableDatabase();
        ContentValues values = new ContentValues();
        values.put(COLUMN_NAME, name);
        long id = db.insert(TABLE_NAME1,null,values);
        db.close();
        return id;
    }

    public long insertSong(String name) {
        SQLiteDatabase db = this.getWritableDatabase();
        ContentValues values = new ContentValues();
        values.put(COLUMN_NAME, name);
        long id = db.insert(TABLE_NAME2,null,values);
        db.close();
        return id;
    }

    public List<Course> getAllCourses() {
        List<Course> courses = new ArrayList<>();
        String selectQuery =
	    "SELECT * FROM " + TABLE_NAME1 + " ORDER BY " +
                COLUMN_NAME;
        SQLiteDatabase db = this.getWritableDatabase();
        Cursor cursor = db.rawQuery(selectQuery, null);
        if (cursor.moveToFirst()) {
            do {
                Course course = new Course();
                course.setId(cursor.getInt(
		    cursor.getColumnIndex(COLUMN_ID)));
                course.setName(cursor.getString(
		    cursor.getColumnIndex(COLUMN_NAME)));
                courses.add(course);
            } while (cursor.moveToNext());
        }
        db.close();
        return courses;
    }

    public List<Song> getAllSongs() {
        List<Song> songs = new ArrayList<>();
//        String where = null;
        String selectQuery = 
	    "SELECT * FROM " + TABLE_NAME2 + " ORDER BY " +
                COLUMN_NAME;
        SQLiteDatabase db = this.getWritableDatabase();
        Cursor cursor = db.rawQuery(selectQuery, null);
        if (cursor.moveToFirst()) {
            do {
                Song song = new Song();
                song.setId(cursor.getInt(
		    cursor.getColumnIndex(COLUMN_ID)));
		song.setName(cursor.getString(
		    cursor.getColumnIndex(COLUMN_NAME)));
                songs.add(song);
            } while (cursor.moveToNext());
        }
        db.close();
//        return cursor;
//        Testing for returning Cursor instead of List
        return songs;
    }
}
	\end{lstlisting}

\end{document}

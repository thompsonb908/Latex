The 1970s is know as the end of the post-World War II \textquote{golden age} because of the economy, the Vietnam War, Americans faith in government, and the state of the environment after the second industrial revolution.

After World War II, many manufacturing plants moved overseas because of the lower labor costs and access to cheaper materials.
With the closing of plants, mass layoffs occurred and corporations stopped promising life-time employment and stopped giving private pensions to its retired employees.

America lost its first war, leaving questions about American power in the world.
The Vietnam War was highly opposed by the American public, shady political dealings were made to enter the war to fight communism.
In the end, the US withdrew from South Vietnam and allowed North Vietnam to keep the territory it had won during the war.
As soon as the US left, North Vietnam initiated the war and took control of South Vietnam.

Faith in government reach a new low as Nixon's criminal activity was revealed and the abuses of Cold War presidents became common knowledge.
During the presidential election of 1972, five former employees of Nixon's campaign broke into the Democratic Party's headquarters.
This chain of events lead to the investigation of Nixon and the revealing of bugging opponents, using the IRS to harass opponents and other abuses of power.
The involvement of America in Vietnam started in World War II when America supported communist guerrillas fighting Japan.
After the war, the US supported the re-imposition of French rule to prevent the communist rule of Vietnam.

Ho Chi Minh and the communist party were extremely popular in Vietnam and were considered the legitimate government by most Vietnamese.
Instead of understanding the politics and opinions of Vietnam, the US believed that communists were trying to take over the country and must step in to save the country for democracy and freedom.

Presidents Kennedy and Johnson feared that if they stopped supporting the Vietnam war, they would be blamed for its fall to communism both domestically and by foreign allies.
On August 2, 1964 an American vessel spying on North Vietnam was attacked, Congress passed the \textquote{Gulf of Tonkin Resolution,} authorizing the president \textquote{to take all necessary steps including the use of armed force} to free Vietnam from the communists.

During Johnson's 164 presidential campaign, Johnson pledged that he would not send troops to Vietnam.
Johnson and his administration had already planned to send troops before his re-election.
The lying of the US government with its involvement and intentions with the war made the public feel as if they had been tricked into participating in it.
These issues cause people to begin scrutinizing the government much more thoroughly.
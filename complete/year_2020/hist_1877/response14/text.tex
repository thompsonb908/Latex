FDR's goal while he was in office was to avoid a revolution that would topple the capitalistic economy he was fond of.
He did this through his 'New Deal' policies, the first of which were a series of acts to strengthen the banking system.
FDR was able to stop the banking crisis, build public works, and expanded governments role in regulating the economy.
Some of the acts implemented were considered unconstitutional and revoked but many still exist today.
Despite all of the progress made, the first New Deal did not end the Depression, but was meant to prevent them later.

The second New Deal was a reaction to the failure of the first New Deal to end the Depression, and the demands of unions.
The goal of the second New Deal was economic security.
The second New Deal did this by creating new jobs, protected the rights of American Workers, and provided citizens with permanent economic security.
3 million people were employed by The Works Progress Administration, focusing on public works to build roads and painting murals.
The Wagner Act outlawed firing or blacklisting labor union sympathizers, and placed the power of the federal government behind the labor unions.
The Social Security Act paid the unemployed until they found a new job, gave pensions to the elderly that could no longer work.

The first New Deal focused on the banking system and preventing more Depressions while the seconds New Deal was more focused on the workers and supporting citizens that could not work.
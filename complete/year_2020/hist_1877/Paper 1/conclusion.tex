After the Civil War, there was a large increase in the power that former slaves had on politics in the south.
Southerners were loath to give up the control they had held for so long and resorted to trickery and violence to reassert antebellum views.
The first document "Jourdon Anderson Writes His Former Enslaver, 1865" showcases the earliest form of whites trying to control newly released slaves with visions of owning and running their own farm.
This resulted in ever increasing debt to their former masters that required them to continue working.
Later the Black Codes enforced more extreme measures of asserting white supremacy.
Namely, blacks could not congregate, carry weapons or be without proof of a job.
Any of these and some others would be grounds for conviction and if they could not pay the fine, they were sold as convict laborers.
These methods reinstated white supremacy during the Reconstruction and placed blacks back into economic and socially subservient positions.
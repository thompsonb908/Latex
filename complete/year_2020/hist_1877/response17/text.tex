During World War I, people of ethnic backgrounds were strongly encouraged to assimilate.
The nations that made up the Axis powers believed in a racist regime that promoted the racial supremacy of their culture.
Germany believed that Germans would ultimately dominate all other races, Japan believed that their racial purity made them superior to other races.

Axis propaganda used the United State's hypocrisy of enslaving blacks and unfair treatment of ethnic groups.
African American soldiers in Italy were given pamphlets to try and convince them to surrender, saying that this was not their fight or that Japan would treat them better than America would.

America, in response to the claims of hypocrisy, promoted the \textquote{Four Freedoms}.
America was still a very racist nation at the time.
War-time propaganda focused on America's ethnic diversity and pluralism, a strong contradiction to WWI.

During the war, FDR declared \textquote{There never have been, there isn't now, and there never will be any race of people on the earth fit to serve as masters over their fellow men.}
Nonwhite Americans used the war to make claims for greater rights.
China became a major ally against Japan.
This caused the Chinese Exclusion Act to be repealed, and challenged racism towards Chinese Americans.
Many blacks hoped  that service and loyalty would lead to greater freedom.
Black leaders began to aggressively push for their rights, marking the beginning of the modern Civil Rights Movement.
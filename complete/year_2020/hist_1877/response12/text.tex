Huge increase in manufacturing.
more urbanized and greater access to consumer goods.


The basis for the notion that the 1920's was a period of excess was the rapid economic growth.
The 1920's was the most prosperous time period during the history of the United States.
The largest area of growth was the automobile industry.
Cars were the consumer good that was most desired by Americans and was probably the most expensive good that a family would buy.
During the 1920's automobile production tripled, making cars the most economically important consumer good.
The growth of the car industry led to growth in many other industries like steel, rubber, oil, roads and tourism.

Because of the amount of prosperity in the market, peoples views about corporations shifted from being bad for society to being good for society.
This consolidated recent consumer views for society.
In 1920 30\% of households had electricity, 60\% of households had electricity in the 1930's.
The availability of electricity led to a high demand for labor saving devices like washing machines and vacuum cleaners.

Movies became more popular with the rise of Hollywood, and the introduction of sound to movies.
Americans also bought radios as they became cheaper and more available.
These forms of entertainment created a unified, American popular culture centered around consumer goods.
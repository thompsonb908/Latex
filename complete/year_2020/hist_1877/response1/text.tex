I do not think that race had a huge impact on America's
founding principles, the Declaration of Independence has
no explicit
neglect for a particular race. There was no verbiage
stating that a particular group of people should be
excluded from the rights defined in the Declaration
of Independence. 

The statement "All men are created equal" in the preamble of
the Declaration of Independence clearly states that all
men are equal in the eyes of god and so it should be in
government, granting men their unalienable rights to life,
liberty and the pursuit of happiness. This however is not
the case for slaves and native americans, being sold between
owners and forced to work for no wages. This has
changed as our views about slavery and race has improved
throughout the years.

However The first nationalization law is more descriptive about the
qualifications to be adopted as an American citizen. 
The law stated "any free white person," followed by other
criteria means that only white skinned persons could be considered
as citizens and thus be granted the rights of an American. This
document does directly exclude a large amount of people but
speculating on why this document was worded this way was probably
to ensure that only wealthy landowning individuals of good esteem 
are able to participate in the politics of the country.

I believe that any
prejudice against a race came from the interpretations
of the time period, Native and African americans were not
thought highly of because of their perceived lack of intelligence
and the founders were not keen on changing the way things
had been done for many years before that.
\documentclass[12pt]{article}
\usepackage[top=1in, bottom=1in, left=1in, right=1in]{geometry}
%\usepackage[margin=1in]{geometry}
\usepackage[onehalfspacing]{setspace}
%\usepackage[doublespacing]{setspace}
\usepackage{amsmath, amssymb, amsthm}
\usepackage{enumerate, enumitem}
\usepackage{fancyhdr, graphicx, proof, comment, multicol}
\usepackage[none]{hyphenat} % This command prevents hyphenation of words
\binoppenalty=\maxdimen % This command and the next prevent in-line equation breaks
\relpenalty=\maxdimen
%    Good website with common symbols
% http://www.artofproblemsolving.com/wiki/index.php/LaTeX%3ASymbols
%    How to change enumeration using enumitem package
% http://tex.stackexchange.com/questions/129951/enumerate-tag-using-the-alphabet-instead-of-numbers
%    Quick post on headers
% http://timmurphy.org/2010/08/07/headers-and-footers-in-latex-using-fancyhdr/
%    Info on alignat
% http://tex.stackexchange.com/questions/229799/align-words-next-to-the-numbering
% http://tex.stackexchange.com/questions/43102/how-to-subtract-two-equations
%    Text align left-center-right
% http://tex.stackexchange.com/questions/55472/how-to-make-text-aligned-left-center-right-in-the-same-line
\usepackage{microtype} % Modifies spacing between letters and words
\usepackage{mathpazo} % Modifies font. Optional package.
\usepackage{mdframed} % Required for boxed problems.
\usepackage{parskip} % Left justifies new paragraphs.
\linespread{1.1} 


%figure support
\usepackage{import}
\usepackage{xifthen}
\pdfminorversion=7
\usepackage{pdfpages}
\usepackage{transparent}
\newcommand{\incfig}[1]{%
	\def\svgwidth{\columnwidth}
	\import{./figures/}{#1.pdf_tex}
}
\graphicspath{ {./figures/} }
\pdfsuppresswarningpagegroup=1

\newenvironment{problem}[1]
{\begin{mdframed}[linewidth=0.6pt]
        \textsc{Problem #1:}

}
    {\end{mdframed}}

\newenvironment{solution}
    {\textsc{Solution:}\\}
    {\newpage}% puts a new page after the solution
    
\newenvironment{statement}[1]
{\begin{mdframed}[linewidth=0.6pt]
        \textsc{Statement #1:}

}
    {\end{mdframed}}

%\newenvironment{prf}
 %   {\textsc{Proof:}\}
 %   {\newpage}% puts a new page after the solution



\begin{document}
% This is the Header
% Make sure you update this information!!!!
\noindent
\textbf{CIS 4367.01} \hfill \textbf{Brandon Thompson} \\
\normalsize Prof. Elibol \hfill Due Date: 1/15/20 \\

% This is where you name your homework
\begin{center}
\textbf{Homework 2}
\end{center}
	\begin{problem}[\#1]
		Write a one-paragraph objective summary of the main ideas in Chapter 1.
	\end{problem}
	\begin{solution}
		Assets of a computer system are hardware, software, data and communication lines.
		These assets are vulnerable to attacks in different ways depending on how the
		assets are threatened. Consequences of threats could be unauthorized disclosure
		of information, authorized entity gets false data and believes it is true,
		disruption of correct system services or control of system services passed to
		an unauthorized entity. Threats can be prevented by use of countermeasures.
		The best way to protect a system is to have a strategy. This strategy consists of
		a policy for what the plan is supposed to do, how the strategy is implemented,
		and assurance for if the plan will work.
	\end{solution}

	\begin{problem}[\#2]
		Compare and contrast the different approaches to characterizing computer
		security in Sections 1.3 and 1.4.
	\end{problem}
	\begin{solution}
		Section 1.3 is functional requirements that a system should address in order to
		provide the proper level of security for federal information. Where section
		1.4 lists fundamental security design principles that limit what users can
		access based on the minimum required for the user to work.
	\end{solution}
\end{document}

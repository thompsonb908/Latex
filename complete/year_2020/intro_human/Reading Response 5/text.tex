The Circle by Dave Eggers is a utopian/dystopian style book about the perfect tech company that makes everyone's lives so much easier.
The catch for using any of the Circle's tools was that you had to do so as your actual self, or the single unified account (TruYou) that the Circle had created.
The effect of this is that there is no anonymity online anymore, everything you do or say online is tracked by the TruYou profile.
\textquote{Overnight, all comment boards became
civil, all posters held accountable. The trolls, who had more or less overtaken the
internet, were driven back into the darkness.} \autocite[16]{eggers1}

Later in the book, with the unveiling of SeeChange, a small, cheap,  wireless camera with incredible resolution, the cameras are hidden around areas of human rights violations.
Bailey describes a soldier, that, never knowing for sure if he is being watched, will be to scared of the accountability to commit an act of violence.
\textquote{You take the average soldier who’s now worried that a dozen cameras will catch him, for all eternity, dragging some woman down the street? Well, he should worry. He should worry about these cameras. He should worry about SeeChange.} \autocite[40]{eggers1}
The belief is that these cameras will reduce crime rates by 70 or 80 percent, much like when TruYou was released, the trolls of the internet dispersed, now the trolls of the real world will as well.
The Circle is attempting to turn society into a type of panopticon, where people will not know if they are being watched so they will not do anything against the law.
The presentation ends with Bailey asking his mother to install the cameras in her house, so that he can keep an eye on her in her old age.
She refuses and Bailey installs the camera anyways, a huge violation of privacy, laughed off by Bailey's charisma and eloquence.

Another interesting observation is that during the presentation, Bailey enforces the need for the cameras with a quote \textquote{I insist that all that happens should be known.}
After Bailey compares the releasing of the cameras to the Dawn of the Second Enlightenment, the quote changes to \textquote{All that happens must be known.}
Then, at the end of the presentation, with the applause thundering through the room, Mae whispers \textquote{All that happens will be known.} \autocite[42]{eggers1}
This method of persuasion is probably used for all of the Circle's new inventions to make people believe that their new technology is not an extreme violation of any human right, but is instead protecting human rights.
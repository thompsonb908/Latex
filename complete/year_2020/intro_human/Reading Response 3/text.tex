{\it Player Piano} by Kurt Vonnegut warns that it "is not a book about what is, but a book about what could be," \cite{Kurt}(11) living in a totalitarian society that mascaraed as a utopia.
The setting is much like the main character Paul Proteus.
Paul has an ideal life, that many would trade for in an instant.
He is the highest paid worker at Ilium Works, married to a beautiful and charming woman, and has clear ambitions to fill his fathers shoes as the "National, Industrial, Commercial, Communications, Foodstuffs, and Resources Director".
Later in the novel we start to see issues both with the way that society has developed and Paul's increasing doubts about what he wants from life.

Paul begins on his journey as a dystopian protagonist after he meets Rudy Hertz, a former machinist at Ilium Works, at a bar in The Homestead where non-managers or engineers live.
Rudy plays a song on the player piano for Paul and points out that it looks like "you can almost see the ghost playing his heart out."
This is a metaphor for Rudy and the whole book where people's movements have been automated by machines. 

Paul's friend Finnerty is an eccentric genius that is not content with his life or job.
Finnerty is a rebel against the way of life of the engineers and managers.
He does not wash or change his clothes often, he will bring women from The Homestead to parties instead of a "classy" wife, and he quit one of the best jobs available.
Paul believes Finnerty's way of life is a slight against how engineers and managers have come to view themselves as better that everyone else.
Finnerty plays an important role for Paul, he is the confident go-getter and is not afraid to voice his opinions, unlike Paul.

On a separate plot line, the Shah, a visiting spiritual leader from another country keeps comparing the general population to slaves, because in his culture there are only the elite, and slaves.
The purpose of the separate plot line is to show the difference between an insider of the system and one that is looking at it from a distance.
To engineers and managers, they are living in a utopia where they provide for the majority and are rewarded for it.
As an outsider, the Shah laughs at the Army and the Reeks and Recks because they are slaves, they just did not know it.
\textit{The Circle} by Dave Eggers represents both a utopian and dystopian society.
In the beginning of the novel, The Circle is a beautiful campus described as \textquote{Heaven} by Mai saying that she \textquote{never wanted to be anywhere else}.
Her fist day they give her a lot of gadgets and nice equipment for her position and she is flattered by the amount of money and quality the Circle puts into its employees.
The utopian aspect is further revealed by the number of people that want to work for and with the Circle, making the Circle the center of everything.

The tools that the Circle develops aid the utopian view with underlying dystopian tones.
SeeChange is released as a way to watch problematic areas for human rights violations and as a way to keep an eye on your possessions and family members.
While the technology was developed with good intent, the saturation of society with public access cameras creates a type of peer to peer panopticon.
Further developments improve the quality of life of its user base while stripping them of more and more freedoms.

There are two people in particular that oppose the constant and mostly unnecessary innovations of the Circle.
The first is Mercer who opposed the Circle from the beginning, wishing to lead a private life away from the Circles influences.
Mercer writes very eloquent letters to Mai describing the dangers of her contribution to the Circle.
Ultimately Mercer decides to take his own life rather than live in a panoptic society.
Kalden or Ty is also opposed to the Circles dominance over personal data, however Ty is the creator of most of this technology and believes that if the Circle reaches completion, there will be no way to go back.
Overall, the Circle has the unwavering support of most of the population on account of having the best tools and being free.
Feeling that privacy is a small price to pay for the quality of life improvements, many people do not think twice about using the Circles tools.
For those that do not wish to partake, the Circle eventually got large enough that avoidance was no longer possible.

\textit{The Circle} shows how a utopia to one person could be a dystopia to another and that we should be scared of these events happening in real life.
\documentclass[12pt]{article}
\usepackage[top=1in, bottom=1in, left=1in, right=1in]{geometry}
%\usepackage[margin=1in]{geometry}
\usepackage[onehalfspacing]{setspace}
%\usepackage[doublespacing]{setspace}
\usepackage{amsmath, amssymb, amsthm}
\usepackage{enumerate, enumitem}
\usepackage{fancyhdr, graphicx, proof, comment, multicol}
\usepackage[none]{hyphenat} % This command prevents hyphenation of words
\binoppenalty=\maxdimen % This command and the next prevent in-line equation breaks
\relpenalty=\maxdimen
%    Good website with common symbols
% http://www.artofproblemsolving.com/wiki/index.php/LaTeX%3ASymbols
%    How to change enumeration using enumitem package
% http://tex.stackexchange.com/questions/129951/enumerate-tag-using-the-alphabet-instead-of-numbers
%    Quick post on headers
% http://timmurphy.org/2010/08/07/headers-and-footers-in-latex-using-fancyhdr/
%    Info on alignat
% http://tex.stackexchange.com/questions/229799/align-words-next-to-the-numbering
% http://tex.stackexchange.com/questions/43102/how-to-subtract-two-equations
%    Text align left-center-right
% http://tex.stackexchange.com/questions/55472/how-to-make-text-aligned-left-center-right-in-the-same-line
\usepackage{microtype} % Modifies spacing between letters and words
\usepackage{mathpazo} % Modifies font. Optional package.
\usepackage{mdframed} % Required for boxed problems.
\usepackage{parskip} % Left justifies new paragraphs.
\linespread{1.1} 


%figure support
\usepackage{import}
\usepackage{xifthen}
\pdfminorversion=7
\usepackage{pdfpages}
\usepackage{transparent}
\newcommand{\incfig}[1]{%
	\def\svgwidth{\columnwidth}
	\import{./figures/}{#1.pdf_tex}
}
\graphicspath{ {./figures/} }
\pdfsuppresswarningpagegroup=1

\newenvironment{problem}[1]
{\begin{mdframed}[linewidth=0.8pt]
        \textsc{Problem #1:}

}
    {\end{mdframed}}

\newenvironment{solution}
    {\textsc{Solution:}\\}
    {\newpage}% puts a new page after the solution
    
\newenvironment{statement}[1]
{\begin{mdframed}[linewidth=0.6pt]
        \textsc{Statement #1:}

}
    {\end{mdframed}}

%\newenvironment{prf}
 %   {\textsc{Proof:}\\}
 %   {\newpage}% puts a new page after the solution

\begin{document}
% This is the Header
% Make sure you update this information!!!!
\noindent
\textbf{CIS4367.01} \hfill \textbf{Brandon Thompson} \\
\normalsize Prof. Elibol \hfill Due Date: 4/2/2020 \\

% This is where you name your homework
\begin{center}
\textbf{Homework 14}
\end{center}
	\begin{problem}{\#7.1}
		In order to implement a classic DoS flood attack, the attacker must generate a sufficiently large volume of packets to exceed the capacity of the link to the target organization. Consider an attack using ICMP echo request (ping) packets that are 100 bytes in size (ignoring framing overhead). How many of these packets per second must the attacker send to flood a target organization using a 8-Mbps link? How many per second if the packets are 1000 bytes in size? Or 1460 bytes?
	\end{problem}
	\begin{solution}
		Link speed: $8Mb = 8\times 10^{6}$\\
		Packet size: $100 = 1\times 10^{2}$ \\
		\\
		$\dfrac{8\times 10^6}{1\times 10^2} = 8\times 10^4$ packets per second or \\
		\\
		$8\times 10^4 = 80,000$ packets per second

		With packet size: 1000 bytes\\
		$\dfrac{8\times 10^6}{1\times 10^3} = 8\times 10^3$ packets per second or,\\
		\\
		$8\times 10^3 = 8000$ packets per second.\\
		\\
		With packet size: 1460 bytes\\
		\\
		$\dfrac{8\times 10^6}{1.46 \times 10^3} = 5.48 \times 10^3$ packets per second or,\\
		\\
		$5.43\times 10^3 = 5430$ packets per second.
	\end{solution}

	\begin{problem}{\#7.2}
		Using a TCP SYN spoofing attack, that attacker aims to flood the table of TCP connection requests on a system so that it is unable to respond to legitimate connection requests. Consider a server system with a table for 256 connection requests. This system will retry sending the SYN-ACK packet five times when it fails to receive an ACK packet in response, at 30 second intervals, before purging the request from its table. Assume no additional countermeasures are used against this attack and the attacker has filled this table with an initial flood of connection requests. At what rate must the attacker continue to send TCP connection requests to this system in order to ensure that the table remains fill? Assuming the TCP SYN packet is 40 bytes in size (ignoring framing overhead), how much bandwidth does the attacker consume to continue this attack?
	\end{problem}
	\begin{solution}
		Every 30 seconds the server will try to send the SYN-ACK packet for 5 tries. $5 \times 30 = 150$ seconds.\\
		Then the attacker has to send a packet every 150 seconds to replace the lost table row.\\
		The TCP packet is 40 bytes which is $40\times 8 = 320$ bits.\\
		which will use 2.13 bits per second to continue the attack.
	\end{solution}

	\begin{problem}{\#7.3}
		Consider a distributed variant of the attack we explore in Problem 7.1. Assume the attacker has compromised a number of broadband-connected residential PCs to use as zombie systems. Also assume each such system has an average uplink capacity of 256kbps. What is the maximum number of 100-byte ICMP echo request packets a single zombie PC can send per second? If the packet size is 1000 bytes? Or 1500 bytes? How many such zombie systems would the attacker need to flood the target organization using an 8-Mbps link? Given reports of botnets composed of many thousands of zombie systems, what can you conclude about their controller's ability to launch DDos attacks on multiple such organizations simultaneously? Or on a major organization with multiple, much larger network links than we have considered in this problem?
	\end{problem}
	\begin{solution}
		100 bytes is equal to $100 \times 8 = 800$ bits, divided by the uplink capacity of the unit: $\dfrac{256\text{kbps}}{800\text{b}} = 2560$ packets per second.\\
		If the packet size was 1000 bytes then:\\
		$\dfrac{256\text{kbps}}{8000\text{b}} = 256$ packets per second.\\
		If the packet size was 1500 bytes then:\\
		$\dfrac{256\text{kbps}}{12000\text{b}} = 170.7$ packets per second.\\
		Given an 8Mbps link the number of zombies with 256kbps link to flood the organization is:\\
		$\dfrac{8\text{Mbps}}{256\text{kbps}} = 31.25 \approx 32$ zombies.\\
		\\
		If a botnet is sufficiently large, and the organization is not protected from such attacks, the botnet is capable of crippling any organization they would like to.
	\end{solution}

	\begin{problem}{\#7.4}
		In order to implement a DNS amplification attack, the attacker must trigger the creation of a sufficiently large volume of DNS response packets from the intermediary to exceed the capacity of the link to the target organization. Consider an attack where the DNS response packets are 100 bytes in size (ignoring framing overhead). How many of these packets per second must the attacker trigger to flood a target organization using an 9-Mbps link? If packet size is 1000 bytes? Or 1500 bytes? If the DNS request packet to the intermediary is 70 bytes in size, how much bandwidth does the attacker consume out of the 8-Mbps link to send the necessary rate of DNS request packets?
	\end{problem}
	\begin{solution}
	The number of response packets of $100\text{ bytes}\times 8 = 800 $ bits to overload a 9 Mbps connection is $\dfrac{9\text{Mbps}}{800\text{b}} = 11250$ packets per second.\\
	If the response packet is 1000 bytes (8000 bits) then to overload a 9Mbps connection there needs to be $\dfrac{9\text{Mbps}}{8000\text{b}} = 1125$ response packets per second.

	If the response packet is 1500 bytes (12000 bits) then to overload a 9Maps connection there needs to be $\dfrac{9\text{Mbps}}{12000\text{b}} = 750$ response packets per second.

	Given a request packet of 70 bytes, the amount of bandwidth needed to overflow a 8Mbps link would depend on the number of response packets generated. 
	\end{solution}

	\begin{problem}{\#7.5}
		It is discussed that an amplification attack, which is a variant of reflection attack, can be launched by using any type of a suitable UDP service, such as the echo service. However, TCP services cannot be used in this attack. Why?
	\end{problem}
	\begin{solution}
		Because TCP services are connection oriented, meaning TCP requests cannot be directed towards a broadcast address.
	\end{solution}

	\begin{problem}{\#7.7}
		Assume a future where security countermeasures against DoS attacks are much more widely implemented than at present. In this future network, antispoofing and directed broadcast filters are widely deployed. In addition, the security of PCs and workstations is much greater, making the creation of botnets difficult. Do the administrators of server systems still have to be concerned about, and take further countermeasures against, DoS attacks? If so, what types of attack can still occur, and what measures can be taken to reduce their impact?
	\end{problem}
	\begin{solution}
		Yes, the slowloris attack could still be effective against web servers because it is using legitimate traffic. The protection against this is to limit the rate of incoming connections from a host. Flash crowds or flash events could still happen though there is not much that can be done to prevent this from occurring. 
	\end{solution}
\end{document}

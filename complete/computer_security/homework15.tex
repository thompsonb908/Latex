\documentclass[12pt]{article}
\usepackage[top=1in, bottom=1in, left=1in, right=1in]{geometry}
%\usepackage[margin=1in]{geometry}
\usepackage[onehalfspacing]{setspace}
%\usepackage[doublespacing]{setspace}
\usepackage{amsmath, amssymb, amsthm}
\usepackage{enumerate, enumitem}
\usepackage{fancyhdr, graphicx, proof, comment, multicol}
\usepackage[none]{hyphenat} % This command prevents hyphenation of words
\binoppenalty=\maxdimen % This command and the next prevent in-line equation breaks
\relpenalty=\maxdimen
%    Good website with common symbols
% http://www.artofproblemsolving.com/wiki/index.php/LaTeX%3ASymbols
%    How to change enumeration using enumitem package
% http://tex.stackexchange.com/questions/129951/enumerate-tag-using-the-alphabet-instead-of-numbers
%    Quick post on headers
% http://timmurphy.org/2010/08/07/headers-and-footers-in-latex-using-fancyhdr/
%    Info on alignat
% http://tex.stackexchange.com/questions/229799/align-words-next-to-the-numbering
% http://tex.stackexchange.com/questions/43102/how-to-subtract-two-equations
%    Text align left-center-right
% http://tex.stackexchange.com/questions/55472/how-to-make-text-aligned-left-center-right-in-the-same-line
\usepackage{microtype} % Modifies spacing between letters and words
\usepackage{mathpazo} % Modifies font. Optional package.
\usepackage{mdframed} % Required for boxed problems.
\usepackage{parskip} % Left justifies new paragraphs.
\linespread{1.1} 


%figure support
\usepackage{import}
\usepackage{xifthen}
\pdfminorversion=7
\usepackage{pdfpages}
\usepackage{transparent}
\newcommand{\incfig}[1]{%
	\def\svgwidth{\columnwidth}
	\import{./figures/}{#1.pdf_tex}
}
\graphicspath{ {./figures/} }
\pdfsuppresswarningpagegroup=1

\newenvironment{problem}[1]
{\begin{mdframed}[linewidth=0.8pt]
        \textsc{Problem #1:}

}
    {\end{mdframed}}

\newenvironment{solution}
    {\textsc{Solution:}\\}
    {\newpage}% puts a new page after the solution
    
\newenvironment{statement}[1]
{\begin{mdframed}[linewidth=0.6pt]
        \textsc{Statement #1:}

}
    {\end{mdframed}}

%\newenvironment{prf}
 %   {\textsc{Proof:}\\}
 %   {\newpage}% puts a new page after the solution

\begin{document}
% This is the Header
% Make sure you update this information!!!!
\noindent
\textbf{CIS4367.01} \hfill \textbf{Brandon Thompson} \\
\normalsize Prof. Elibol \hfill Due Date: 4/2/2020 \\

% This is where you name your homework
\begin{center}
\textbf{Homework 15}
\end{center}
	\begin{problem}{\#1}
		Prepare a one paragraph objective summary of the main ideas in this chapter (Chapter 7).
	\end{problem}
	\begin{solution}
		Chapter 7 is about denial of service (DoS) attacks, which is an attempt to use all of a systems critical resources (network, system, application) to compromise the availability of the system. With the bandwidth increase of the Internet DDoS attacks are becoming more potent (600 Gbps in an attack on BBC in 2015). These attacks are simple to setup, difficult to stop, and very effective. Generally these attacks will try and overload a network connection by using a higher-capacity network, or spoof a sending address of a TCP connection to take advantage of the three way handshake. In order for an attacker to generate more network bandwidth than the target, they use a botnet of compromised systems to send requests to the target. Reflection attacks and amplification attacks lessen the load for the attacker by multiplying the requests to the target through DNS servers.
	\end{solution}

	\begin{problem}{\#2}
		Which of the main topics of this chapter (Chapter 7) do you think is hardest to understand? Explain.
	\end{problem}
	\begin{solution}
		Amplification attacks were the most difficult for me because I am not familiar with how a broadcast address works and the best defense against the amplification and reflection attacks are to block spoofed addresses and I am not sure how one would do that.
	\end{solution}
\end{document}

\section{NMAP Option List}

Usage: \verb|nmap [Scan Type(s)] [Options] {target specification}|

\begin{itemize}
    \item TARGET SPECIFICATION: Can pass hostnames, IP addresses, networks, etc.
    \begin{itemize}
        \item \verb|-iL <inputfilename>| Input from list of hosts/network
        \item \verb|-iR <num hosts>| Choose random targets
        \item  \verb|--exclude <host1[,host2][,host3],...>| Exclude hosts/networks
        \item \verb|--excludefile <exclude_file>| Exclude list from file
    \end{itemize}
    \item HOST DISCOVERY
    \begin{itemize}
        \item \verb|-sL| List scan - simply list all targets to scan
        \item \verb|-sn| Ping scan - disable port scan
        \item \verb|-Pn| Treat all hosts as online -- skip host discovery
        \item \verb|-PS/PA/PU/PY[portlist]| TCP SYN/ACK, UDP, or SCTP discovery to given ports
        \item \verb|-PE/PP/PM| ICMP echo, timestamp, and netmask request discovery probes
        \item \verb|-PO[protocol list]| IP Protocol Ping
        \item \verb|-n/-R| Never do DNS resolution/Always resolve [default: sometimes]
        \item \verb|--dns-servers <serv1[,serv2],...>| Specify custom DNS servers
        \item \verb|--system-dns| Use OS's DNS resolver
        \item \verb|--traceroute| Trace hop path to heach host
    \end{itemize}
    \item SCAN TECHNIQUES
    \begin{itemize}
        \item \verb|-sS/sT/sA/sQ/sM| TCP SYN/Connect()/ACK/Window/Maimon scans
        \item \verb|-sU| UDP scan
        \item \verb|-sN/sF/sX| TCP Null, FIN, and Xmas scans
        \item \verb|--scanflags <flags>| Customize TCP scan flags
        \item \verb|-sI <zombie host[:probeport]>| Idle scan
        \item \verb|-sY/sZ| SCTP INIT/COOKIE-ECHO scans
        \item \verb|-sO| IP protocol scan
        \item \verb|-b <FTP relay host>| FTP bounce scan
    \end{itemize}
    \item PORT SPECIFICATION AND SCAN ORDER
    \begin{itemize}
        \item \verb|-p <port ranges>| Only scan specified ports ex. \verb|-p22|; \verb|-p-65535|; \verb|-p U:53,111,137,T:21-25,80,139,8080,S:9|
        \item \verb|--exclude-ports <port ranges>| Exclude the specified ports from scanning
        \item \verb|-F| Fast mode - Scan fewer ports than the default scan
        \item \verb|-r| Scan ports consecutively - don't randomize
        \item \verb|--top-ports <number>| Scan \verb|<number>| most common ports
        \item \verb|--port-ratio <ratio>| Scan ports more common than \verb|<ratio>| 
    \end{itemize}
    \item SERVICE/VERSION DETECTION
    \begin{itemize}
        \item \verb|-sV| Probe open ports to determine service/version info
        \item \verb|--version-intensity <level>| Set from 0 (light) to 9 (try all probes)
        \item \verb|--version-light| Limit to most likely probes (intesity 2)
        \item \verb|--version-all| Try every single probe (intensity 9)
        \item \verb|--version-trace| show detailed version scan activity (for debugging)
    \end{itemize}
    \item SCRIPT SCAN
    \begin{itemize}
        \item \verb|-sC| equivalent to \verb|--script=default|
        \item \verb|--script=<Lua scripts>| \verb|<Lua scripts>| is a comma separated list of directories, script files or script-categories
        \item \verb|--script-args=<n1=v1,[n2=v2,...]>| provide arguments to scripts
        \item \verb|--script-args-file=filename| provide NSE script args in a file
        \item \verb|--script-trace| Show all data sent and received
        \item \verb|--script-updatedb| Update the script database.
        \item \verb|--script-help=<Lua scripts>| Show help about scripts. <Lua scripts> is a comma-separated list of script-files or script-categories.
    \end{itemize}
    \item OS DETECTION
    \begin{itemize}
        \item \verb|-O| Enable OS detection
        \item \verb|--osscan-limit| Limit OS detection to promising targets
        \item \verb|--osscan-guess| Guess OS more aggressively
    \end{itemize}
    \item TIMING AND PERFORMANCE\\
    Options which take \verb|<time>| are in seconds, or append 'ms' (milliseconds), 's' (seconds), 'm' (minutes) or 'h' (hours) to the value (e.g. 30m)
    \begin{itemize}
        \item \verb|-T<0-5>|: Set timing template (higher is faster)
        \item \verb|--min-hostgroup/max-hostgroup <size>|: Parallel host scan group sizes
        \item \verb|--min-parallelism/max-parallelism <numprobes>|: Probe parallelization
        \item \verb|--min-rtt-timeout/max-rtt-timeout/initial-rtt-timeout <time>|: Specifies probe round trip time.
        \item \verb|--max-retries <tries>|: Caps number of port scan probe retransmissions.
        \item \verb|--host-timeout <time>|: Give up on target after this long
        \item \verb|--scan-delay/--max-scan-delay <time>|: Adjust delay between probes
        \item \verb|--min-rate <number>|: Send packets no slower than <number> per second
        \item \verb|--max-rate <number>|: Send packets no faster than <number> per second
    \end{itemize}
    \item FIREWALL/IDS EVASION AND SPOOFING
    \begin{itemize}
        \item \verb|-f; --mtu <val>|: fragment packets (optionally w/given MTU)
        \item \verb|-D <decoy1,decoy2[,ME],...>|: Cloak a scan with decoys
        \item \verb|-S <IP_Address>|: Spoof source address
        \item \verb|-e <iface>|: Use specified interface
        \item \verb|-g/--source-port <portnum>|: Use given port number
        \item \verb|--proxies <url1,[url2],...>|: Relay connections through HTTP/SOCKS4 proxies
        \item \verb|--data <hex string>|: Append a custom payload to sent packets
        \item \verb|--data-string <string>|: Append a custom ASCII string to sent packets
        \item \verb|--data-length <num>|: Append random data to sent packets
        \item \verb|--ip-options <options>|: Send packets with specified ip options
        \item \verb|--ttl <val>|: Set IP time-to-live field
        \item \verb|--spoof-mac <mac address/prefix/vendor name>|: Spoof your MAC address
        \item \verb|--badsum|: Send packets with a bogus TCP/UDP/SCTP checksum
    \end{itemize}
    \item OUTPUT
    \begin{itemize}
        \item \verb|-oN/-oX/-oS/-oG <file>: Output scan in normal, XML, s|<rIpt kIddi3,
        and Grepable format, respectively, to the given filename.
     \item \verb|-oA <basename>|: Output in the three major formats at once
     \item \verb|-v|: Increase verbosity level (use -vv or more for greater effect)
     \item \verb|-d|: Increase debugging level (use -dd or more for greater effect)
     \item \verb|--reason|: Display the reason a port is in a particular state
     \item \verb|--open|: Only show open (or possibly open) ports
     \item \verb|--packet-trace|: Show all packets sent and received
     \item \verb|--iflist|: Print host interfaces and routes (for debugging)
     \item \verb|--append-output|: Append to rather than clobber specified output files
     \item \verb|--resume <filename>|: Resume an aborted scan
     \item \verb|--noninteractive|: Disable runtime interactions via keyboard
     \item \verb|--stylesheet <path/URL>|: XSL stylesheet to transform XML output to HTML
     \item \verb|--webxml|: Reference stylesheet from Nmap.Org for more portable XML
     \item \verb|--no-stylesheet|: Prevent associating of XSL stylesheet w/XML output
    \end{itemize}
    \item MISC
    \begin{itemize}
        \item \verb|-6|: Enable IPv6 scanning
        \item \verb|-A|: Enable OS detection, version detection, script scanning, and traceroute
        \item \verb|--datadir <dirname>|: Specify custom Nmap data file location
        \item \verb|--send-eth/--send-ip|: Send using raw ethernet frames or IP packets
        \item \verb|--privileged|: Assume that the user is fully privileged
        \item \verb|--unprivileged|: Assume the user lacks raw socket privileges
        \item \verb|-V|: Print version number
        \item \verb|-h|: Print this help summary page.
    \end{itemize}
\end{itemize}
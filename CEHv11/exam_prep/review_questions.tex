\section{Review Questions}

\subsection{Scanning}
\begin{itemize}
    \item In the SYN scan; Nmap will send a SYN message to the target. What is the response if the port is open or closed?
    \begin{enumerate}
        \item Open: A SYN/ACK packet
        \item Closed A RST packet
        \item Filtered: No response given
    \end{enumerate}
    \item Active banner grabbing techniques used by an attacker to determine the OS running on a remote target system.
    \begin{itemize}
        \item TCP Sequence ability test
        \item Port Unreachable
    \end{itemize}
    \item Passive banner grabbing techniques:
    \begin{itemize}
        \item Banner grabbing from error messages
        \item Sniffing the network traffic
        \item Banner grabbing from page extensions
    \end{itemize}
    \item Countermeasures to prevent information disclosure through banner grabbing
    \begin{itemize}
        \item \textbf{Display false banners to mislead or deceive attackers.}
        \item Turn off unnecessary services on the network host to limit information disclosure.
        \item Disabling open relay feature protect from SMTP enumeration.
        \item Disabling the DNS zone transfers to untrusted hosts protect from DNS enumeration
        \item Restricting anonymous access through RestrictNullSessAccess parameter from the Windows Registry protects from SMB enumeration.
    \end{itemize}
\end{itemize}

\subsection{Chapter 4: Enumeration}
\subsubsection{Enumeration Concepts}
\begin{enumerate}
    \item Which of the following enumeration techniques does an attacker take advantage of different error messages generated during the service authentication process?
    \begin{itemize}
        \item \textbf{Brute-force Active Directory}
        \item Extract usernames using SNMP: attackers guess read-only or read-write community strings by using SNMP API to extract usernames.
        \item Extract usernames using email IDs: emails contain a username and a domain name.
        \item Extract information using default passwords
    \end{itemize}
\end{enumerate}

\subsection{Chapter 5: System Hacking}
\subsubsection{Escalating Privaleges}
\begin{itemize}
    \item A pen tester is using Metasploit to exploit an FTP server and pivot to a LAN. How will the pen tester pivot using Metasploit?
    \begin{itemize}
        \item Create a route statement in the meterpreter.
    \end{itemize}
\end{itemize}
\subsubsection{Maintaining Access}


\subsection{Sniffing}
\subsubsection{Sniffing Concepts}
\begin{itemize}
    \item Which of the following OSI layers do sniffers operate and perform an initial compromize?
    \begin{itemize}
        \item Data link layer: the second layer of the OSI model. Data packets are encoded and decoded into bits. OSI layers are designed to work independently of each other; thus if a sniffer sniffs data in the data link layer, the upper OSI layers will not be aware of the sniffing.
    \end{itemize}
    \item Which of the following techniques is also a type of netowkr protocol used for PNAC that is used to defend against MAC address spoofing and to enforce access control at the point where a user joins a network.
    \begin{itemize}
        \item \textbf{Implementation of IEEE 802.1X Suites}: This is a type of network protocol for port-based Network Access control (PNAC), and its main purpose is to enforce access control at the point where a user joins the network.
        \item DCHP Snooping Binding Table: DHCP snooping process filters untrusted DHCP messages and helps to build and bind a DHCP binding table. This table contains the MAC address, IP address, lease time, binding type, CLAN number, and interface information to correspond with untrusted interfaces of a switch. It acts as a firewall between untrusted hosts and DHCP servers. It also helps in differentiating between trusted and untrusted interfaces.
        \item Dynamic ARP Inspection: The system checks the IP-MAC address binding for each ARP packet in a network. While performing a DAI, the system will automatically drop invalid IP-MAC address binding.
        \item IP Source Guard: IP source gurad is a security feature in switches that restricts the IP traffic on untrusted layer 2 ports by filtering traffic based on the DHCP snooping binding database. It prevents spoofing attacks when the attacker tries to spoof or use the IP address of another host. 
    \end{itemize}
    \item Which of the following Cisco switch port configuration commands is used to enter a secure MAC address for the interface and the maximum number of secure MAC addresses?
    \begin{itemize}
        \item \textbf{\verb|switchport port-security mac-address mac_address|} Enters a secure MAC address for the interface. You can use this command to enter the maximum number of secure MAC addreses.
        \item \verb|switchport port-security limit rate invalid-source-max|: sets the rate limit for bad packets.
        \item \verb|switchport port-security maximum value|: Sets the maximum number of secure MAC addresses for the interface. The range is 1 to 3072; the default is 1.
        \item \verb|switchport port-security mac-address sticky| Enables sticky learning on the interface.
    \end{itemize}
    \item Which of the following techniques enables devices to detect the existence of unidirectional links and disable the affected interfaces in the network, in addition to causing STP topology loops.
    \begin{itemize}
        \item \textbf{UDLD (Unidirectional Link Detection)}: def in question
        \item BPDU Guard: BPDU guard must be enabled on the ports that should never recieve a BPDU from their connected devices. This is used to avoid the transimission of BPDUs on PortFast-enabled ports. This feature helps in preventing potential bridging loops in the network.
        \item Root Guard: Protects the root bridge and ensures that it remains as the root in the STP topology. It forces the interfaces to become the designaged ports (forwarding ports) to prevent the nearby switches from becoming root switches.
        \item Loop Guard: Loop guard improves the stability of the network by preventing it against the bridging loops. it is generally used to protect against a malformed switch.
    \end{itemize}
    \item Which of the following IPv4 DHCP packet fields inculdes random number chosen by a client to associate request messages and their responses between the client and server?
    \begin{itemize}
        \item Opcode: 1 octet, contains the message opcode that represents the message type: opcode "1" represents messages sent by the client, while "2" represents responses sent by the server.
        \item \textbf{Transaction ID (XID)}: 4 Octets, a random number is chosen by the client to associate the request messages and their responses between a client and a server.
        \item Flags: 2 octets, Flags set by the client; For example, if the client cannot recieve unicast IP datagrams, then the broadcast flag is set.
        \item Server Name (SNAME): 64 octets, Optional server hostname.
    \end{itemize}
    \item Which of the following IOS global commands verifies the DHCP snooping configuration?
    \begin{itemize}
        \item \textbf{\verb|show ip dhcp spoofing|}: Verifies the configuration.
        \item \verb|ip dhcp snooping|: Enables DHCP snooping globally.
        \item \verb|ip dhcp snooping trust|: Configures the interface as trusted or untrusted.
        \item \verb|no ip dhcp snooping information option|: To disable the insertion and the removal of the option-82 field, use the no ip dhcp snooping information option in global configuration command.
    \end{itemize}
    \item In which of the following attacks does an attacker send spoofed router advertisement messages so that all the data packets travel through thri system to collect valuble information and launch MITM and DoS attacks?
    \begin{itemize}
        \item \textbf{IRDP Spoofing}: An attacker can use this to send spoofed router advertisement messages so that all the data packets travel through the attacker;s system. Thus, the attacker can sniff the traffic and collect valuble information from the data packets. Attackers can use IRDP spoofing to launch MITM, DoS and passive sniffing attacks.
        \item MAC Spoofing: in this attack, the attacker first retrieves the MAC addresses of clients who are actively associated with the switch port. Then, the attacker spoofs a MAC address with the MAC address of the legitimate client. If the spoofing is successful, then the attacker can recieve all the traffic destined for the client. Thus, an attacker can gain access to the network and take over the identity of someone on the network.
        \item ARP Spoofing Attack: ARP spoofing is a method of attacking an Ethernet LAN. When a legitimate user initiates a session with another user in the same layer 2 broadcast domain, the switch broadcasts an ARP request using the recipient's IP address, while the sender waits for the recipient to respond with a MAC address. An attacker eavesdropping on this unprotected layer 2 broadcast domain can respond to the broadcast ARP request and replies to the sender by spoofing the intended recipient's IP address.
        \item STP Attack: If an attacker has access to two switches, he/she introduces a rogue switch in the network with a priority lower than any other switch in the network. This makes the rogue switch the root bridge, thus allowing the attacker to sniff all the traffic flowing in the network.
    \end{itemize}
    \item In one of the following techniques, an attacker must be connected to a LAN to sniff packets, and on successful sniffing, they can send a malicious reply to the sender before the actual DNS server.
    \begin{itemize}
        \item \textbf{Intranet DNS Spoofing}: An attacker can perform an intranet DNS spoofing attack on a switched LAN with the help of the ARP poisoning technique. To perform this attack, the attacker must be connected to the LAN and be able to sniff the traffic or packets. An attacker who succeeds in sniffing the ID of the DNS request from the intranet can send a malicious reply to the sender before the actual DNS server.
        \item  DNS Cache poisoning: DNS cache poisoning refers to altering or adding forged DNS records in the DNS resolver cache so that a DNS query is redirected to a malicious site. The DNS system uses cache memory to hold the recently resolved domain names.
        \item Proxy Server DNS Poisoning: In the proxy server DNS poisoning technique, the attacker sets up a proxy server on the attacker's system. The attacker also configures a fraudulent DNS and makes its IP address a primary DNS entry in the proxy server. The attacker changes the proxy server settings of the victim with the help of a Trojan. The proxy serves as a primary DNS and redirects the victim's traffic to the fake website, where the attacker can sniff the confidential information of the victim and then redirect the request to the real website.
        \item Internet DNS Spoofing: Internet DNS poisoning is also known as remote DNS poisoning. Attackers can perform DNS spoofing attacks on a single victim or on multiple victims anywhere in the world. To perform this attack, the attacker sets up a rogue DNS server with a static IP address.
    \end{itemize}
    \item Which of the following is not a mitigation technique against MAC address spoofing?
    \begin{itemize}
        \item \textbf{DNS security (DNSSEC)}: Implement Domain Name System Security Extension to prevent DNS spoofing attacks.
        \item 
    \end{itemize}
\end{itemize}

\subsubsection{Sniffing Tools and countermeasures}
\begin{itemize}
    \item What is the correct pcap filter to capture all transmission control protocol (TCP) traffic going to or from host 192.168.0.125 on port 25?
    \begin{itemize}
        \item tcp.port == 25 and ip.addr == 192.168.0.125
    \end{itemize}
\end{itemize}

\subsection{Social Engineering}
\subsubsection{Social Engineering Concepts}
\begin{itemize}
    \item Mat, a software engineer, received an email from his colleague John, stating that project files were missing from his system and asking Mat to send them to his personal email. Mat was suspicious and called John on his personal number. To his surprise, John replied that he has never written an email recently to Mat.\\ Which of the following types of attacks was Mat subjected to?
    \begin{itemize}
        \item \textbf{Intimidation}
    \end{itemize}
\end{itemize}
\subsubsection{Social Engineering Techniques}
\begin{itemize}
    \item A consultant is hired to do a physical penetration test at a large financial company. On the first day of his assessment, the consultant goes to the company's building dressed as an electrician and waits in the lobby for an employee to pass through the main access gate, and then the consultant follows the employee behind to get into the restricted area. Which type of attack did the consultant perform?
    \begin{itemize}
        \item \textbf{Tailgating} implies access to enter into the building or secured area without the consent of the authorized person. It is the act of following an authorized person through a secure entrance, as when a polite user opens and then holds the door for those following. An attacker wears a fake badge and attempts to enter a secured area by closely following an authorized person through a door requiring key access. He/she can then try to get into restricted areas by pretending to be an authorized person.
    \end{itemize}
\end{itemize}


\subsection{Denial-of-Service}
\subsubsection{DoS/DDoS Concepts}
\begin{itemize}
    \item 
\end{itemize}
\subsubsection{DoS/DDoS Attack Techniques and Tools}
\begin{itemize}
    \item When a client's computer is infected with malicious software which connects to the remote computer to receive commands, the remote computer is called ___________.
    \begin{itemize}
        \item Answer is C&C, which will instruct the Bot what to do. When a client's computer is infected with malicious software which connects to the remote computer to receive commands, the remote computer is called C&C. Bot and Botnet respectively represent infected computer and network of the infected computers managed by C&C and server is not used in this terminology.
    \end{itemize}
    \item The DDoS tool used by anonymous in the so-called Operation Payback is called _______.
    \begin{itemize}
        \item LOIC is the first version of the tool and it was used in Operation Payback. HOIC is the second version of the tool with some additional features, and it was used in the Operation Megaupload. BanglaDos and Dereil do not have direct connection with anonymous group.
    \end{itemize}
\end{itemize}
\subsubsection{DoS/DDoS Protection Tools and Countermeasures}
\begin{itemize}
    \item What is the DoS/DDoS countermeasure strategy to at least keep the critical services functional?
    \begin{itemize}
        \item Degrading services: During an attack, if it is not possible to keep all the services functioning, then it is a good idea to keep at least the critical services functional. To do this, first, identify the critical services and then customize the network, systems, and application designs to cut down on the noncritical services. This may help you to keep the critical services functional.
    \end{itemize}
    \item Ivan works as security consultant at “Ask Us Intl.” One of his clients is under a large-scale volume-based DDoS attack, and they have to decide how to deal with the issue. They have some DDoS appliances that are currently not configured. They also have a good communication channel with providers, and some of the providers have fast network connections. In an ideal scenario, what would be the best option to deal with this attack. Bear in mind that this is a volume-based DDoS attack with at least 1 000 000 bots sending the traffic from the entire globe!
    \begin{itemize}
        \item The answer is “Absorb the attack,” since this is a really large volume of traffic, and using additional capacity (DDoS appliances that are currently not configured) to absorb the attack. Most of the other options are not practically feasible. Blocking the traffic at the provider level is a viable option, but in this case, since the attack cannot be easily filtered (Since the traffic coming from the entire globe), this is not an apt solution. Filtering the traffic at the provider level is the same thing as blocking the traffic at the provider level, so this is not a correct answer and filtering the traffic at the company's Internet facing routers option will not work because the traffic is already there, and in this case, it is impossible to do anything at the client's site.
    \end{itemize}
    \item John's company is facing a DDoS attack. While analyzing the attack, John has learned that the attack is originating from the entire globe, and filtering the traffic at the Internet Service Provider's (ISP) level is an impossible task to do. After a while, John has observed that his personal computer at home was also compromised similar to that of the company's computers. He observed that his computer is sending large amounts of UDP data directed toward his company's public IPs.

    John takes his personal computer to work and starts a forensic investigation. Two hours later, he earns crucial information: the infected computer is connecting to the C&C server, and unfortunately, the communication between C&C and the infected computer is encrypted. Therefore, John intentionally lets the infection spread to another machine in his company's secure network, where he can observe and record all the traffic between the Bot software and the Botnet. After thorough analysis he discovered an interesting thing that the initial process of infection downloaded the malware from an FTP server which consists of username and password in cleartext format. John connects to the FTP Server and finds the Botnet software including the C&C on it, with username and password for C&C in configuration file. What can John do with this information?
    \begin{itemize}
        \item The correct answer is “neutralize handlers,” because with admin's access to C&C John can stop the attack, disable the C&C software, and/or change the password to stop the DDoS attack on his company's network. Deflect the attack and mitigate the attack are not the correct answers because in both these cases, he is literally stopping the attack. Protect secondary victims is not the correct answer because secondary victims are still infected.
    \end{itemize}
    \item After successfully stopping the attack against his network, and informing the CERT about the Botnet and new password which he used to stop the attack and kick off the attackers from C&C, John starts to analyze all the data collected during the incident and creating the so-called “Lessons learned” document. What is John doing?
    \begin{itemize}
        \item John is trying the postattack forensics in order to learn how to fight this type of attacks in the future. John is not trying to neutralize the handlers because this requires some type of access to C&C, which was already done, and he is not trying to prevent potential attacks and protect secondary victims—this was already done in previous steps.
    \end{itemize}
\end{itemize}


\subsection{Session Hijacking}
\subsubsection{Session Hijacking Concepts}
\begin{itemize}
    \item 
\end{itemize}
\subsubsection{Application Level Session Hijacking}"
\begin{itemize}
    \item
\end{itemize}
\subsubsection{Network Level Session Hijacking}
\begin{itemize}
    \item In order to hijack TCP traffic, an attacker has to understand the next sequence and the acknowledge number that the remote computer expects. Explain how the sequence and acknowledgment numbers are incremented during the 3-way handshake process.
    \begin{itemize}
        \item During the 3-way handshake, sequence and acknowledgment numbers are (relatively) incremented by one. After that acknowledge number will be incremented for the size of the packet received.
    \end{itemize}
    \item Maira wants to establish a connection with a server using the three-way handshake. As a first step she sends a packet to the server with the SYN flag set. In the second step, as a response for SYN, she receives packet with a flag set.

    Which flag does she receive from the server?
    \begin{itemize}
        \item In the second step, the server sends a response to her with the SYN + ACK flag and an ISN (Initial Sequence Number) for the server.
        In the third step, Maira sets the ACK flag acknowledging the receipt of the packet and increments the sequence number by 1.
    \end{itemize}
\end{itemize}
\subsubsection{Session Hijacking Tools}
\begin{itemize}
    \item Marin was using sslstrip tool for many yearsagainst most of the websites, like Gmail, Facebook, Twitter, etc. He was supposed to give a demo on internet (in)security and wanted to show a demo where he can intercept 302 redirects between his machine and Gmail server. But unfortunately it does not work anymore. He tried the same on Facebook and Twitter and the result was the same. He then tried to do it on the company OWA (Outlook Web Access) deployment and it worked! He now wants to use it against Gmail in his demo because CISO thinks that security through obscurity is a best way to a secure system (obviously BAD CISO) and demonstrating something like that on company live system is not allowed. How can Marin use sslstrip or similar tool to strip S from HTTP?
    \begin{itemize}
        \item HSTS protection is basically the cookie that the website issues to the web browser, when user visits the website for the first time. It's long term cookie, which means that it will not expire. If the cookie is set - web browser prevents visiting the website over HTTP connection. So, by using sslstrip+ with dnsspoof module, one can effectively combat the protection if the user NEVER visited this website before. That's why he has to use IE in InPrivate browsing mode because it will not read the HSTS cookie. This is NOT the case with Firefox or Chrome though!
        SslstripHSTS tool does not exist.
    \end{itemize}
\end{itemize}
\subsubsection{Session Hijacking Countermeasures}
\begin{itemize}
    \item Which of the following countermeasures should be followed to defend against session hijacking?
    \begin{itemize}
        \item Use HTTP Public Key Pinning (HPKP) to allow users to authenticate web servers
    \end{itemize}
    \item Which of the following techniques mitigates the risk of ARP spoofing and other session hijacking attacks caused when using a hub network?
    \begin{itemize}
        \item \textbf{Switch Netowrk}: Mitigates the risk of ARP spoofing and other session hijacking attacks.
    \end{itemize}
    \item Which of the following techniques protects the client–server communication against session hijacking attacks by creating a public–private key pair for every connection to a remote server?
    \begin{itemize}
        \item Token Binding
    \end{itemize}
\end{itemize}


\subsection{Evading IDS, Firewalls, and Honeypots}
\subsubsection{IDS, IPS, Firewall and Honeypot Concepts}
\begin{itemize}
    \item Which of the following attributes in a packet can be used to check whether the packet originated from an unreliable zone?
    \begin{itemize}
        \item Source IP address
    \end{itemize}
    \item What is the main advantage that a network-based IDS/IPS system has over a host-based solution?
    \begin{itemize}
        \item They do not use host system resources. Host-based intrusion detection systems (IDSes) protect just that: the host or endpoint. This includes workstations, servers, mobile devices and the like. Host-based IDSes are not just one of the last layers of defense, but they're also one of the best security controls because they can be fine-tuned to the specific workstation, application, user role or workflows required. A network-based IDS often sits on the ingress or egress point(s) of the network to monitor what's coming and going. Given that a network-based IDS sits further out on the network, so it doesn't use any host system resources and it may not provide enough granular protection to keep everything in check -- especially for network traffic that's protected by SSL, TLS or SSH.
    \end{itemize}
    \item Which of the following is a hardware requirement that either an IDS/IPS system or a proxy server must have in order to properly function?
    \begin{itemize}
        \item They must be Dual-homed. Dual-homed devices have two interfaces; a public interface that directly connected to the Internet and a private interface connected to the Intranet. It is a hardware requirement that either an IDS/IPS system or a proxy server must have in order to properly function. The bastion host is an example of dual-homed system designed for defending the network against attacks. It acts as a mediator between inside and outside networks. A bastion host is a computer system designed and configured to protect network resources from attack. Traffic entering or leaving the network passes through the firewall.
    \end{itemize}
    \item Which of the following descriptions is true about a static NAT?
    \begin{itemize}
        \item A static NAT uses a one-to-one mapping.
    \end{itemize}
    \item Jamie needs to keep data safe in a large datacenter, which is in desperate need of a firewall replacement for the end of life firewall. The director has asked Jamie to select and deploy an appropriate firewall for the existing datacenter. The director indicates that the amount of throughput will increase over the next few years and this firewall will need to keep up with the demand while other security systems do their part with the passing data. What firewall will Jamie use to meet the requirements?
    \begin{itemize}
        \item Performance is the key focus of the question; therefore, the test taker will have to focus on the real need of the most enterprise businesses and not get distracted by other slower firewall types. Packet filtering firewall may seem old school to less experienced test takers and they may immediately choose other options.
        Packet filtering firewalls are best performing of the choices.
    \end{itemize}
    \item When analyzing the IDS logs, the system administrator notices connections from outside of the LAN have been sending packets where the source IP address and destination IP address are the same. However, no alerts have been sent via email or logged in the IDS. Which type of an alert is this?
    \begin{itemize}
        \item False Negative
    \end{itemize}
    \item When analyzing the IDS logs, the system administrator noticed an alert was logged when the external router was accessed from the administrator’s computer to update the router configuration. What type of an alert is this?
    \begin{itemize}
        \item False Positive
    \end{itemize}
    \item At which two traffic layers do most commercial IDSes generate signatures? (Select Two)
    \begin{itemize}
        \item According to New 'semantics-aware' IDS reduces false positives (https://searchsecurity.techtarget.com/news/1113940/New-semantics-aware-IDS-reduces-false-positives), https://www.sanfoundry.com/computer-networks-questions-answers-entrance-exams/, and https://searchsecurity.techtarget.com/quiz/Quiz-IDS-IPS, the most commercial IDSes generate signatures at the network and transport layers.
    \end{itemize}
\end{itemize}
\subsubsection{IDS, IPS, Firewall, and Honeypot Solutions}
\begin{itemize}
    \item When an alert rule is matched in a network-based IDS like snort, the IDS does which of the following:
    \begin{itemize}
        \item   Continues to evaluate the packet until all rules are checked
    \end{itemize}
\end{itemize}
\subsubsection{Evading IDS}
\begin{itemize}
    \item How many bit checksum is used by the TCP protocol for error checking of the header and data and to ensure that communication is reliable?
    \begin{itemize}
        \item  16-bitwa
    \end{itemize}
    \item An attacker hides the shellcode by encrypting it with an unknown encryption algorithm and by including the decryption code as part of the attack packet. He encodes the payload and then places a decoder before the payload. Identify the type of attack executed by attacker.
    \begin{itemize}
        \item Polymorphic Shellcode
    \end{itemize}
\end{itemize}
\subsubsection{Evading Firewalls}
\begin{itemize}
    \item 
\end{itemize}
\subsubsection{Honeypot, IDS, and Firewall Evasion Countermeasures}
\begin{itemize}
    \item 
\end{itemize}



\subsection{General}
\begin{itemize}
    \item Where does Microsoft Windows store authentication credentials and passwords?
    \begin{enumerate}
        \item \verb|C:\windows\system32\config|
    \end{enumerate}
    \item What netstat command will you use if you want to display all connections and listening ports, with addresses and port numbers in numerical form?
    \begin{enumerate}
        \item \verb|netstat -an|
    \end{enumerate}
    \item What type of rootkit uses system-level calls to hide their existence?
    \begin{enumerate}
        \item Library Level rootkit (user-level), replaces or modifies the functionality of system calls to the operating system.
    \end{enumerate}
    \item \begin{tabular}[H]{|l|l|l|}
        \hline
        \textbf{Issue} & \textbf{Solution} & \textbf{Notest}\\
        \hline
        Telnet, rlogin & Secure Shell (SSH) or openSSH & Sends encrypted data and makes it difficult for an attacker to send correctly encrypted data if a session is hijacked.\\
        \hline
        Any remote connection & Virtual Private Network (VPN) & Implementing encrypting VPNs such as PPTP, layer 2 protocol tunneling (L2PT), and IPsec, for remote connections prevents session hijacking.\\
        \hline
        Server message Block (SMB) & SMB Signing & Improves the security of the SMB protocol and reduces the chances of session hijacking.\\
        \hline
        Hub Network & Switch network & Mitigates the risk of ARP spoofing and othe session hijacking attacks. 
    \end{tabular}
\end{itemize}
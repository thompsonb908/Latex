\section{Definitions}

\subsection{Chapter 1: Introduction To Ethical Hacking}
\subsubsection{Information Security Overview}
\begin{itemize}
    \item Intelligence based warfare: A sensor-based technology that directly corrupts technological systems. "Warfare that consists of the design, protection, and denial of systems that seek sufficient knowledge to dominate the battle space.
    \item
\end{itemize}

\subsubsection{Cyber Kill Chain Concepts}
\begin{itemize}
    \item Reconnaissance: An Adversary performs reconnaissance to collect as much information about the target as possible to probe for weak points before attacking.
    \item Installation: Adversary downloads and intalls more malicious software on the target system to maintain access to the target network for an extended period.
    \item Command and control: The adversary creates a command and control channel, which establishes two-way communication between the victim's system and adverary-controlled servers to communicate and pass data back and forth.
    \item Weaponization: Adversary selects or creates a tailored deliverable malicious payload (remote access malware weapon) using an exploit and a backdoor to send it to the victim.
    \item
\end{itemize}
\subsubsection{Hacking and Ethical Hacking Concepts}

\subsubsection{Information security controls, laws and standards}
\begin{itemize}
    \item SOX Titles:
    \begin{itemize}
        \item Title 3: Corpoate Responsability, eight sections and mandates that senior executives take individual responsability for the accuracy and completeness of corporate financial reports.
        \item Title 5: Analyst Conflicts of Intrest: One section that discusses the measures designed to help restore investor confidence in the reporting of securities analyst. Defines the code of conduct for securities analysts and requires that they disclose any knowable conflicts of interest.
        \item Title 6: Commission Resources and Authority: four sections defining practices to restore investor confidence in securities analysts. Defines the SEC's authority to censure or bar securities professionals from practice and defines the conditions to bar a person from practicing as a broker, advisor, or dealer.
        \item Title 7: Studies and Reports: five sections, requires the Comptroller General and the Securities and Exchange Commission (SEC) to perform various studies and to report their findings.
    \end{itemize}
\end{itemize}
\subsection{Chapter 2: Footprinting and Reconnaissance}
\subsubsection{Footprinting Concepts}
\begin{itemize}
    \item Sherlock: To search a vast number of social networking sites for a target username. This tool helps the attacker to locate the target user on various social networking sites along with the complete URL.
    \item BeRoot: BeRoot is a post-exploitation tool to check for common misconfigurations which can allow an attacker to escalate their privileges.
    \item OpUtils: SNMP enumeration protocol that helps to monitor, diagnose and trouble shoot the IT resources.
    \item Sublist3r: Sublist3r is a Python script designed to enumerate the subdomains of websites using OSINT. It enables you to enumerate subdomains across multiple sources at once.
    \item Passive footprinting: no direct interaction, archived and stored information from publically accessible sources.
    \begin{itemize}
        \item Finding information through search engines
        \item Finding the Top-level Domains (TLDs) and sub-domains of a target network through web services.
        \item Collecting information on the target through web services.
        \item Performing people search using social networking sites and people search engines.
        \item Gathering financial information about the target through financial services.
        \item Gathering infrastructure details of the target organization through job sites.
        \item Monitoring target using alert services.
    \end{itemize}
    \item Active footprinting, direct interaction with the target network:
    \begin{itemize}
        \item Querying published name servers of the target.
        \item Extracting metadata of published documents and files.
        \item Gathering website information using web spiderin and mirroring tools.
        \item Gathering information through email tracking.
        \item Performing Whois lookup
        \item Extracting DNS Information
        \item Performing traceroute analysis
        \item Performing social engineering.
    \end{itemize}
\end{itemize}

\subsubsection{Footprinting Methodology}

\subsubsection{Footprinting Tools and Countermeasures}

\subsection{Chapter 3: Scanning Networks}

\subsubsection{Network Scanning Concepts and Tools}

\subsubsection{Host, Port and Service Discovery}

\subsubsection{OS Discovery and Scanning Beyond IDS/Firewall}

\subsection{Chapter 4: Enumeration}

\subsubsection{Enumeration Concepts}

\subsubsection{NetBIOS and SNMP Enumeration}
\begin{itemize}
    \item Border Gateway Protocol (BGP) is a standardized exterior gateway protocol designed to exchange routing and reachability information among autonomous systems on the internet. Used by ISPs to maintain large routing tables. Utilizes port 179
\end{itemize}

\subsubsection{LDAP, NTP, NFS, and SMTP Enumeration}
\begin{itemize}
    \item LDAP - Lightweight Directory Access Protocol
\end{itemize}

\subsection{Chapter 5: Vulnerability Assessment}
\subsubsection{Vulnerability Assessment Concepts}
\begin{itemize}
    \item Vulnerability management lifecycle:
    \begin{itemize}
        \item Risk assessment: All serious uncertanties that are associated with the system are assessed and prioritized, and remediation is planned to permanently eliminate system flaws.
        \item Remediation: The process of applying fixes on vulnerable systems in order to reduce the impact and severity of vulnerabilities.
        \item Verification: Provides clear visibility into the firm and allows the security team to check whether all the previous phases have been perfectly employed or not.
        \item Monitoring: Organizations need to perform regular monitoring to maintain system security. Continuous monitoring identifies potential threats and any new vulnerabilities.
    \end{itemize}
    \item The Common Vulnerability Scoring System (CVSS) provides an open framework for communicating the characteristics and impacts of IT vulnerabilities.
    \begin{itemize}
        \item Base metric group
        \begin{itemize}
            \item Exploitability Metrics
            \begin{itemize}
                \item Attack Vector
                \item Attack Complexity
                \item Privileges Required
                \item User Interaction
                \item Scope
            \end{itemize}
            \item Impact Metrics
            \begin{itemize}
                \item Compatibility Impact
                \item Integrity Impact
                \item Availability impact
                \item Scope
            \end{itemize}
        \end{itemize}
    \end{itemize}
    \item Temporal Metric group
    \begin{itemize}
        \item Exploit Code maturity
        \item Remediation level
        \item Report confidence
    \end{itemize}
    \item Environmental Metric group
    \begin{itemize}
        \item Confidentiality Requirement
        \item Integrity Requirement
        \item Availability Requirement
        \item modified Base Metrics
    \end{itemize}
\end{itemize}

\subsubsection{Vulnerability Classification and Assessment Types}
\begin{itemize}
    \item Internal Assessment: Involves scrutinizing the internal network to find exploits and vulnerabilities.
    \item Network-based Assessment: Discover network resources and map the ports and services running to various areas on the network.
    \item Non-credentialed Assessment: Hacker does not possess any credentials.
    \item Credentialed Assessment: The ethical hacker possesses the credentials of all machines present in the assessed network.
    \item Distributed Assessment: employed by organizations with assets like servers and clients at different locations, involves simultaneously assessing the distributed organization assets, such as client and server applications using appropriate synchronization techniques.
\end{itemize}

\subsubsection{Vulnerability Assessment Solutions, Tools and Reports}
\begin{itemize}
    \item Product-Based Solutions: Solutions are installed either on a private or non-routable space or on the internet-addressable portion of an organization's network.
    \item Tree-Based Assessment: the auditor (parent) selects different strategies for each machine or component (child nodes) of the information system. This approach relies on the administrator to provide a starting piece of intelligence and then to start scanning continuously without incorporating any information found at the time of scanning.
    \item Service-Based Solutions: Offered by third parties, such as auditing or security consulting firms. Some solutions are hosted inside the network, while others are hosted outside the network.
    \item Inference-Based Assessment: Scanning starts by building an inventory of the protocols found on the machine.
    \item Depth Assessment Tools: Used to discover and identify previously unknown vulnerabilities in a system. Generally tools such as fuzzers, which provide arbitrary input to a system's interface, are used to identify vulnerabilities to an unstable depth.
    \item Host-Based Vulnerability Assessment Tools: appropriate for servers running various applications, such as the Web, critical files, databases, directories, and remote accesses. These host based scanners can detect high levels of vulnerabilities and provide required information about the fixes (patches)
    \item Scope assessment tools: Scope assessment tools provide an assessment of the security by testing vulnerabilities in the applications and operating system. These tools provide standard controls and a reporting interface that allows the user to select a suitable scan.
    \item Application-Layer Vulnerability Assessment Tools: Designed to sever the needs of all kinds of operating system types and applications. Various resources pose a variety of security threats and are identified by the tools designed for that purpose.
    \item Vulnerability scanning solutions steps:
    \begin{enumerate}
        \item Locating nodes: locate live hosts in the target network using various scanning techniques.
        \item Performing service and OS discovery: enumerate the open ports and services along with the operating system on the target systems.
        \item Testing for vulnerabilities: test for vulnerabilities on target nodes.
    \end{enumerate}
    \item Tools
    \begin{itemize}
        \item \verb|theHarvester|: used for open-source intelligence gathering and helps to determine a company's external threat landscape on the Internet. Attackers use this tool to perform enumeration on the LinkdIn social networking site to find employees of the target company along with their job titles.
        \item \verb|Qualys VM|: Cloud based service that gives immediate global visibility into where IT systems might be vulnerable to the latest Internet threats and how to protect them. Helps to continuously identify threats and monitor unexpected changes in a netowrk before they turn into breaches.
        \item \verb|Sherlock|: Searches a vast number of social networking sites for a target username.
        \item \verb|Octoparse|: Offers automatic data extraction, scrapes web data without coding and turns web pages into structured data. gathers text, links, image urls and html code. 
    \end{itemize}
    \item Report sections
    \begin{itemize}
        \item Scan information: Provides information such as the name of the scanning tool, its version, and the network ports to be scanned.
        \item Target Information: information about the target system's name and address.
        \item Results: A complete scanning report containing subtopics such as target, services, vulnerability, classification, and assessment.
        \item Target: Includes each host's detailed information and contains the following information:
        \begin{itemize}
            \item \verb|<Node>| name and address of the host.
            \item \verb|<OS>| Operating system
            \item \verb|<Date>| Date of the test.
        \end{itemize}
        \item Services: Defines the network services by their names and ports.
        \item Classification: Allows the system administrator to obtain additional information about the scan, such as its origin.
        \item Assessment: provides information regarding the scanner's assessment of discovered vulnerabilities.
    \end{itemize}
\end{itemize}

\subsection{System Hacking}
\subsubsection{System Hacking Concepts}
\subsection{Gaining Access (Cracking Passwords and Vulnerability Exploitation)}
\begin{itemize}
    \item Kerberos authentication: Employs a key distribution center (KDC) that consists of an authentication server (AS) and a ticket-granting server (TGS), and uses "tickets" to prove a user's identity.
    \item Markov-Chain Attack: Attackers gather a password database and split each password entry into two and three character syllables (2-grams and 3-grams); using these character elements, a new alphabet is developed, which is then matched with the existing password database.
    \item PRINCE Attack: A \textbf{PR}obability \textbf{IN}finite \textbf{C}hained \textbf{E}lements (PRINCE) attack is an advanced version of a combinator attack in which, instead of taking inputs from two different dictionaries, attackers use a single input dictionary to build chains of combined words.
    \item Combinator Attack: Attacker combines the entries of the first dictionary with those of the second dictionary. The resultant list of entries can be used to produce full names and compound words.
    \item Fingerprint Attack: The passphrase is broken down into fingerprints consisting of single- and multi- character combinations that a target user might choose as his/her password.
    \item Spiking: Allows attackers to send crafted TCP or UDP packets to the vulnerable server in order to make it crash.
    \item Generate shellcode: Attackers use the msfvenom command to generate the shellcode and inject it into the EIP register to gain shell access to the target vulnerable server.
    \item EIP Register: Extended Instruction Pointer (EIP) register contains the address of the next instruction to be executed.
    \item Fuzzing: Allows the attacker to send large amounts of data to the target server so that it experiences buffer overflow and overwrites the EIP register.
    \item Overwrite the EIP register allows attackers to identify whether the EIP register can be controlled and can be overwritten with malicious shellcode.
    \item Tools
    \begin{itemize}
        \item Factiva: Global news database and licensed content provider. It is a business information and research tool that gets information from licensed and free sources and provides capabilities such as searching, alerting, dissemination, and business information management.
        \item Shodan: Computer search engine that searches the Internet for connected devices (routers, servers, and IoT).
        \item SecurityFocus: database of the recently reported security vulnerabilities.
        \item Maltego: program that can be used to determine the relationship and real-world links between people, groups, organizations, websites, Internet infrastructure and documents.
        \item Infoga: Used for gathering email account information (IP,hostname, country) from different public sources and it checks if the email was leaked using the \verb|haveibeenpwned.com| API.
        \item Splint: Can be used to detect common security vulnerabilities including buffer overflows.
    \end{itemize}
    \item NTLMv2 us a default authentication sceme that performs authentication using a challenge/response strategy. Can be cracked with dictionary or brute force, not rainbow table because NTMLv2 adds a salt value that is exchanged in the messaging, thus it cannot be used in a pass-the-hast attack either.
    \item 
\end{itemize}
\subsubsection{Escalating Privileges}
\begin{itemize}
    \item Meltdown vulnerability - This is found in all the Intel processors and ARM processors deployed by Apple. This vulnerability leads to tricking a process to access out-of-bounds memory by exploiting CPU optimization mechanisms such as speculative execution.
    \item Dylib hijacking - Allows an attacker to inject a malicious sylib in one of the primary directories and simply load the malicious dylib at runtime.
    \item Spectre Vulnerability - Found in many modern processors such as AMD, ARM, Intel, Samsung and Qualcomm. Leads to tricking a processor to exploit speculative execution to read restricted data. Modern processors implement speculative execution to predict the future and to complete the execution faster.
    \item DLL hijacking - Attacker places a malicious DLL in the application directory; the application will execute the malicious DLL in place of the real DLL.
    \item Application Shimming - Malicious technique on Microsoft Windows in which application shim's are abused to establish persistence, inject DLLs, elevate privileges, and much more. The Microsoft Windows Application Compatibility Framework can be used to create Shim Database (.sdb) files that are typically used to fix software compatibility issued, however they can instead be abused for nefarious purposes.
\end{itemize}
\subsubsection{Maintaining Access (Executing Applications and Hiding Files)}
\begin{itemize}
    \item Rootkits
    \begin{itemize}
        \item Boot Loader Level Rootkit: Replaces the original bootloader with the one controlled by a remote attacker.
        \item Hardware/Firmware Rootkit: Hides in hardware devices or platform firmaware that are not inspected for code integrity.
        \item Hypervisor level rootkit: Acts as a hypervisor and modifies the boot sequence of the computer system to load the host operating system as a virtual machine.
        \item Library Level Rootkit: Replaced the original system calls with fake ones to hide information about the attacker.
        \item Application level rootkit: Operate inside the victims computer by replaceing the standard application files (binaries) with rootkits or by modifying behavior of resent applications with patches, injected malicious code, and so on.
        \item Kernel level rootkit: the kernel is the core of the operating system. Kernel level rootkits run in Ring-0 with the highest operating system privileges. These cover backdoors on the computer and are created by writing additional code or by substituting portions of kernel code with modified code via device drivers in Windows or loadable kernel modules in Linux. Of the kit's code contains mistakes or bugs, kernel-level rootkits affect the stability of the system. These have the same privileges of the operating system; hence they are difficult to detect and intercept or subvery the operatings of operating systems.
    \end{itemize}
    \item Hiding data
    \begin{itemize}
        \item Spread Spectrum Techniques: Communcation signals occupy more bandwidth than required to sent the infomration. The sender increases the band spread by means of code (independent of data), and the reciever uses a synchonized reception with the code to recover the information from the spread spectrum data.
        \item Transform Domain Techniques: Hides information in significant parts of the cover image, such as cropping, compression, and some other image processing areas.
        \item Substitution Techniques: Attacker tried to encode secret information by substituting the insignificant bits with the secret message.
        \item Distortion Techniques: The user implements a sequence of modifications to the cover to obtain a stego-object. The sequence of modifications represents the transformation of a specific message.
    \end{itemize}
    \item Stego-Attacks
    \begin{itemize}
        \item Stego-only attack: the steganalyst or attack does not have access to any information except the stego-medium or stego-object. In this attack, the steganalyst must try every possible steganography algorithm and related attack to revoce the hidden information.
        \item Chosen-message attack: The steganalyst uses a known message to generate a stego-object by using various steganography tools to find the the steganography algorithm used to hide information.
        \item Chosen-stego attack: Takes place when the steganalyst knows both the stego-object and steganography tool or algorithm to hide the message.
        \item Chi-square attack: The chi-square method is based on probability analysis to test whether a given stego-object and the original data are the same or not. If the differece between both is nearly zero, then not data are embedded; otherwise, the stego-object includes embedded data inside.
    \end{itemize}
\end{itemize}

\subsubsection{Clearing logs}
\begin{itemize}
    \item Commands
    \begin{itemize}
        \item \verb|history -c|: useful in clearing the stored history.
        \item \verb|export HISTSIZE=0|: This command disables the BASH shell from saving the history by setting the size of the history file to 0.
        \item \verb|history-w|: This command only deletes the history of the current shell, wheras the command history of other shells remain unaffected.
        \item \verb|shred ~\.bash_history|: This command shreds the history file, making its contents unreadable.
    \end{itemize}
    \item TCP Parameters: Can be used by the attacker to distribute the payload and to create covert channels. Some of the TCP fields where data can be hidden are:
    \begin{itemize}
        \item IP Identification field: one character is encapsulated per packet.
        \item TCP acknowledgement number: Uses a bounce server that revieves packets from the victim and sends it to an attacker. Here one hidden character is relayed by the bounce server per packet.
        \item TCP initial sequence number: does not require an established connection between two systems. Here, one hidden character is encapsulated per SYN request and Reset packets.
    \end{itemize}
    \item Clear Online Tracks: Attacker clear online tracks maintained using web history, logs. cookies, cache, downloads, visited time, and other on the target computer, so that victims cannot notice what online activities attackers have performed.
    \item Programs
    \begin{itemize}
        \item \verb|Auditpol.exe|: command line utility tool to change Audit Security settings at the category and sub-category levels. Attackers can use AuditPol to enable or disable security auditing on local or remote systems and to adjust the audit criteria for different categories of security events.
        \item \verb|Clear_Event_Viewer_Logs.bat/clearlogs.exe| utility for wiping the logs of a target system.
        \item \verb|SECEVENT.EVT|: Deletes security events
        \item \verb|SYSEVENT.EVT|
        \item \verb|APPEVENT.EVT|
    \end{itemize}
\end{itemize}

\subsection{Malware Threats}
\subsubsection{Malware Concepts}
\begin{itemize}
    \item Social Engineering Click-jacking: Inject malware into websites that appear legitimate to trick users into clicking them. When clicked, the malware embedded in the link executes without the knowledeg of the user.
    \item Malvertizing: Embedding malware-laden advertisements in legitimate onlone advertising channels to spread malware on systems of unsuspecting users.
    \item Black hat search Engine Optimization (SEO): also known as unethical SEO uses agressive SEO tactics such as keyword stuffing, inserting doorway pages, page swapping, and adding unrelated keywords to get higher search engine rankings for malware pages.
    \item Compromised Legitimate Websites
    \item Malware Components
    \begin{itemize}
        \item Downloader: Type of trojan that downloads other malware or malicious code files from the internet on to the PC or device. Attackers usually install downloaders when they first gain access to a system.
        \item Crypters: software that encrypts the original binary code of the .exe file. Crypters hide viruses, spyware, keyloggers, Remote Access Trojans (RATs), and others to make them undetectable to anti-viruses.
        \item Obfuscator: Obfuscation means to make code harder to understand or read, generally for privacy or security concerns. Converts a straightforward program into one that works the same way but is much harder to understand. It is a program to conceal the malicious code of malware via various techniques, thus making it hard for security mechanisms to detect or remove it.
        \item Payload: Part of the malware that performs desired activity when activated. 
    \end{itemize}
\end{itemize}
\subsubsection{APT Concepts}
\begin{itemize}
    \item 
\end{itemize}
\subsubsection{Trojan Concepts}
\begin{itemize}
    \item Ports for trojans:
    \begin{itemize}
        \item Port 80: Necurs, NetWire, Ismdoor, Poison Ivy, Executer, Codered, APT 18, APT 19, APT 32, BBSRAT, Calisto, Carbanak, Carbon, Comnie, Empire, FIN7, InvisiMole, Lazarus Group, MirageFox, Mis-Type, Misdat, Mivast, MoonWind, Night Dragon, POWERSTATS, RedLeaves, S-Type, Threat Group-3390, UBoatRAT.
        \item Port 20/22/80/442: Emotet
        \item Port 8080: Zeus, APT 37, Comnie, EvilGrab, FELIXROOT, FIN7, HTTPBrowser, Lazarus Group, Magic Hound, OceanSalt, S-Type, Shamoon, TYPEFRAME, Volgmer.
        \item Port 11000: Senna Spy
    \end{itemize}
    \item Banking trojan - steals credentials before they are encrypted by the system and sends them to the attacker.
    \begin{itemize}
        \item TAN Grapper: Transaction Authentication Number (TAN) is a single-use password for authenticating online banking transactions. Banking trojans intercept valid TANs entered by users and replace them with random numbers. Subsiquently, the attacker misuses the intercepted TAN with the target's login details.
        \item HTML Injection: Trojan creates fake form fields on e-banking pages, therby enabling the attacker to collect the target's account details, credit card number, date of birth, etc. The attacker can use this information to impersonate the target and compromise his/her account.
        \item Form Grabber: Type of malware that captures a target's sensitive data such as IDs and passwords, from a web browser form or page. It is an advanced method for collecting the target's Internet banking infomration. It analyses POST requests and responses to the victims browser. it compromises the scramble pad authentication and intercepts the scramble pad input as the user enters his/her Customer Number and Personal Access Code.
        \item Covert Credential Grabber: This malware remains dormant until the user performs an online financial transaction. It works covertly to replicate itself on the computer and edits the registry entries each time the computer is started. The trojan also searches the cookie files that had been stored on the computer while browsing financial websites. Once the user attempts to make an online transaction, the Trojan covertly steals the login credentials and transmits them to the hacker.
    \end{itemize}
    \item Covert Channel: methods attackers use to hide data in an undetectable protocol. Rely on tunneling, which enables one protocol to transmit ofver the other. Any process or a bit of data can be a covert channel. Attackers can use covert channels to install backdoors on the target machine.
    \item Asymmetric routing: Routing technique where packets flowing through TCP connections travel through different routes to different directions.
    \item Tools:
    \begin{itemize}
        \item Trojan.Gen: generic detection for many individual but varied Trojans for which specific definitions have not been created.
        \item Senna Spy Trojan Generator: Trojan that comer hidden in malicious programs. Once you install the source program, the trojan attempts to gain 'root' access without knowledge.
        \item Win32.Trojan.BAT: System destructive trojan program. It will crash the system by deleting files.
        \item DarkHorse Trojan Maker: Used to create user-specific trojans by selecting from various options.
    \end{itemize}
    \item Trojans
    \begin{itemize}
        \item Mirai: a self-propagating botnet that infects poorly protected internet devices (IoT). Uses Telnet port 23 or 2323 to find devices that are using their factory default username and password. Mirai is used to coordinate and mount a DDoS attack against a chosen victim.
        \item Netwire: type of RAT
        \item Theef: type of RAT
        \item Kedi RAT: type of RAT
    \end{itemize}
\end{itemize}

\subsubsection{Virus and Worm Concepts}
\begin{itemize}
    \item Virus lifecycle Stages
    \begin{itemize}
        \item Replication: Virus replicates for a period within the target system and then spreads itself.
        \item Launch: Virus is activated when the user performs specific actions such as running an infected program.
        \item Detection: Virus is identified as a threat infecting the target system.
        \item Execution of the damage routine: User installs antivirus updates and eliminate the virus threats.
    \end{itemize}
    \item Types of viruses
    \begin{itemize}
        \item Sparse infector virus: infect less often and try to minimize their probability of discovery. Only infect on a certain condition or those files whose lengths fall within a narrow range.
        \item Metamorphic Viruses: Programmed such that they rewrite themselves completely each time they infect a new exe.
        \item Cavity Viruses: Some programs have empty spaces in them. Cavity viruses, or space fillers, overwrite a part of the host file with a constant (usually nulls), without increasing the length of the file while preserving its functionality. Maintaining a constant file size when infecting allows the vius to avoid detection.
        \item Polymorphic Viruses: Infect a file with an encrypted copy of a polymorphic code already decoded by a decryption module. Polymorphic viruses modify their code for each replication to avoid detection.
        \item Tunneling Viruses: Tries to hide from antivirus by actively altering and corrupting the service call interrupts while running. The virus code replaces the requests to perform operations with respect to these service call interrupts. They state false information to hide their presence from antivirus programs.
        \item Macro Viruses: Infect Microsoft Word or similar applications by automatically performing a sequence of actions after triggering an application. Mose macro viruses are written using the macro language Visual Basic or Applications (VBA), and they infect templates or convert infected documents into template files while maintaining their appearance of common document files.
        \item File Viruses: Infect files executed or interpreted in the system, such as COM, EXE, SYS, OVL, OBJ, PRG, MNU, and BAT files. File viruses can be direct-action (non-resident) or memory-resident-viruses.
        \item System or Boot Sector Viruses: Most common targets for a virus are the system sectors, which include the mastor boot record (MBR) and the DOS boot record system sectors. An OS executes code in these areas while booting. Every disk has some sort of system sector. MBRs are the most virus prone zones because if the MBR is corrupted, all data will be lost. The DOS boot sector also executes during system booting. This is a crucial point of attack for viruses.
    \end{itemize}
\end{itemize}

\subsubsection{Fileless Malware Concepts}
\begin{itemize}
    \item 
\end{itemize}

\subsubsection{Malware Analysis}
\begin{itemize}
    \item DLLs
    \begin{itemize}
        \item \verb|Kernel32.dll|: Core functionality, such as access and manipulation of memory, files, and hardware.
        \item \verb|Advapi32.dll|: Provides access to advanced core Windows components such as the Service Manager and Registry.
        \item \verb|WSock32.dll| and \verb|Ws2_32.dll|: Networking DLLs that help connect to a network or perform network-related tasks.
        \item \verb|Ntdll.dll|: Interface to the Windows kernel.
    \end{itemize}
    \item Tools
    \begin{itemize}
        \item Resource Hacker: A resource editor for 32 and 64 bit Windows applications. Both a resource compiler (for .rc files), and a decompiler - enabling viewing and editing of resources in executables (.exe; .dll; .src; etc.) and compiled resource libraries (.res, .mui).
        \item Ghirda: Software reverse engineering (SRE) framework created and maintained by the National Security Agency Reserach Directorate. Framework includes a suite of full-featured, high-end software analysis tools that enable users to analyze compiled code on a variety of platforms including Windows macOS, and Linux. Capabilities include disassembly, assembly, decompilation, graphine, and scripting, along with hundreds of other features.
        \item Hakiri: Monitors Ruby apps for dependency and code security vulnerabilities.
        \item Synk: Platform developers choose to build cloud native applications securly.
        \item BinText: small text extractor utility that can extract text from any kind of file and includes the ability to find plain ASCII text, Unicode (double byte ANSI) text and Resource strings, providing useful information for each item in the optional 'advanced' view mode.
        \item UPX (Ultimate Pakcer for Executables): FOSS exe packer supporting a number of file formats from different operating systems.
        \item ASPack: Advanced exe packer created to compress Win32 exe files and to protect them against non-professional reverse engineering.
        \item PE Explorer: Allows you to open, view and edit a variety of different 32-bit Windows exe file types (PE files) ranging from common (EXE, DLL, ActiveX) to less familiar types (SCR \{Screensavers\}, CPL \{Control panel applets\}), SYS, MSSTYLES, BPL, DPL, and more.
    \end{itemize}
    \item Malware Encryption
    \begin{itemize}
        \item SamSam: uses RSA-2048 asymmetric encryption technique
        \item WannaCry: Uses a combination of the RSA and AES algorithms to encrypt files
        \item Dharma: Encrypts files using an AES 256 algorithm. the AES key is also encrypted with an RSA 1024.
        \item Cerber: uses RC4 and RSA algorithms for encryption.
    \end{itemize}
    \item EXE file sections
    \begin{itemize}
        \item \verb|.rdata|: Contains the import and export information as well as other read-only data used by the program.
        \item \verb|.data|: Contains the program's global data, which the system can access from anywhere.
        \item \verb|.rsrc|: Consists of the resources employed by the executable, such as icons, images, menus, and strings, as this section offers multi-lingual support.
        \item \verb|.text|: Contains instructions and program code that the cpu executes.
    \end{itemize}
    \item Monitoring
    \begin{itemize}
        \item Startup Programms monitoring is used to detect suspicious startup programs and processes.
        \item Registry Monitoring is used to examing the changes made to the system's registry by malware.
        \item Process monitoring is used to scan for malicious processes.
        \item Windows services monitoring traces malicious services initiated by the malware. Since malware employs rootkit techniques to manipulate \verb|HKEY_LOCAL_MACHINE\System\CurrentControlSet\Services| registry keys to hide its processes, windows service monitoring can be used to identify such manipulations.
    \end{itemize}
\end{itemize}

\subsubsection{Malware Countermeasures}
\begin{itemize}
    \item Tools:
    \begin{itemize}
        \item AlienVault USM Anywhere: A fileless malware detection tool that provides a unified platform for threat detection, incident response, and compliance management. It centralizes security monitoring of networks and devices in the cloud, on premis, and at remote locations, helping to detect threats anywhere.
        \item GFI LanGuard: patch management software scans the network and installs and manages security and non-security patches.
        \item Sonar Lite: Used to troubleshoot network connetivity, domain resolution issues or find out registration information for any domain.
        \item Monit: M/Monit can monitor and manage distributed computer systems, conduct automatic maintenance and repair, and execute meaningful casual actions in error situations.
        \item ClamWin: Free antivirus program for Windows.
        \item DriverView: Displays the list of all device drivers loaded on the system. Gives additional information about the driver as well.
    \end{itemize}
    \item Malware:
    \begin{itemize}
        \item ZeuS: Also known as Zbot, a powerful banking trojan that explicitly attempts to steal confidential infomration like system information, online credentials, banking details, etc. Zeus is spread through drive-by-downloads and phishing schemes.
    \end{itemize}
\end{itemize}

\subsection{Sniffing}
\subsubsection{Sniffing Concepts}
\begin{itemize}
    \item 
\end{itemize}
\subsubsection{Sniffing Techniques}
\begin{itemize}
    \item Tools
    \begin{itemize}
        \item Nikto: A web server assessment tool that examines a web server to discover potential problems and security vulnerabilities.
        \item dsniff: a collection of tools for netowrk auditing and penetration testing and can also be used to perform ARP poisoning.
        \item OpenVAS: a framework of several services and tools offering a comprehensive and powerful vulnerability scaning and vulnerability management solution
        \item Nexpose: Vulnerability scanner which aims to support the entire vulnerability management lifecycle, including discovery, detection, verification, risk classification, impact analysis, reporting and mitigation.
    \end{itemize}
\end{itemize}


\subsection{General / Unsorted}
\begin{itemize}
    \item CIA Triad:
    \begin{itemize}
        \item Confidentiality: unauthorized access to information.
        \item Integrity: Trustworthiness of data
        \item Availability: accessible when required
        \item (Other) Non-repudiation: Sender of a message cannot deny having sent the message, same for receiver.
        \item (Other) Authenticity: quality of being genuine
    \end{itemize}
    \item OSI model - Open System Interconnection model
    \item Local Area Network (LAN): Computer network that connects two or more computers within a limited area.
    \item Virtual Local Area Network (VLAN): Broadcast domain that is divided in a computer network at the data link layer (OSI layer 2).
    \item Wide Area Network (WAN): Covers larger area than a LAN, typically involves telecommunication circuits for a special purpose, ie: banking network. Nodes are more than 10 miles apart.
    \item Time to live (TTL): time period a message can live on the network before it is discarded. (8-bits). Number of seconds or number of hops?
    \item User Datagram Protocol (UDP): light weight communication protocol that gives no assurance of delivery.
    If the application receives out of order packets they are destroyed rather than worrying about reordering them.
    \item Transmission Control Protocol (TCP):
    \item Internet of Things (IoT): Devices with embedded software and network access.
   
    \item Malware: software created to harm or infiltrate a computer system without the owners consent.
    \begin{itemize}
        \item Virus: Create copies of themselves in other programs and activate from a trigger event.
        \item Worm
        \item Spyware
        \item Trojan
    \end{itemize}
    \item Information Security Policy: set of rules sanction by an organization to ensure that user of networks abide by the prescriptions regarding the security of data stored within the boundaries of the organization.
    \item Event: Something that happens that is detectable
    \item Incident: an event that violates policy.
    \item Certificate Authority: Organization that issues digital certificates.
    \item Vulnerability Scanner: Computer program designed to assess computer systems, network or applications for known weaknesses.
    \item Uniform Resource Locator (URL): reference to a web resource. Is a specific type of URI.
    \item Uniform Resource Identifier (URI): Unique sequence of characters that identifies a logical or physical resource used by web technologies. the \verb|http://| part of the url.
    \item DNS Zone transfer: Used to duplicate or make copies of DNS data across a number of DNS servers or to back up DNS files.
    \item Open-source intelligence: to describe identifying information about a target using freely available sources.
    \item Defence in breadth: planned, systematic set of multi-disciplinary activities that seek to identify, manage, and reduce risk of exploitable vulnerabilities at every stage of the system, network, or sub-component lifecycle.
    \item Defence in depth (DiD): Information security approach in which a series of security mechanisms and controls are layered throughout a computer network.
    \item Lawful Interception: Process of legally intercepting communications betwen two or more parties for surveillance on telecommunications, VoIP, data, and multiservice networks.
    \item Internet Zones
    \begin{itemize}
        \item Internet (uncontrolled zone): outside the boundary of your organization.
        \item Internet DMZ (controlled zone): Internet-facing controlled zone that contains components in which clients may directly communicate with. Usually buffered by two firewalls one from internet to DMZ and one from DMZ to the internal network.
        \item Production network (restricted zone): A restricted zone supports functions to which access must be strictly controlled; direct access from an uncontrolled network should not be permitted. In a large enterprise, several network zones might be designated as restricted. As with an internet DMZ, a restricted zone is typically bounded by one or more firewalls that filter incoming and outgoing traffic.
        \item Intranet (controlled zone): is not heavily restricted in use, but an appropriate span of control is in place to assure that network traffic does not compromise the operation of critical business functions.
        \item Management network (secured zone): In a secured zone, access is tightly controlled and available to only to a small number of authorized users. Access to one area of the zone does not necessarily apply to another area of the zone.
    \end{itemize}
\end{itemize}

\subsection{Attacks}
\begin{itemize}
    \item SQL Injection:
    \begin{itemize}
        \item In-band SQL Injection: Attacker uses the same communication channel to launch the attack and gather results. (error-based and union-based SQL injection).
    \end{itemize}
    \item Bluetooth
    \begin{itemize}
        \item Bluesnarfing: Theft of information from a target device using a bluetooth connection.
        \item Bluejacking: Transmission of data to a target device using a bluetooth connection.
    \end{itemize}
    \item Operating System Attacks
    \item Application-Level Attacks
    \item Shrink Wrap Code Attacks
    \item Misconfiguration Attacks
    \item DHCP starvation attack: Broadcasting DHCP requests with spoofed MAC addresses to expend the available address pool, denying access to new users.
    \item MAC flooding attack: Attacker floods the switch MAC table to push legitimate MAC addresses out of the switch. This causes significant amounts of frames to be broadcasted to all ports.
\end{itemize}

\subsection{Organizations}
\begin{itemize}
    \item Open Web Application Security Project (OWASP): International non-profit organization focused on web application security.
    \item Federal Risk and Authorization Management Program (FedRAMP): Cloud computing regulatory effort, government-wide, delivers systemized approach to security assessment, authorization, and continuous monitoring of cloud products and services.
\end{itemize}

\subsection{Cloud computing}
\begin{itemize}
    \item Platform as a service (PaaS): Third-party provider delivers hardware and software tools to users over the internet. PaaS frees developers from having to install in-house hardware and software to develop or run a new application.
    \item Infrastructure as a Service (IaaS):
    \item Hardware as a Service (HaaS):
    \item Software as a Service (SaaS):
    \item Models:
    \begin{itemize}
        \item Private
        \item Public
        \item Community: Infrastructure is shared by several organizations, usually with the same policy and compliance considerations.
        \item Hybrid
    \end{itemize}
\end{itemize}

\subsection{Cryptography}
\begin{itemize}
    \item Ciphers
    \begin{itemize}
        \item Symmetric Ciphers: Single key is used for encryption and decryption
        \begin{itemize}
            \item Data Encryption Standard (DES): Symmetric-key block cipher with key size of 56-bits
            \item Triple Data Encryption Algorithm (3DES, TDES, TDEA): Applies the DES algorithm 3 times to each data block. Key length of \(56 \times 3 = 168\) bits when 3 independent keys are used, or 112 when two keys are independent.
        \end{itemize}
        \item Asymmetric Ciphers (Public key cryptography): One key can encrypt and one key can decrypt.
        \begin{itemize}
            \item
        \end{itemize}
    \end{itemize}
\end{itemize}

\subsection{Registers}
\begin{itemize}
    \item EIP - Extended Instruction Pointer stores the address of the next instruction to be executed.
    \item ESP - Stack pointer, contains the address of the next element to be stored onto the stack.
    \item EBP - Extended Base pointer (StackBase), contains the address of the bottom (first element) of the stack frame.
    \item EDI - Destination Index, used with string instruction.
    \item ESI - Source Index, used with string instruction.
\end{itemize}
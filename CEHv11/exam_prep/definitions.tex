\section{Definitions}

\subsection{Chapter 1: Introduction To Ethical Hacking}
\subsubsection{Information Security Overview}
\begin{itemize}
    \item Intelligence based warfare: A sensor-based technology that directly corrupts technological systems. "Warfare that consists of the design, protection, and denial of systems that seek sufficient knowledge to dominate the battle space.
    \item
\end{itemize}

\subsubsection{Cyber Kill Chain Concepts}
\begin{itemize}
    \item Reconnaissance: An Adversary performs reconnaissance to collect as much information about the target as possible to probe for weak points before attacking.
    \item Installation: Adversary downloads and intalls more malicious software on the target system to maintain access to the target network for an extended period.
    \item Command and control: The adversary creates a command and control channel, which establishes two-way communication between the victim's system and adverary-controlled servers to communicate and pass data back and forth.
    \item Weaponization: Adversary selects or creates a tailored deliverable malicious payload (remote access malware weapon) using an exploit and a backdoor to send it to the victim.
    \item
\end{itemize}
\subsubsection{Hacking and Ethical Hacking Concepts}

\subsubsection{Information security controls, laws and standards}
\begin{itemize}
    \item SOX Titles:
    \begin{itemize}
        \item Title 3: Corpoate Responsability, eight sections and mandates that senior executives take individual responsability for the accuracy and completeness of corporate financial reports.
        \item Title 5: Analyst Conflicts of Intrest: One section that discusses the measures designed to help restore investor confidence in the reporting of securities analyst. Defines the code of conduct for securities analysts and requires that they disclose any knowable conflicts of interest.
        \item Title 6: Commission Resources and Authority: four sections defining practices to restore investor confidence in securities analysts. Defines the SEC's authority to censure or bar securities professionals from practice and defines the conditions to bar a person from practicing as a broker, advisor, or dealer.
        \item Title 7: Studies and Reports: five sections, requires the Comptroller General and the Securities and Exchange Commission (SEC) to perform various studies and to report their findings.
    \end{itemize}
\end{itemize}
\subsection{Chapter 2: Footprinting and Reconnaissance}
\subsubsection{Footprinting Concepts}
\begin{itemize}
    \item Sherlock: To search a vast number of social networking sites for a target username. This tool helps the attacker to locate the target user on various social networking sites along with the complete URL.
    \item BeRoot: BeRoot is a post-exploitation tool to check for common misconfigurations which can allow an attacker to escalate their privileges.
    \item OpUtils: SNMP enumeration protocol that helps to monitor, diagnose and trouble shoot the IT resources.
    \item Sublist3r: Sublist3r is a Python script designed to enumerate the subdomains of websites using OSINT. It enables you to enumerate subdomains across multiple sources at once.
    \item Passive footprinting: no direct interaction, archived and stored information from publically accessible sources.
    \begin{itemize}
        \item Finding information through search engines
        \item Finding the Top-level Domains (TLDs) and sub-domains of a target network through web services.
        \item Collecting information on the target through web services.
        \item Performing people search using social networking sites and people search engines.
        \item Gathering financial information about the target through financial services.
        \item Gathering infrastructure details of the target organization through job sites.
        \item Monitoring target using alert services.
    \end{itemize}
    \item Active footprinting, direct interaction with the target network:
    \begin{itemize}
        \item Querying published name servers of the target.
        \item Extracting metadata of published documents and files.
        \item Gathering website information using web spiderin and mirroring tools.
        \item Gathering information through email tracking.
        \item Performing Whois lookup
        \item Extracting DNS Information
        \item Performing traceroute analysis
        \item Performing social engineering.
    \end{itemize}
\end{itemize}

\subsubsection{Footprinting Methodology}

\subsubsection{Footprinting Tools and Countermeasures}

\subsection{Chapter 3: Scanning Networks}

\subsubsection{Network Scanning Concepts and Tools}

\subsubsection{Host, Port and Service Discovery}

\subsubsection{OS Discovery and Scanning Beyond IDS/Firewall}

\subsection{Chapter 4: Enumeration}

\subsubsection{Enumeration Concepts}

\subsubsection{NetBIOS and SNMP Enumeration}
\begin{itemize}
    \item Border Gateway Protocol (BGP) is a standardized exterior gateway protocol designed to exchange routing and reachability information among autonomous systems on the internet. Used by ISPs to maintain large routing tables. Utilizes port 179
\end{itemize}

\subsubsection{LDAP, NTP, NFS, and SMTP Enumeration}
\begin{itemize}
    \item LDAP - Lightweight Directory Access Protocol
\end{itemize}

\subsection{Chapter 5: Vulnerability Assessment}
\subsubsection{Vulnerability Assessment Concepts}
\begin{itemize}
    \item Vulnerability management lifecycle:
    \begin{itemize}
        \item Risk assessment: All serious uncertanties that are associated with the system are assessed and prioritized, and remediation is planned to permanently eliminate system flaws.
        \item Remediation: The process of applying fixes on vulnerable systems in order to reduce the impact and severity of vulnerabilities.
        \item Verification: Provides clear visibility into the firm and allows the security team to check whether all the previous phases have been perfectly employed or not.
        \item Monitoring: Organizations need to perform regular monitoring to maintain system security. Continuous monitoring identifies potential threats and any new vulnerabilities.
    \end{itemize}
    \item The Common Vulnerability Scoring System (CVSS) provides an open framework for communicating the characteristics and impacts of IT vulnerabilities.
    \begin{itemize}
        \item Base metric group
        \begin{itemize}
            \item Exploitability Metrics
            \begin{itemize}
                \item Attack Vector
                \item Attack Complexity
                \item Privileges Required
                \item User Interaction
                \item Scope
            \end{itemize}
            \item Impact Metrics
            \begin{itemize}
                \item Compatibility Impact
                \item Integrity Impact
                \item Availability impact
                \item Scope
            \end{itemize}
        \end{itemize}
    \end{itemize}
    \item Temporal Metric group
    \begin{itemize}
        \item Exploit Code maturity
        \item Remediation level
        \item Report confidence
    \end{itemize}
    \item Environmental Metric group
    \begin{itemize}
        \item Confidentiality Requirement
        \item Integrity Requirement
        \item Availability Requirement
        \item modified Base Metrics
    \end{itemize}
\end{itemize}

\subsubsection{Vulnerability Classification and Assessment Types}
\begin{itemize}
    \item Internal Assessment: Involves scrutinizing the internal network to find exploits and vulnerabilities.
    \item Network-based Assessment: Discover network resources and map the ports and services running to various areas on the network.
    \item Non-credentialed Assessment: Hacker does not possess any credentials.
    \item Credentialed Assessment: The ethical hacker possesses the credentials of all machines present in the assessed network.
    \item Distributed Assessment: employed by organizations with assets like servers and clients at different locations, involves simultaneously assessing the distributed organization assets, such as client and server applications using appropriate synchronization techniques.
\end{itemize}

\subsubsection{Vulnerability Assessment Solutions, Tools and Reports}
\begin{itemize}
    \item Product-Based Solutions: Solutions are installed either on a private or non-routable space or on the internet-addressable portion of an organization's network.
    \item Tree-Based Assessment: the auditor (parent) selects different strategies for each machine or component (child nodes) of the information system. This approach relies on the administrator to provide a starting piece of intelligence and then to start scanning continuously without incorporating any information found at the time of scanning.
    \item Service-Based Solutions: Offered by third parties, such as auditing or security consulting firms. Some solutions are hosted inside the network, while others are hosted outside the network.
    \item Inference-Based Assessment: Scanning starts by building an inventory of the protocols found on the machine.
    \item Depth Assessment Tools: Used to discover and identify previously unknown vulnerabilities in a system. Generally tools such as fuzzers, which provide arbitrary input to a system's interface, are used to identify vulnerabilities to an unstable depth.
    \item Host-Based Vulnerability Assessment Tools: appropriate for servers running various applications, such as the Web, critical files, databases, directories, and remote accesses. These host based scanners can detect high levels of vulnerabilities and provide required information about the fixes (patches)
    \item Scope assessment tools: Scope assessment tools provide an assessment of the security by testing vulnerabilities in the applications and operating system. These tools provide standard controls and a reporting interface that allows the user to select a suitable scan.
    \item Application-Layer Vulnerability Assessment Tools: Designed to sever the needs of all kinds of operating system types and applications. Various resources pose a variety of security threats and are identified by the tools designed for that purpose.
    \item Vulnerability scanning solutions steps:
    \begin{enumerate}
        \item Locating nodes: locate live hosts in the target network using various scanning techniques.
        \item Performing service and OS discovery: enumerate the open ports and services along with the operating system on the target systems.
        \item Testing for vulnerabilities: test for vulnerabilities on target nodes.
    \end{enumerate}
    \item Tools
    \begin{itemize}
        \item \verb|theHarvester|: used for open-source intelligence gathering and helps to determine a company's external threat landscape on the Internet. Attackers use this tool to perform enumeration on the LinkdIn social networking site to find employees of the target company along with their job titles.
        \item \verb|Qualys VM|: Cloud based service that gives immediate global visibility into where IT systems might be vulnerable to the latest Internet threats and how to protect them. Helps to continuously identify threats and monitor unexpected changes in a netowrk before they turn into breaches.
        \item \verb|Sherlock|: Searches a vast number of social networking sites for a target username.
        \item \verb|Octoparse|: Offers automatic data extraction, scrapes web data without coding and turns web pages into structured data. gathers text, links, image urls and html code. 
    \end{itemize}
    \item Report sections
    \begin{itemize}
        \item Scan information: Provides information such as the name of the scanning tool, its version, and the network ports to be scanned.
        \item Target Information: information about the target system's name and address.
        \item Results: A complete scanning report containing subtopics such as target, services, vulnerability, classification, and assessment.
        \item Target: Includes each host's detailed information and contains the following information:
        \begin{itemize}
            \item \verb|<Node>| name and address of the host.
            \item \verb|<OS>| Operating system
            \item \verb|<Date>| Date of the test.
        \end{itemize}
        \item Services: Defines the network services by their names and ports.
        \item Classification: Allows the system administrator to obtain additional information about the scan, such as its origin.
        \item Assessment: provides information regarding the scanner's assessment of discovered vulnerabilities.
    \end{itemize}
\end{itemize}

\subsection{System Hacking}
\subsubsection{System Hacking Concepts}
\subsection{Gaining Access (Cracking Passwords and Vulnerability Exploitation)}
\begin{itemize}
    \item Kerberos authentication: Employs a key distribution center (KDC) that consists of an authentication server (AS) and a ticket-granting server (TGS), and uses "tickets" to prove a user's identity.
    \item Markov-Chain Attack: Attackers gather a password database and split each password entry into two and three character syllables (2-grams and 3-grams); using these character elements, a new alphabet is developed, which is then matched with the existing password database.
    \item PRINCE Attack: A \textbf{PR}obability \textbf{IN}finite \textbf{C}hained \textbf{E}lements (PRINCE) attack is an advanced version of a combinator attack in which, instead of taking inputs from two different dictionaries, attackers use a single input dictionary to build chains of combined words.
    \item Combinator Attack: Attacker combines the entries of the first dictionary with those of the second dictionary. The resultant list of entries can be used to produce full names and compound words.
    \item Fingerprint Attack: The passphrase is broken down into fingerprints consisting of single- and multi- character combinations that a target user might choose as his/her password.
    \item Spiking: Allows attackers to send crafted TCP or UDP packets to the vulnerable server in order to make it crash.
    \item Generate shellcode: Attackers use the msfvenom command to generate the shellcode and inject it into the EIP register to gain shell access to the target vulnerable server.
    \item EIP Register: Extended Instruction Pointer (EIP) register contains the address of the next instruction to be executed.
    \item Fuzzing: Allows the attacker to send large amounts of data to the target server so that it experiences buffer overflow and overwrites the EIP register.
    \item Overwrite the EIP register allows attackers to identify whether the EIP register can be controlled and can be overwritten with malicious shellcode.
    \item Tools
    \begin{itemize}
        \item Factiva: Global news database and licensed content provider. It is a business information and research tool that gets information from licensed and free sources and provides capabilities such as searching, alerting, dissemination, and business information management.
        \item Shodan: Computer search engine that searches the Internet for connected devices (routers, servers, and IoT).
        \item SecurityFocus: database of the recently reported security vulnerabilities.
        \item Maltego: program that can be used to determine the relationship and real-world links between people, groups, organizations, websites, Internet infrastructure and documents.
        \item Infoga: Used for gathering email account information (IP,hostname, country) from different public sources and it checks if the email was leaked using the \verb|haveibeenpwned.com| API.
        \item Splint: Can be used to detect common security vulnerabilities including buffer overflows.
    \end{itemize}
    \item NTLMv2 us a default authentication sceme that performs authentication using a challenge/response strategy. Can be cracked with dictionary or brute force, not rainbow table because NTMLv2 adds a salt value that is exchanged in the messaging, thus it cannot be used in a pass-the-hast attack either.
    \item 
\end{itemize}
\subsubsection{Escalating Privileges}
\begin{itemize}
    \item Meltdown vulnerability - This is found in all the Intel processors and ARM processors deployed by Apple. This vulnerability leads to tricking a process to access out-of-bounds memory by exploiting CPU optimization mechanisms such as speculative execution.
    \item Dylib hijacking - Allows an attacker to inject a malicious sylib in one of the primary directories and simply load the malicious dylib at runtime.
    \item Spectre Vulnerability - Found in many modern processors such as AMD, ARM, Intel, Samsung and Qualcomm. Leads to tricking a processor to exploit speculative execution to read restricted data. Modern processors implement speculative execution to predict the future and to complete the execution faster.
    \item DLL hijacking - Attacker places a malicious DLL in the application directory; the application will execute the malicious DLL in place of the real DLL.
    \item Application Shimming - Malicious technique on Microsoft Windows in which application shim's are abused to establish persistence, inject DLLs, elevate privileges, and much more. The Microsoft Windows Application Compatibility Framework can be used to create Shim Database (.sdb) files that are typically used to fix software compatibility issued, however they can instead be abused for nefarious purposes.
\end{itemize}
\subsubsection{Maintaining Access (Executing Applications and Hiding Files)}
\begin{itemize}
    \item Rootkits
    \begin{itemize}
        \item Boot Loader Level Rootkit: Replaces the original bootloader with the one controlled by a remote attacker.
        \item Hardware/Firmware Rootkit: Hides in hardware devices or platform firmaware that are not inspected for code integrity.
        \item Hypervisor level rootkit: Acts as a hypervisor and modifies the boot sequence of the computer system to load the host operating system as a virtual machine.
        \item Library Level Rootkit: Replaced the original system calls with fake ones to hide information about the attacker.
        \item Application level rootkit: Operate inside the victims computer by replaceing the standard application files (binaries) with rootkits or by modifying behavior of resent applications with patches, injected malicious code, and so on.
        \item Kernel level rootkit: the kernel is the core of the operating system. Kernel level rootkits run in Ring-0 with the highest operating system privileges. These cover backdoors on the computer and are created by writing additional code or by substituting portions of kernel code with modified code via device drivers in Windows or loadable kernel modules in Linux. Of the kit's code contains mistakes or bugs, kernel-level rootkits affect the stability of the system. These have the same privileges of the operating system; hence they are difficult to detect and intercept or subvery the operatings of operating systems.
    \end{itemize}
    \item Hiding data
    \begin{itemize}
        \item Spread Spectrum Techniques: Communcation signals occupy more bandwidth than required to sent the infomration. The sender increases the band spread by means of code (independent of data), and the reciever uses a synchonized reception with the code to recover the information from the spread spectrum data.
        \item Transform Domain Techniques: Hides information in significant parts of the cover image, such as cropping, compression, and some other image processing areas.
        \item Substitution Techniques: Attacker tried to encode secret information by substituting the insignificant bits with the secret message.
        \item Distortion Techniques: The user implements a sequence of modifications to the cover to obtain a stego-object. The sequence of modifications represents the transformation of a specific message.
    \end{itemize}
    \item Stego-Attacks
    \begin{itemize}
        \item Stego-only attack: the steganalyst or attack does not have access to any information except the stego-medium or stego-object. In this attack, the steganalyst must try every possible steganography algorithm and related attack to revoce the hidden information.
        \item Chosen-message attack: The steganalyst uses a known message to generate a stego-object by using various steganography tools to find the the steganography algorithm used to hide information.
        \item Chosen-stego attack: Takes place when the steganalyst knows both the stego-object and steganography tool or algorithm to hide the message.
        \item Chi-square attack: The chi-square method is based on probability analysis to test whether a given stego-object and the original data are the same or not. If the differece between both is nearly zero, then not data are embedded; otherwise, the stego-object includes embedded data inside.
    \end{itemize}
\end{itemize}

\subsubsection{Clearing logs}
\begin{itemize}
    \item Commands
    \begin{itemize}
        \item \verb|history -c|: useful in clearing the stored history.
        \item \verb|export HISTSIZE=0|: This command disables the BASH shell from saving the history by setting the size of the history file to 0.
        \item \verb|history-w|: This command only deletes the history of the current shell, wheras the command history of other shells remain unaffected.
        \item \verb|shred ~\.bash_history|: This command shreds the history file, making its contents unreadable.
    \end{itemize}
    \item TCP Parameters: Can be used by the attacker to distribute the payload and to create covert channels. Some of the TCP fields where data can be hidden are:
    \begin{itemize}
        \item IP Identification field: one character is encapsulated per packet.
        \item TCP acknowledgement number: Uses a bounce server that revieves packets from the victim and sends it to an attacker. Here one hidden character is relayed by the bounce server per packet.
        \item TCP initial sequence number: does not require an established connection between two systems. Here, one hidden character is encapsulated per SYN request and Reset packets.
    \end{itemize}
    \item Clear Online Tracks: Attacker clear online tracks maintained using web history, logs. cookies, cache, downloads, visited time, and other on the target computer, so that victims cannot notice what online activities attackers have performed.
    \item Programs
    \begin{itemize}
        \item \verb|Auditpol.exe|: command line utility tool to change Audit Security settings at the category and sub-category levels. Attackers can use AuditPol to enable or disable security auditing on local or remote systems and to adjust the audit criteria for different categories of security events.
        \item \verb|Clear_Event_Viewer_Logs.bat/clearlogs.exe| utility for wiping the logs of a target system.
        \item \verb|SECEVENT.EVT|: Deletes security events
        \item \verb|SYSEVENT.EVT|
        \item \verb|APPEVENT.EVT|
    \end{itemize}
\end{itemize}

\subsection{Malware Threats}
\subsubsection{Malware Concepts}
\begin{itemize}
    \item Social Engineering Click-jacking: Inject malware into websites that appear legitimate to trick users into clicking them. When clicked, the malware embedded in the link executes without the knowledeg of the user.
    \item Malvertizing: Embedding malware-laden advertisements in legitimate onlone advertising channels to spread malware on systems of unsuspecting users.
    \item Black hat search Engine Optimization (SEO): also known as unethical SEO uses agressive SEO tactics such as keyword stuffing, inserting doorway pages, page swapping, and adding unrelated keywords to get higher search engine rankings for malware pages.
    \item Compromised Legitimate Websites
    \item Malware Components
    \begin{itemize}
        \item Downloader: Type of trojan that downloads other malware or malicious code files from the internet on to the PC or device. Attackers usually install downloaders when they first gain access to a system.
        \item Crypters: software that encrypts the original binary code of the .exe file. Crypters hide viruses, spyware, keyloggers, Remote Access Trojans (RATs), and others to make them undetectable to anti-viruses.
        \item Obfuscator: Obfuscation means to make code harder to understand or read, generally for privacy or security concerns. Converts a straightforward program into one that works the same way but is much harder to understand. It is a program to conceal the malicious code of malware via various techniques, thus making it hard for security mechanisms to detect or remove it.
        \item Payload: Part of the malware that performs desired activity when activated. 
    \end{itemize}
\end{itemize}
\subsubsection{APT Concepts}
\begin{itemize}
    \item 
\end{itemize}
\subsubsection{Trojan Concepts}
\begin{itemize}
    \item Ports for trojans:
    \begin{itemize}
        \item Port 80: Necurs, NetWire, Ismdoor, Poison Ivy, Executer, Codered, APT 18, APT 19, APT 32, BBSRAT, Calisto, Carbanak, Carbon, Comnie, Empire, FIN7, InvisiMole, Lazarus Group, MirageFox, Mis-Type, Misdat, Mivast, MoonWind, Night Dragon, POWERSTATS, RedLeaves, S-Type, Threat Group-3390, UBoatRAT.
        \item Port 20/22/80/442: Emotet
        \item Port 8080: Zeus, APT 37, Comnie, EvilGrab, FELIXROOT, FIN7, HTTPBrowser, Lazarus Group, Magic Hound, OceanSalt, S-Type, Shamoon, TYPEFRAME, Volgmer.
        \item Port 11000: Senna Spy
    \end{itemize}
    \item Banking trojan - steals credentials before they are encrypted by the system and sends them to the attacker.
    \begin{itemize}
        \item TAN Grapper: Transaction Authentication Number (TAN) is a single-use password for authenticating online banking transactions. Banking trojans intercept valid TANs entered by users and replace them with random numbers. Subsiquently, the attacker misuses the intercepted TAN with the target's login details.
        \item HTML Injection: Trojan creates fake form fields on e-banking pages, therby enabling the attacker to collect the target's account details, credit card number, date of birth, etc. The attacker can use this information to impersonate the target and compromise his/her account.
        \item Form Grabber: Type of malware that captures a target's sensitive data such as IDs and passwords, from a web browser form or page. It is an advanced method for collecting the target's Internet banking infomration. It analyses POST requests and responses to the victims browser. it compromises the scramble pad authentication and intercepts the scramble pad input as the user enters his/her Customer Number and Personal Access Code.
        \item Covert Credential Grabber: This malware remains dormant until the user performs an online financial transaction. It works covertly to replicate itself on the computer and edits the registry entries each time the computer is started. The trojan also searches the cookie files that had been stored on the computer while browsing financial websites. Once the user attempts to make an online transaction, the Trojan covertly steals the login credentials and transmits them to the hacker.
    \end{itemize}
    \item Covert Channel: methods attackers use to hide data in an undetectable protocol. Rely on tunneling, which enables one protocol to transmit ofver the other. Any process or a bit of data can be a covert channel. Attackers can use covert channels to install backdoors on the target machine.
    \item Asymmetric routing: Routing technique where packets flowing through TCP connections travel through different routes to different directions.
    \item Tools:
    \begin{itemize}
        \item Trojan.Gen: generic detection for many individual but varied Trojans for which specific definitions have not been created.
        \item Senna Spy Trojan Generator: Trojan that comer hidden in malicious programs. Once you install the source program, the trojan attempts to gain 'root' access without knowledge.
        \item Win32.Trojan.BAT: System destructive trojan program. It will crash the system by deleting files.
        \item DarkHorse Trojan Maker: Used to create user-specific trojans by selecting from various options.
    \end{itemize}
    \item Trojans
    \begin{itemize}
        \item Mirai: a self-propagating botnet that infects poorly protected internet devices (IoT). Uses Telnet port 23 or 2323 to find devices that are using their factory default username and password. Mirai is used to coordinate and mount a DDoS attack against a chosen victim.
        \item Netwire: type of RAT
        \item Theef: type of RAT
        \item Kedi RAT: type of RAT
    \end{itemize}
\end{itemize}

\subsubsection{Virus and Worm Concepts}
\begin{itemize}
    \item Virus lifecycle Stages
    \begin{itemize}
        \item Replication: Virus replicates for a period within the target system and then spreads itself.
        \item Launch: Virus is activated when the user performs specific actions such as running an infected program.
        \item Detection: Virus is identified as a threat infecting the target system.
        \item Execution of the damage routine: User installs antivirus updates and eliminate the virus threats.
    \end{itemize}
    \item Types of viruses
    \begin{itemize}
        \item Sparse infector virus: infect less often and try to minimize their probability of discovery. Only infect on a certain condition or those files whose lengths fall within a narrow range.
        \item Metamorphic Viruses: Programmed such that they rewrite themselves completely each time they infect a new exe.
        \item Cavity Viruses: Some programs have empty spaces in them. Cavity viruses, or space fillers, overwrite a part of the host file with a constant (usually nulls), without increasing the length of the file while preserving its functionality. Maintaining a constant file size when infecting allows the vius to avoid detection.
        \item Polymorphic Viruses: Infect a file with an encrypted copy of a polymorphic code already decoded by a decryption module. Polymorphic viruses modify their code for each replication to avoid detection.
        \item Tunneling Viruses: Tries to hide from antivirus by actively altering and corrupting the service call interrupts while running. The virus code replaces the requests to perform operations with respect to these service call interrupts. They state false information to hide their presence from antivirus programs.
        \item Macro Viruses: Infect Microsoft Word or similar applications by automatically performing a sequence of actions after triggering an application. Mose macro viruses are written using the macro language Visual Basic or Applications (VBA), and they infect templates or convert infected documents into template files while maintaining their appearance of common document files.
        \item File Viruses: Infect files executed or interpreted in the system, such as COM, EXE, SYS, OVL, OBJ, PRG, MNU, and BAT files. File viruses can be direct-action (non-resident) or memory-resident-viruses.
        \item System or Boot Sector Viruses: Most common targets for a virus are the system sectors, which include the mastor boot record (MBR) and the DOS boot record system sectors. An OS executes code in these areas while booting. Every disk has some sort of system sector. MBRs are the most virus prone zones because if the MBR is corrupted, all data will be lost. The DOS boot sector also executes during system booting. This is a crucial point of attack for viruses.
    \end{itemize}
\end{itemize}

\subsubsection{Fileless Malware Concepts}
\begin{itemize}
    \item 
\end{itemize}

\subsubsection{Malware Analysis}
\begin{itemize}
    \item DLLs
    \begin{itemize}
        \item \verb|Kernel32.dll|: Core functionality, such as access and manipulation of memory, files, and hardware.
        \item \verb|Advapi32.dll|: Provides access to advanced core Windows components such as the Service Manager and Registry.
        \item \verb|WSock32.dll| and \verb|Ws2_32.dll|: Networking DLLs that help connect to a network or perform network-related tasks.
        \item \verb|Ntdll.dll|: Interface to the Windows kernel.
    \end{itemize}
    \item Tools
    \begin{itemize}
        \item Resource Hacker: A resource editor for 32 and 64 bit Windows applications. Both a resource compiler (for .rc files), and a decompiler - enabling viewing and editing of resources in executables (.exe; .dll; .src; etc.) and compiled resource libraries (.res, .mui).
        \item Ghirda: Software reverse engineering (SRE) framework created and maintained by the National Security Agency Reserach Directorate. Framework includes a suite of full-featured, high-end software analysis tools that enable users to analyze compiled code on a variety of platforms including Windows macOS, and Linux. Capabilities include disassembly, assembly, decompilation, graphine, and scripting, along with hundreds of other features.
        \item Hakiri: Monitors Ruby apps for dependency and code security vulnerabilities.
        \item Synk: Platform developers choose to build cloud native applications securly.
        \item BinText: small text extractor utility that can extract text from any kind of file and includes the ability to find plain ASCII text, Unicode (double byte ANSI) text and Resource strings, providing useful information for each item in the optional 'advanced' view mode.
        \item UPX (Ultimate Pakcer for Executables): FOSS exe packer supporting a number of file formats from different operating systems.
        \item ASPack: Advanced exe packer created to compress Win32 exe files and to protect them against non-professional reverse engineering.
        \item PE Explorer: Allows you to open, view and edit a variety of different 32-bit Windows exe file types (PE files) ranging from common (EXE, DLL, ActiveX) to less familiar types (SCR \{Screensavers\}, CPL \{Control panel applets\}), SYS, MSSTYLES, BPL, DPL, and more.
    \end{itemize}
    \item Malware Encryption
    \begin{itemize}
        \item SamSam: uses RSA-2048 asymmetric encryption technique
        \item WannaCry: Uses a combination of the RSA and AES algorithms to encrypt files
        \item Dharma: Encrypts files using an AES 256 algorithm. the AES key is also encrypted with an RSA 1024.
        \item Cerber: uses RC4 and RSA algorithms for encryption.
    \end{itemize}
    \item EXE file sections
    \begin{itemize}
        \item \verb|.rdata|: Contains the import and export information as well as other read-only data used by the program.
        \item \verb|.data|: Contains the program's global data, which the system can access from anywhere.
        \item \verb|.rsrc|: Consists of the resources employed by the executable, such as icons, images, menus, and strings, as this section offers multi-lingual support.
        \item \verb|.text|: Contains instructions and program code that the cpu executes.
    \end{itemize}
    \item Monitoring
    \begin{itemize}
        \item Startup Programms monitoring is used to detect suspicious startup programs and processes.
        \item Registry Monitoring is used to examing the changes made to the system's registry by malware.
        \item Process monitoring is used to scan for malicious processes.
        \item Windows services monitoring traces malicious services initiated by the malware. Since malware employs rootkit techniques to manipulate \verb|HKEY_LOCAL_MACHINE\System\CurrentControlSet\Services| registry keys to hide its processes, windows service monitoring can be used to identify such manipulations.
    \end{itemize}
\end{itemize}

\subsubsection{Malware Countermeasures}
\begin{itemize}
    \item Tools:
    \begin{itemize}
        \item AlienVault USM Anywhere: A fileless malware detection tool that provides a unified platform for threat detection, incident response, and compliance management. It centralizes security monitoring of networks and devices in the cloud, on premis, and at remote locations, helping to detect threats anywhere.
        \item GFI LanGuard: patch management software scans the network and installs and manages security and non-security patches.
        \item Sonar Lite: Used to troubleshoot network connetivity, domain resolution issues or find out registration information for any domain.
        \item Monit: M/Monit can monitor and manage distributed computer systems, conduct automatic maintenance and repair, and execute meaningful casual actions in error situations.
        \item ClamWin: Free antivirus program for Windows.
        \item DriverView: Displays the list of all device drivers loaded on the system. Gives additional information about the driver as well.
    \end{itemize}
    \item Malware:
    \begin{itemize}
        \item ZeuS: Also known as Zbot, a powerful banking trojan that explicitly attempts to steal confidential infomration like system information, online credentials, banking details, etc. Zeus is spread through drive-by-downloads and phishing schemes.
    \end{itemize}
\end{itemize}

\subsection{Sniffing}
\subsubsection{Sniffing Concepts}
\begin{itemize}
    \item 
\end{itemize}
\subsubsection{Sniffing Techniques}
\begin{itemize}
    \item Tools
    \begin{itemize}
        \item Nikto: A web server and web application assessment tool that examines a web server to discover potential problems and security vulnerabilities.
        \item dsniff: a collection of tools for netowrk auditing and penetration testing and can also be used to perform ARP poisoning.
        \item OpenVAS: a framework of several services and tools offering a comprehensive and powerful vulnerability scaning and vulnerability management solution
        \item Nexpose: Vulnerability scanner which aims to support the entire vulnerability management lifecycle, including discovery, detection, verification, risk classification, impact analysis, reporting and mitigation.
        \item AnDOSid: Allows the attacker to simulate a DoS attack (an HTTP POST flood attack) and DDoS attack on a web server from mobile phones.
        \item Xplico: extracts application data from captured internet traffic. Is an open source Network Forensic Analysis Tool (NFAT).
        \item Akamai: provides DDoS protection for enterprises regularly targeted by DDoS attacks. Akamai Kona Site Defender delivers multi-layerd defense that effectivly protects websites and web applications against the increasing threat, sophistication, and scale of DDoS attacks.
        \item Vindicate: A LLMNR/NBNS/mDNS spoofing detection toolkit for network administrators. Security professionals use this tool to detect name service spoofing.
    \end{itemize}
    \item DNS Poisoning Techniques: sniff DNS traffic of a target network. An attacker can obtain the ID of the DNS request by sniffing and can send a malicious reply to the sender before the actual DNS server.
    \begin{itemize}
        \item Intranet DNS spoofing: An attacker can perform an intranet DNS spoofing attack on a switched LAN with the help of the ARP poisoning technique. To perform this attack, the attacker must be connected to the LAN and be able to sniff the traffic or packets. An attacker who succeeds in sniffing the ID of the DNS request from the intranet can send a malicious reply to the sender before the actual DNS server.
        \item Internet DNS spoofing: Attackers perform Internet DNS spoofing with the help of Trojans when the victim's system connects to the Internet. It is an MITM attack in which the attacker changes the primary DNS entries of the victim's computer.
        \item Proxy server DNS poisoning: In the proxy server DNS poisoning technique, the attacker sets up a proxy server on the attacker's system. The attacker also configures a fraudulent DNS and makes its IP address a primary DNS entry in the proxy server.
        \item DNS cache poisoning: Attackers target this DNS cache and make changes or add entries to the DNS cache. If the DNS resolver cannot validate that the DNS responses have come from an authoritative source, it will cache the incorrect entries locally and serve them to users who make the same request.
    \end{itemize}
\end{itemize}
\subsubsection{Sniffing Tools and Countermeasures}
\begin{itemize}
    \item Tools
    \begin{itemize}
        \item Spoof-Me-Now: program to change (spoof) your MAC address.
        \item OmniPeek: Network analyzer provides real time visibility and expert analysis of each part of the target network. will analyze, drill down, and fix performance bottlenecks across multiple network segments.
        \item DerpNSpoof: DNS poisoning tool that assists in spoofing the DNS query packet of a certian IP address or group of hosts on the network.
        \item ike-scan: discovers IKE hosts and can fingerprint them using the retransmission backoff pattern.
        \item Nmap: Used to scan networks, has a NSE script that allows you to check if a target on a local Ethernet has its network card in promiscuous mode by doing the ARP test.
        \item FaceNiff: Android app that can sniff and intercept web session profiles over the WiFi connected to the mobile. This app works on rooted Android devices. Whe WiFi connection should be over Open, WEP, WPA-PSK, or WPA2-PSK networks while sniffing the session.
        \item shARP: an anti ARP-spoofing application software that uses active and passive scanning methods to detect and remove any ARP-spoofer from the network.
    \end{itemize}
    \item Sniffing Attacks
    \begin{itemize}
        \item ARP Spoofing: A method of attacking an Ethernet LAN. When a legitimate user initiates a session with another user in the same layer 2 broadcast domain, the switch broadcasts and ARP request using the recipient's IP address, while the sender waits for the recipient to respond with a MAC address.
        \item ARP Poisoning: With the help of ARP poisoning, an attacker can use fake ARP messages to divert all communications between two machines so that all traffic redirects via the attacker's PC.
        \item ARP Method: Sends a non-broadcast ARP to all nodes in the network. The node that runs in promiscuous mode on the netwrok will cache the local ARP address. Then it will broadcast a ping message on the network with the local IP address but a different MAC address. In this case, onlt the node that has the MAC address (cached eariler) will be able to respond to your braodcast ping request.
        \item Ping method: To detect a sniffer on a network, identify the system on the network running in promiscuous mode. The ping method is useful in detecting a system that runs in promiscuous mode, which in turn helps detect sniffers installed on the network.s
    \end{itemize}
\end{itemize}

\subsection{Social Engineering}
\subsubsection{Social Engineering Concepts}
\begin{itemize}
    \item Intimidation: refers to an attempt to intimidate a victim into taking several actions by using bullying tactics.
    \item Scarcity: Implies the state of being scarce. In the context of social engineering, scarcity often implies creating a feeling of urgency in a decision making process.
    \item Consensus or Social Proof: Refers to the fact that people are usually willing to like things or do things that other people like or do.
    \item Authority: Implies the right to exercise power in an organization. Attackers take advantage of this by presenting themselves as a person of authority, such as a technician or an executive.
    \item Steps of a social engineering attack: Research on target company -> selecting target -> develop relationship -> exploit the relationship.
\end{itemize}
\subsubsection{Social Engineering Techniques}
\begin{itemize}
    \item Pop-Up Windows: windows that suddenly pop up while surfing the Internet and ask for user information to login or sign-in.
    \item Hoax (Letters): Emails or popups that issue warnings to the user about new viruses, Trojans, or worms that may harm the user's system.
    \item Instant Chat Messenger: Gathering personal information by chatting with a selected user online to get information such as birth dates and maiden names.
    \item Chain Letters: A chain letter is a message or email offering free gifts, such as money and software, on the condition that the user forward the email to a predetermined number of recipients.
    \item Pharming: Also known as "phishing without a lure" and performed by using DNS Cache Poisoning or Host File Modification.
    \item Whaling: Attacker tries to trick the victim into revealing critical corporate and personal information through email or website spoofing.
    \item Spimming: SPIM (Spamming over instant messaging) exploits Instant Messaging platforms and uses IM as a tool to spread spam. A person who generates spam over IM is called a Spimmer. Spimmers generally make use of bots to harvest Instan Messaging IDs and forward spam messgaes to them.
    \item Spear Phising: Sending a specialized message with social engineering content directed at a specific person, or small group.
    \item Skimming: refers to stealing credit/debit card number by using special storage devices called scimmers or wedges when processing the card.
    \item Wardriving: Attackers search for unsecured WiFi networks in moving vehicles containing laptops, smartphones, or PDAs. Once they find unsecured netowrks, they access any sensitive information stored on the devices of the users on the networks.
    \item Pretexting: fraudsters may pose as executives from financial institutions, telephone companiesm and so on who rely on "smooth talking" and win the trust of an individual to reveal sensitive information.
    \item Pharming: an advanced form of phishing in which attackers modify DNS protocol and redirects the connection between the IP address and its target server.
\end{itemize}
\subsubsection{Insider threats and Identity Theft}
\begin{itemize}
    \item Malicious Insider: Malicious insider threats come from disgruntled or terminated employees who steal data or destroy company networks intentionally by injectin malware into the corporate network.
    \item Negligent Insider: Insiders, who are uneducated on potential security threats or simply bypass general security procedures to meet workplace efficiency are more vulnerable to social engineering attcks. A large number of insider attacks result from employee's laxity towards security measures, policies and practices.
    \item Professional Insider: The most harmful insiders where they use their technical knowledge to identify weaknesses and vulnerabilities of the company's network and sell the confidential information to the competitors or black market bidders.
    \item Compromised Insider: An outsider compromises insiders having access to critical assets or computing devices of an organization. This type of threat is more difficult to detect since the outsider masquerades as a genuine insider.
    \item Tax Identity Theft: This type of identity theft occurs when perpetrator steals the victim's Social Security Number or SSN in order to file fraudulent tax returns and obtain fraudulent tax refunds. It creates difficulties for the victim in accessing the legitimate tax refunds and results in a loss of funds.
    \item Identity cloning and concealment: This is a type of identity theft which encompasses all forms of identity theft where the perpetrators attempt to impersonate someone else in order to simply hide their identity. These perpetrators could be illegal immigrants or those hiding from creditors or simply want to become “anonymous” due to some other reasons.
    \item Synthetic identity theft: This is one of the most sophisticated types of identity theft where the perpetrator obtains information from different victims to create a new identity. Firstly, he steals a Social Security Number or SSN and uses it with a combination of fake names, date of birth, address and other details required for creating new identity. The perpetrator uses this new identity to open new accounts, loans, credit cards, phones, other goods and services.
    \item Social identity theft: This is another most common type of identity theft where the perpetrator steals victim's Social Security Number or SSN in order to derive various benefits 
\end{itemize}
\subsubsection{Social Engineering Countermeasures}
\begin{itemize}
    \item Social Engineers Toolkit
\end{itemize}


\subsection{Denial-of-Service}
\subsubsection{DoS/DDoS Concepts}
\begin{itemize}
    \item DoS attakcs have various forms and target various services. The attacks may cause the following:
    \begin{itemize}
        \item Consumption of resources
        \item consumption of bandwidth, disk space, CPU time, or data structures.
        \item Actual physical destruction or alteration of network components
        \item Destruction of programming and files in a computer system.
    \end{itemize}
\end{itemize}
\subsubsection{DoS/DDoS Attack Techniques and Tools}
\begin{itemize}
    \item Back chaining Propogation: In this technique, the attacker places an attack toolkit on their own system, and a copy of the attack toolkit is transferred to a newly discovered vulnerable system. The attack tools installed on the attacking machine use some special methods to accept a connection from the compromised system and then transfer a file containing the attack tools to it.
    \item Autonomous Propagation: In autonomous propagation, the attacking host itself transfers the attack toolkit to a newly discovered vulnerable system, exactly at the time it breaks into that system.
    \item Central Source Propagation: In this technique, the attacker places an attack toolkit on a central source and a copy of the attack toolkit is transferred to a newly discovered vulnerable system. Once the attacker finds a vulnerable machine, they instruct the central source to transfer a copy of the attack toolkit to the newly compromised machine, on which attack tools are automatically installed under management by a scripting mechanism.
    \item Spyware Propagation: As its name implies, spyware is installed without user knowledge or consent, and this can be accomplished by “piggybacking” the spyware onto other applications.
    \item Tools:
    \begin{itemize}
        \item CORE Impact: Finds vulnerabilities in an organization's web server. This tool allows a user to evaluate the security posture of a web server by using the same techniques currently employed by cyber criminals.
        \item HULK: Denial of Service tool used to attack web servers by generating unique and obfuscated traffic volumes and its generated traffic also bypasses caching engines and hits the server's direct resource pool.
        \item Pupy: cross platform, muli function RAT and post-exploitation tool used for executing applications remotely.
        \item NetVisor: Desktop and child monitoring spyware that comes with an unparalleled task recording feature set that in secret records everything employees do on your network.
        \item Fritzing: assists attackers in designing electronic diagrams and circuits.
        \item Stormwall PRO: Filtering mitigation of all existing types of DDoS attacks on network, transport and session layers as well as application layer for HTTP(S)/Websocket traffic.
        \item Suphacap: a Z-Wave sniffer, is a hardware tool used to sniff traffic generated by smart devices connected in the network. It allows attackers to perform real-time monitoring and capturing of packets from all Z-Wave networks.
        \item KillerBee: Python based framework and tool set for exploring and exploiting the security of ZigBee and IEEE 802.15.4 netowrk.
    \end{itemize}
    \item Application-level flood attacks result in the loss of services of a particular network resource. Examples include email, network resources, temporary ceasing of applications and services, and so on. By using this attack, attackers exploit weaknesses in programming source code to prevent the application from processing legitimate requests. In this type of attack, an attacker tries to exploit the vulnerabilities in application layer protocol or in the application itself to prevent the access of the application to the legitimate user.
    \begin{itemize}
        \item Flood web applications to legitimate user traffic (GET/POST)
        \item Disrupt service to a specific system or person, for example, blocking a user's access by repeating invalid login attempts.
        \item Jam the application database connectino by crafting malicious SQL queries.
        \item Slowloris
        \item OS Vulnerabilities.
    \end{itemize}
    \item Protocol Attack: includes SYN floods, fragmented packets, ping of death, smurf DDos, teardrop, land, and other attacks.
    \item Volume-based attack: UDP floods, ICMP floods, and other spoofed packet floods.
\end{itemize}
\subsubsection{DoS/DDoS Protection Tools and Countermeasures}
\begin{itemize}
    \item Activity profiling: Performed based on the average packet flow rate for network flow, which consists of consecutive packets with similar packt header information.
    \item Wavelet-Based Signal Analysis: The wavelet analysis technique analyzes network traffic in terms of spectral components. it divides incoming signals into various frequencies and analyzes different frequency components separately.
    \item Sequential Change-Point Detection: Change-Point detection algorithms isolate changes in network traffic statistics and in the traffic flow rate caused by attacks. Uses cumulative sum algorithms.
    \item Absorbing the attack: Is a DoS/DDoS countermeasure strategy, in which additional capacity is used to absorb an attack, which requires preplanning and additional resources.
    \item Cisco IPS Source and reputation filtering: reputation services help in determining if an IP or service is a source of threat.
    \item Black Hole Filtering: refers to discarded packets at the routing level.
    \item RFC 3704 Filtering: a basic access control list (ACL) filter, which limits the impact of DDoS attacks by blocking traffic with spoofed addresses.
    \item DDoS Prevention Offering from ISP or DDoS service: Enable IP Source Gurad (in CISCO) or similar features in other routers to filter traffic based on the DHCP snooping binding database or IP source bindings, preventing a bot from succeeding with spoofed packets.
    \item Ingress Filtering protects against flooding attacks that originate from valid prefixes (IP addresses).
    \item Egress filtering scans the headers of IP packets going out of the network.
    \item TCP intercept: In this mode the router intercepts the SYN packets sent by the clients to the server and matches with an extended access list. If there is a match, then on behalf of the destination server, the intercept software establishes a connection with the client. Similarly, the intercept software also setablishes a connection with the destination server on behalf of the client. Once the two half connections are established, the intercept software combines them transparently. Prevents the attempts of fake connection from reaching the server. Acts as a mediator between the server and the client throughout the connection.
    \item MAC address filtering allows you to define a list of devices and only allows those devices on your network.
    \item Tools
    \begin{itemize}
        \item DDoS-Guard: online service to protect against DDoS
        \item A10 Thunder TPS: an Appliance that ensures reliable access to key network services by detecting and blocking external threats such as DDoS and other cyber-attacks before they escalate into costly service outages.
        \item Imperva Incapsula DDoS protection: Quickly miticgates attacks of any size without affecting legitimate traffic or increasing latency.
    \end{itemize}
\end{itemize}

\subsection{Session Hijacking}
\subsubsection{Session Hijacking Concepts}
\begin{itemize}
    \item 
\end{itemize}
\subsubsection{Application Level Session Hijacking}"
\begin{itemize}
    \item Man in the Middle Attack: A MITM attack is used to intrude into an exiting connection between systems and to intercept messages being transmitted. In this attack, attacks use different techniques and split a TCP connectin into two: client-to-attacker and attacker-to-server connections.
    \item Fragmentation Attack: These attacks destroy a victim's ability to reassemble fragmented packets by flooding it with TCP or UDP fragments, resulting in reduced performance. The attacker sends a large number of fragmented (1500+ byte) packets to a target web server with a relatively small packet rate.
    \item Man in the Browser Attack: Similar to a MITM attack. The difference between the two is that a MITB attack uses a Trojan horse to intercept and mainpulate calls between a browser and its security mechanisms or libraries. An attack positions a previously installed Trojan between the browser and its security mechanism, and the Trojan can modify web pages and transaction content or insert additional transactions. All of the Trojan's activities are invisible to both the user and the web application.
    \item Client-side Attack: Target vulnerabilities in client applications that interact with a malicious server or process malicious data. Depending on the nature of vulnerabilities, an attacker can exploit an application by sending an email with a malicious ling or otherwise tricking a user into visiting a malicious website. 
    \item XXS: enables attackers to inject malicious client-side scripts into web pages viewed by other users.
    \item Trojans: can change the proxy settings in the user's browser to send all sessions through an attacker's machine.
    \item Malicious JavaScript Codes: An attacker can embed in a web page a malicious script that does not generate any warning but captures session tokens in the background and sends them to the attacker.
    \item Session donation Attack: An attacker donates his/her own session identifier (SID) to the target user. The attacker first obtains a valid SID by logging into a service and later feeds the same SID to the target user. This SID links a target user back to the attacker's account page without any information to the victim.
    \item Proxy servers: An attacker lures the victim to click on a bogus link, which looks legitimate but redirects the user to the attackers server. The attacker forwards the request to the legitimate server on behalf of the victim and servers as a proxy for the entire transaction. The attacker then captures the session's information during the interaction of the legitimate server and user.
    \item CRIME Atatck: Compression Ratio Info-Leak Made Easy (CRIME) is a client side attack that exploits the vulnerabilities present in the data compression feature of protocols, such as SSL/TLS, SPDY, and HTTPS. Attackers hijack the session by decrypting secret session cookies. The authentication information obtained from the session cookies is used to establish a new session with the web application.
    \item Forbidden attack: Type of MITM used to hijack HTTPS sessions. It exploits the reuse of cryptographic nonce during the TLS handshake. After hijacking the GTTPS session, the attacker inject malicious code and forged content that prompts the victim to disclose sensitive information, such as bank account numbers, passwords, and social security numbers.
    \item Session replay attack: An attacker captures the authentication token of a user by listening to a conversation between the user and the server and reiterates the authentication request to the server with the captured authentication token to gain unauthorized access to the server.
    \item Application Level Hijacking: gaining control over HTTP's user session by obtaining the session IDs.
    \item Network Level hijacking: interception of packets during transmission in a TCP and UDP session between a server and client communication. attacks transport an internet level protocols
\end{itemize}
\subsubsection{Network Level Session Hijacking}
\begin{itemize}
    \item IP Spoofing: Source routed packets: useful in gaining unauthorized access to a computer with the help of a truseted host's IP address. This type of hijacking allows attackers to create their own acceptable packets to insert into the TCP session. First, the attacker spoofs the trusted host's IP address so that the server managing a session with the host accepts the packets from the attacker. The packets are source routed, so the sender specifies the path for packets from the source to the destination IP. Using this source-routing technique, attackers can fool the server into thinking that it is communicating with the user.
    \item Blind Hijacking: A hacker can inject malicious data or commands into the intercepted communications in a TCP session, even if the victim disables source routing. Here, an attacker correctly guesses the next ISN of a computer attempting to establish a connection; the attacker sends malicious data or a command, such as password setting to allow access from another location on the network, but the attacker can never see the response. To be able to see the response, a MITM attack works much better.
    \item TCP/IP hijacking: an attacker intercepts an established connection between two communicating parties using spoofed packets, and then pretends to be one of them. In this approach, the attacker uses spoofed packets to redirect the TCP traffic to his/her own machine. Once this is successful, the victim's connection hangs and the attacker is able to communicate with the host's machine on behalf of the victim.
    \item UDP hijacking
    \item RST Hijacking
\end{itemize}
\subsubsection{Session Hijacking Tools}
\begin{itemize}
    \item  Burp Suite: Burp Suite is a web security testing tool that can hijack session IDs in established sessions. The Sequencer tool in Burp Suite tests the randomness of session tokens. With this tool, an attacker can predict the next possible session ID token and use that to take over a valid session
    \item Vega: a free and open-source web security scanner and web security testing platform for testing the security of web applications. Vega helps you to find and validate SQL injection, XXS, inadvertently disclosed sensitive information and other vulnerabilities.
    \item PortQry: Reports the port status of TCP and UDP ports on a selected target. Attackers can use PortQry tool to perform TFTP enumeration. This utility reports the port status of target TCP and UDP ports on a local or remote computer.
    \item DroidSheep: Used for session hijacking on Android devices connected to a common wireless network. It obtains the session ID of active users on the WiFi network and uses it to access a website as an authorized user. A DroidSheep user can easily observe the activities of authorized users on websites. It can also hijack social accounts by obtaining the session ID.
    \item ShellPhish: A phishing tool used to phish user credentials from various social networking platforms such as Instagram, Facebook, Twitter, and LinkdIn. Also displays to victim system's public IP address, browser information, hostname, geolocation, and other information.
    \item Netcraft: Netcraft provides Internet security services, including anti-fraud and anti-phishing services, application testing, and PCI scanning. They also analyze the market share of web servers, operating systems, hosting providers and SSL certificate authorities, and other parameters of the Internet. The Netcraft anti-phishing community is a giant neighborhood watch scheme, empowering the most alert and most expert members to defend everyone within the community against phishing attacks. The Netcraft Toolbar provides updated information about sites that users visit regularly and blocks dangerous sites
    \item OhPhish: OhPhish is a web-based portal for testing employees' susceptibility to social engineering attacks. It is a phishing simulation tool that provides the organization with a platform to launch phishing simulation campaigns on its employees
    Apility.io: Apility.io is an anti-abuse API that helps security professionals to know if the IP address, domain, or email of a user is blacklisted. It is a collection of various tools delivered “as a service” to help security professionals, product managers, IT shops, enterprises, and start-ups to acquire more details about their potential visitors, users, customers, and threat actors.
    \item FaceNiff: FaceNiff is an Android app that allows a user to sniff and intercept web-session profiles over the WiFi network that the user's mobile device is connected to. Although FaceNiff can hijack sessions only when the WiFi network does not use the Extensible Authentication Protocol (EAP), it works on any private network, including open, Wired Equivalent Privacy (WEP), Wi-Fi Protected Access-pre-shared key (WPA-PSK), and WPA2-PSK networks.
    \item sslstrip: Sslstrip tool is exploiting user behavior and if a user does not type https:// in front of the link, and the website has redirection from HTTP to HTTPS, it will intercept HTTP 302 redirection and send the user exactly what the user asked for, i.e. HTTPsite
\end{itemize}
\subsubsection{Session Hijacking Countermeasures}
\begin{itemize}
    \item Tools:
    \begin{itemize}
        \item AlienVault USM
        \item Fiddler: Used for security testing of web applications such as  decrypting HTTPS traffic, and manipulating requests using man-in-th-middle decrpytion technique.
        \item BetterCAP: ARP poisoning
        \item MITMf: ARP poisoning
        \item Cain an Abel: ARP poisoning.
    \end{itemize}
    \item IPsec: used to secure VPN sessions
    \item IPsec Components:
    \begin{itemize}
        \item IPsec Driver: Software that performs protocol-level functions required to encrypt and decrypt packets.
        \item Internet Key Exchange (IKE): A protocol that produces security keys for IPsec and other protocols.
        \item Internet Security Association and Key Management Protocol (ISAKMP): Software that allows two computers to communicate by encrypting the data exchanged between them.
        \item Oakley: A protocol that uses Diffie-Hellman algorithm to create a master key and a key that is specific to each session in IPsec data transfer.
        \item IPsec Policy Agent:
    \end{itemize}
    \item IPsec architecture:
    \begin{itemize}
        \item Authentication Header (AH): Offers integrity and data origin authentication, with optional anti-replay featues.
        \item Encapsulating Security payload (ESP): Offers all the services offered by AH as well as confidentiality.
        \item IPsec Domain of Interpretation (DOI): Defines the payload formats, types of exchange, and naming conventions for security information such as cryptographic algorithms or security policies. IPsec DOI instantiates ISAKMP for use with IP when IP uses ISAKMP to negotiate security associations.
        \item IPsec Policies: useful in providing network security. Defines when and how to secure data, as well as security methods to use at different levels in the network. One can configure IPsec policies to meet the security requirements of a system, domain, site, organizational unit and so on.
    \end{itemize}
    \item HTTP Strict Transport Security (HSTS): a web security policy that protects HTTPS websites against MITM attacks. The HSTS policy helps web servers force web browsers to interact with them using HTTPS. With the HSTS policy, all insecure HTTP connections are automatically converted into HTTPS connections. This policy ensures that all the communication between a web server and a web browser is encrypted and that all responses that are delivered and recieved originate from an authenticated server.
    \item HTTP Public Key Pinning (HPKP): A trust on first use (TOFU) technique used in an HTTP header that allows a web client to associate a specific public key certificate with a prticular server to minimize the risk of MITM attacks based on fraudulent certificates. In TLS sessions, to verify the authenticity of a servers public key, the public key is enclosed in an X.509 digital certificate, which is signed by a certificate authority (CA).
    \item WEP/WPA Encryption: Wired Equivalent Privacy (WEP) and Wireless Protected Access (WPA) are wireless protocols that are intended to protect the traffic that is sent and received by users over a wireless network. The implementation of these protocols can thwart the attempts of unwanted users to connect to the network. A weak encryption mechanism enables attackers to brute force credentials and enter the target network to perform an MITM attack.
    \item Token Binding: When a user logs into a web application, a cookie with a session ID, called a token, is generated. The user utilizes this random token to send requests to the server and access resources. An attacker can impersonate the user and hijack the connection by capturing and reusing a valid session ID. Token binding protects client-server communication against session hijacking attacks. The client creates a public-private key pair for every connection to a remote server.
    
\end{itemize}


\subsection{Evading IDS, Firewalls, and Honeypots}
\subsubsection{IDS, IPS, Firewall and Honeypot Concepts}
\begin{itemize}
    \item Signature Recognition: also known as misuse detection, tries to identify events that indicate an abuse of a system or network resource. This technique involves first creating models of possible intrusions and then comparing these models with incoming events to make a detection decision.
    \item Protocol Anomaly Detection: Protocol anomaly detection depends on the anomalies specific to a protocol. It identifies particular flaws between how vendors deploy the TCP/IP protocol. Protocols designs according to RFC specifications, which dictate standard handshakes to permit universal communication. The protocol anomaly detector can identify new attacks.
    \item Anomaly Detection: Anomaly detection, or “not-use detection,” differs from the signature-recognition model. Anomaly detection consists of a database of anomalies. An anomaly can be detected when an event occurs outside the tolerance threshold of normal traffic. Therefore, any deviation from regular use is an attack. Anomaly detection detects the intrusion based on the fixed behavioral characteristics of the users and components in a computer system. Creating a model of normal use is the most challenging task in creating an anomaly detector.
    \item Obfuscating: Obfuscating is an IDS evasion technique used by attackers to encode the attack packet payload in such a way that the destination host can only decode the packet but not the IDS. An attacker manipulates the path referenced in the signature to fool the HIDS. Using the Unicode character, an attacker could encode attack packets that the IDS would not recognize, but an IIS web server would decode.
    \item Bastion Host: The bastion host is designed for defending the network against attacks. It acts as a mediator between inside and outside networks. A bastion host is a computer system designed and configured to protect network resources from attacks. Traffic entering or leaving the network passes through the firewall
    \item File System Intrusion: By observing system files, the presence of an intrusion can be identified. System files record the activities of the system.
    \begin{itemize}
        \item If you find new, unknown files / programs on youe system. Unexplained modification in file size are also an indication of an attack.
        \item You can identify unfamiliar file names in directories, including executable files with strange extensions and double extensions.
        \item Missing files are also a sign of a probable intrusion/attack.
    \end{itemize}
    \item Network Intrusions: general indications of network intrusions include the following
    \begin{itemize}
        \item A sudden increas in bandwidth consumption
        \item repeated probes of the available services on your machine.
        \item connection requests from IPs other than those in the network range, which imply that an unauthorized user (intruder) is attempting to connect to the network.
        \item Repeated login attempts from remote hosts.
        \item A sudden influx of log data, which could indicate attempts at DoS attakcs, bandwidth consumption, and DDoS attacks.
    \end{itemize}
    \item System Intrusions:
    \begin{itemize}
        \item sudden changes in logs such as short or incomplete logs.
        \item Unusually slow system performance.
        \item Missing logs or logs with incorrect permissions or ownership.
        \item Unusual graphic displays or text messages.
        \item Gaps in system accounting.
    \end{itemize}
    \item Signature recognition: is an IDS intrusion detection method, also known as misuse detection, tries to identify events that indicate an abuse of a system or network resource.
    \item Honeynet: Very effective in determining the entire capabilities of adversaries and is mostly deployed in an isolated virutal environment along with a combination of vulnerable servers?
    \item Packet information:
    \begin{itemize}
        \item Direction: Used to check whether the packet is entering or leaving the private network.
        \item Interface: Used to check whether the packet is coming from an unreliable zone.
        \item TCP flag bits: Used to check whether the packet has SYN, ACK, or other bits set for the connection to be made.
        \item Source IP address: Userd to check whether the packet is coming from a valid source. The information about the source IP address can be found from the IP header of the packet.
    \end{itemize}
    \item Circuit-level gateway firewall: The firewall is only monitoring TCP handshaking of packets at the session layer of the OSI model
    \item Stateful Multilayer Instection firewall: They filter packets at the network layer, to determine whether session packets are legitimate, and evaluate the contents of packets at the application layer. With the use of stateful packet filtering, you can overcome the limitation of packet firewalls that can only filter on IP address, port, protocol, and so on. This multilayer firewall can perform deep packet inspection.
    \item Application-level Firewall: Application-based proxy firewalls concentrate on the application layer rather than just the packets. The need for application-level firewall arises when huge amount of voice, video, and collaborative traffic are accessed at data-link layer and network layer utilized for unauthorized access to internal and external networks. Useful to filter specific commands such as \verb|http:post|
    \item Packet filtering firewall:  A packet filtering firewall investigates each individual packet passing through it and makes a decision whether to pass the packet or drop it. It works at the Internet protocol (IP) layer of the TCP/IP model. Packet filter–based firewalls concentrate on individual packets, analyze their header information, and determine which way they need to be directed.
    \item 
\end{itemize}
\subsubsection{IDS, IPS, Firewall, and Honeypot Solutions}
\begin{itemize}
    \item Wifiphisher: A rouge AP framework for conducting Red Team Engagements or WiFi security testing. Using Wifiphisher, penetration testers can easily achieve an MITM position against wireless clients by performing targeted WiFi association attacks.
    \item Reaver: designed to be a robust and practical tool against WiFi Protected Setup (WPS) registrar PINs in roder to recover WPA/WPA2 passphrases, and it has been tested against a wide variety of APs and WPS implementations.
    \item Wifi Inspector: Allows you to find all the devices connected to the network (via both wired and WiFi connections, including consoles, TVs, PCs, tablets, and phones); it gives relevant data such as the IP addresses, manufacturer names, and MAC addresses of connected devices. It also allows you to save a list of known devices with a custom name and finds intruders in a short period.
    \item WIBR+: application for testing the security of WPA/WPA2 PSK WiFi networks. It discovers weak passwords. WIBR+ supports queuing, custom dictionaries, a brute-force generator, and advanced monitoring.
    \item NetPatch firewall is a full-featured advanced android noroot firewall. It can be used to fully control over mobile device network. With NetPatch firewall, you can create network rules based on APP, IP address, domain name, and so on. This firewall is designed to save mobile device's network traffic and battery consumption, and improve network security and protect privacy.
    \item Comodo Firewall
    \item Glasswire
    \item TinyWall
    \item PeerBlock
    \item SPECTER: SPECTER is a honeypot. It automatically investigates attackers while they are still trying to break in. It provides massive amounts of decoy content, and it generates decoy programs that cannot leave hidden marks on the attacker's computer. Automated weekly online updates of the honeypot's content and vulnerability databases allow the honeypot to change regularly without user interaction.
    \item Vanguard Enforcer:
    \item zIPS: Zimperium's zIPS™ is a mobile intrusion prevention system app that provides comprehensive protection for iOS and Android devices against mobile network, device, and application cyber-attacks.
    \item ZoneAlarm PRO FIREWALL 2019: ZoneAlarm PRO Firewall blocks attackers and intruders from accessing your system. It monitors programs for suspicious behavior spotting and stopping new attacks that bypass traditional anti-virus protection. It prevents identity theft by guarding your data. It even erases your tracks allowing you to surf the web in complete privacy. Furthermore, it locks out attackers, blocks intrusions, and makes your PC invisible online. Also, it filters out an annoying and potentially dangerous email.
\end{itemize}
\subsubsection{Evading IDS}
\begin{itemize}
    \item Invalid RST Packets: The TCP uses 16-bit checksums for error checking of the header and data and to ensure that communication is reliable. It adds a checksum to every transmitted segment that is checked at the receiving end. When a checksum differs from the checksum expected by the receiving host, the TCP drops the packet at the receiver's end. The TCP also uses an RST packet to end two-way communications. Attackers can use this feature to elude detection by sending RST packets with an invalid checksum.
    \item Fragmentation attack: Fragmentation can be used as an attack vector when fragmentation timeouts vary between the IDS and the host. Through the process of fragmenting and reassembling, attackers can send malicious packets over the network to exploit and attack systems.
    \item Obfuscating: It is an IDS evasion technique used by attackers to encode the attack packet payload in such a way that the destination host can only decode the packet but not the IDS. An attacker manipulates the path referenced in the signature to fool the HIDS. Using Unicode characters, an attacker can encode attack packets that the IDS would not recognize but which an IIS web server can decode
    \item Insertion Attack: Insertion is the process by which the attacker confuses the IDS by forcing it to read invalid packets (i.e., the system may not accept the packet addressed to it). An IDS blindly trusts and accepts a packet that an end system rejects. If a packet is malformed or if it does not reach its actual destination, the packet is invalid. If the IDS reads an invalid packet, it gets confused. An attacker exploits this condition and inserts data into the IDS.
    \item Flooding: an IDS evasion technique used by an attacker to send a huge amount of unnecessary traffic to produce noise or fake traffic. If the IDS does not analyze the noise traffic, the true attack traffic goes undetected.
    \item Overlapping fragments:
    \item Encryption:
    \item Polymorphic shellcode:  an attacker use an existing buffer-overflow exploit and set the “return” memory address on the overflowed stack to the entrance point of the decryption code.
    \item Session Splicing: Attacker splits the attack traffic into an excessive number of packets sudh that no single packet triggers the IDS.
\end{itemize}
\subsubsection{Evading Firewalls}
\begin{itemize}
    \item Tools
    \begin{itemize}
        \item Snort: Snort is an open-source network intrusion detection system capable of performing real-time traffic analysis and packet logging on IP networks. It can perform protocol analysis and content searching/matching, and it is used to detect a variety of attacks and probes, such as buffer overflows, stealth port scans, CGI attacks, SMB probes, and OS fingerprinting attempts.
        \item Suricata: Suricata is a robust network threat detection engine capable of real-time intrusion detection (IDS), inline intrusion prevention (IPS), network security monitoring (NSM), and offline pcap processing.
        \item Bitvise: Bitvise SSH Server provides secure remote login capabilities to Windows workstations and servers by encrypting data during transmission. It is ideal for remote administration of Windows servers, for advanced users who wish to access their home machine from work or their work machine from home, and for a wide spectrum of advanced tasks, such as establishing a VPN using the SSH TCP/IP tunneling feature or providing a secure file depository using SFTP.
        \item HTTPTunnel: Uses technique of tunneling traffic across TCP port 80 to bypass firewall.
        \item Loki: ICMP tunneling is used to execute commands of choice by tunneling them inside the payload of ICMP echo packets.
        \item AckCmd: (http://ntsecurity.nu) use ACK tunneling
        \item Super Network Tunnel: a two-way HTTP tunneling software that connects two computers utilizeing HTTP-tunnel client and HTTP-tunnel server. It can bypass any firewall to surf the web, use IM applications, games, and so on. Super network tunnel integrates SocksCap function along with  bidirectional HTTP tunneling and remote control to simplify the configuration.
        \item SecurePipes: SSH tunneling tool
        \item Traffic IQ Professional: Traffic IQ Professional is a tool that audits and validates the behavior of security devices by generating the standard application traffic or attack traffic between two virtual machines. This tool is generally used by security personnel for assessing, auditing, and testing the behavioral characteristics of any non-proxy packet-filtering device, which can include application firewalls, IDS, IPS, routers, switches, etc. However, as this tool can generate custom attack traffic, it is extensively employed by attackers to bypass the installed perimeter devices in the target network.
        \item Colasoft Packet Builder: Colasoft Packet Builder: Colasoft Packet Builder is used to create custom network packets and fragmenting packets. Attackers use this tool to create custom malicious packets and fragment them such that firewalls cannot detect them. They can create custom network packets such as Ethernet Packet, ARP Packet, IP Packet, TCP Packet, and UDP Packet. Security professionals use this tool to check your network's protection against attacks and intruders.
    \end{itemize}
    \item Firewalking is a method of collecting information about remote netowrks behind firewalls. It is a technique that uses TTL values to determine gateway ACL filters and map networks by analyzing the IP packet response.
    \item Banner Grabbing: A simple method of fingerprinting that helps in detecting the vendor of a firewall and the firmware version. It identifies the service running on the system. Attackers use banner grabbing to fingerprint services and thus discover the services running on firewall.
    \item IP address spoofing: a hijacking technique in which an attacker masquerades as a trusted host to conceal his identity, spoof a website, hijack browsers, or gain unauthorized access to a network. In IP spoofing, the attacker creates IP packets by using a forged IP address and gains access to the system or network without authorization.
    \item Tiny fragments: Attackers create tiny fragments of outgoing packets, forcing some of the TCP packet's header information to go into the next fragment. The IDS filter rules that specify patterns will not match with the fragmented packets owing to the broken header information. The attack will succeed if the filtering router examines 
    \item ACK Tunneling method: Allows tunneling a backdoor application with TCP packets with the ACK bit set. The ACK bit is used to acknowledge the reciept of a packet. Some firewalls do not check packets with the ACK bit set because ACK bits are supposed to be used in response to legitimate traffic.
    \item source routing: using this technique, the sender of the packet designates the route (partially or entirely) that a packet should take through the network such that the designated route should bypass the firewall node. Thus the attack can evade firewall restrictions.
    \item Anonymizer: Anonymizer's VPN routes all traffic through an encrypted tunnel directly from your laptop to secure and harden servers and networks. It then masks the real IP address to ensure complete and continuous anonymity for all online activities.
\end{itemize}
\subsubsection{Honeypot, IDS, and Firewall Evasion Countermeasures}
\begin{itemize}
    \item Tools:
    \begin{itemize}
        \item Sebek: catches read() system calls.
    \end{itemize}
\end{itemize}

\subsection{Hacking Web Servers}
\subsubsection{Web Server Concepts}
\begin{itemize}
    \item Document Root: The document root is one of the root file directories of the web server that stores critical HTML files related to the web pages of a domain name, which will be sent in response to requests.
    \item Server Root: It is the top-level root directory under the directory tree in which the server's configuration and error, executable, and log files are stored.
    \item Virtual Hosting: It is a technique of hosting multiple domains or websites on the same server. This technique allows the sharing of resources among various servers.
    \item \item Virtual Document Tree: A virtual document tree provides storage on a different machine or disk after the original disk becomes full.
    \item Data Tampering: alters or deletes the data of a web server and replaces the data with malware.
    \item Web Proxy: A proxy server is located between the web client and web server. Owing to the placement of web proxies, all requests from clients are passed on to the web server through the web proxies. They are used to prevent IP blocking and maintain anonymity
    \item PHP: application layer and is used to generate dynamic web content.
\end{itemize}
\subsubsection{Web Server Attacks}
\begin{itemize}
    \item Session hijacking attacks:
    \begin{itemize}
        \item Session fixation
        \item Session sidejacking
        \item Cross-site scripting
    \end{itemize}
    \item DNS Hijacking: malicious attack that modifies or overrides a systems TCP/IP settings to redirect it at a rouge DNS server, thereby invalidating the default DNS settings.
\end{itemize}
\subsubsection{Web Server Attack Methodology}
\begin{itemize}
    \item Tools:
    \begin{itemize}
        \item NCollector Studio: a website mirroring tool used to download content from the web to a local computer. This tool enables users to crawl for specific file types, make any website available for offline browsing, or simply download a website to a local computer.
        \item ID Server: A simple internet server identification utility also performs HTTP Server Identification, Non-HTTP Server Identification and Reverse DNS lookup.
        \item Open Sez Me: A lookup database for default passwords, credentials and ports.
        \item HTTrack: HTTrack is an offline browser utility that is capable of performing website mirroring by downloading a website from the Internet to a local directory, building all the directories recursively, and getting HTML, images, and other files from the server.
        \item Nessus: Vulnerability scanner
        \item Hydra: Password Cracker
    \end{itemize}
\end{itemize}
\subsubsection{Web Server Attack Countermeasures}
\begin{itemize}
    \item Tools:
    \begin{itemize}
        \item \item Mimikatz: Allows attackers to pass Kerberos TGT to other computer and sign in using the victim's ticket. The tool also helps in extracting paintext passwords, hashes, PIN codes, and Kerberos tickets from memory.
        \item N-Stalker: N-Stalker is a web application security scanner that searches for vulnerabilities to attacks such as clickjacking, SQL injection, and XSS. It allows spider crawling throughout the application and the creation of web macros for form authentication. It also provides proxy capabilities for “drive-thru” attacks and identifies components through reverse proxies that distribute different platforms in the same application URL.
        \item Immunity Debugger: tool used to write exploits, analyze malware, and reverse engineer binary files.
        \item Fortify WebInspect: Webserver security tools
        \item Retina CS: Webserver security Tool
        \item NetIQ secure configuration manager: Webserver security tool.
    \end{itemize}
\end{itemize}
\subsubsection{Patch Management}
\begin{itemize}
    \item Tools:
    \begin{itemize}
        \item 
    \end{itemize}
\end{itemize}







\subsection{General / Unsorted}
\begin{itemize}
    \item CIA Triad:
    \begin{itemize}
        \item Confidentiality: unauthorized access to information.
        \item Integrity: Trustworthiness of data
        \item Availability: accessible when required
        \item (Other) Non-repudiation: Sender of a message cannot deny having sent the message, same for receiver.
        \item (Other) Authenticity: quality of being genuine
    \end{itemize}
    \item OSI model - Open System Interconnection model
    \item Local Area Network (LAN): Computer network that connects two or more computers within a limited area.
    \item Virtual Local Area Network (VLAN): Broadcast domain that is divided in a computer network at the data link layer (OSI layer 2).
    \item Wide Area Network (WAN): Covers larger area than a LAN, typically involves telecommunication circuits for a special purpose, ie: banking network. Nodes are more than 10 miles apart.
    \item Time to live (TTL): time period a message can live on the network before it is discarded. (8-bits). Number of seconds or number of hops?
    \item User Datagram Protocol (UDP): light weight communication protocol that gives no assurance of delivery.
    If the application receives out of order packets they are destroyed rather than worrying about reordering them.
    \item Transmission Control Protocol (TCP):
    \item Internet of Things (IoT): Devices with embedded software and network access.
   
    \item Malware: software created to harm or infiltrate a computer system without the owners consent.
    \begin{itemize}
        \item Virus: Create copies of themselves in other programs and activate from a trigger event.
        \item Worm
        \item Spyware
        \item Trojan
    \end{itemize}
    \item Information Security Policy: set of rules sanction by an organization to ensure that user of networks abide by the prescriptions regarding the security of data stored within the boundaries of the organization.
    \item Event: Something that happens that is detectable
    \item Incident: an event that violates policy.
    \item Certificate Authority: Organization that issues digital certificates.
    \item Vulnerability Scanner: Computer program designed to assess computer systems, network or applications for known weaknesses.
    \item Uniform Resource Locator (URL): reference to a web resource. Is a specific type of URI.
    \item Uniform Resource Identifier (URI): Unique sequence of characters that identifies a logical or physical resource used by web technologies. the \verb|http://| part of the url.
    \item DNS Zone transfer: Used to duplicate or make copies of DNS data across a number of DNS servers or to back up DNS files.
    \item Open-source intelligence: to describe identifying information about a target using freely available sources.
    \item Defence in breadth: planned, systematic set of multi-disciplinary activities that seek to identify, manage, and reduce risk of exploitable vulnerabilities at every stage of the system, network, or sub-component lifecycle.
    \item Defence in depth (DiD): Information security approach in which a series of security mechanisms and controls are layered throughout a computer network.
    \item Lawful Interception: Process of legally intercepting communications betwen two or more parties for surveillance on telecommunications, VoIP, data, and multiservice networks.
    \item Internet Zones
    \begin{itemize}
        \item Internet (uncontrolled zone): outside the boundary of your organization.
        \item Internet DMZ (controlled zone): Internet-facing controlled zone that contains components in which clients may directly communicate with. Usually buffered by two firewalls one from internet to DMZ and one from DMZ to the internal network.
        \item Production network (restricted zone): A restricted zone supports functions to which access must be strictly controlled; direct access from an uncontrolled network should not be permitted. In a large enterprise, several network zones might be designated as restricted. As with an internet DMZ, a restricted zone is typically bounded by one or more firewalls that filter incoming and outgoing traffic.
        \item Intranet (controlled zone): is not heavily restricted in use, but an appropriate span of control is in place to assure that network traffic does not compromise the operation of critical business functions.
        \item Management network (secured zone): In a secured zone, access is tightly controlled and available to only to a small number of authorized users. Access to one area of the zone does not necessarily apply to another area of the zone.
    \end{itemize}
\end{itemize}

\subsection{Attacks}
\begin{itemize}
    \item SQL Injection:
    \begin{itemize}
        \item In-band SQL Injection: Attacker uses the same communication channel to launch the attack and gather results. (error-based and union-based SQL injection).
    \end{itemize}
    \item Bluetooth
    \begin{itemize}
        \item Bluesnarfing: Theft of information from a target device using a bluetooth connection.
        \item Bluejacking: Transmission of data to a target device using a bluetooth connection.
    \end{itemize}
    \item Operating System Attacks
    \item Application-Level Attacks
    \item Shrink Wrap Code Attacks
    \item Misconfiguration Attacks
    \item DHCP starvation attack: Broadcasting DHCP requests with spoofed MAC addresses to expend the available address pool, denying access to new users.
    \item MAC flooding attack: Attacker floods the switch MAC table to push legitimate MAC addresses out of the switch. This causes significant amounts of frames to be broadcasted to all ports.
\end{itemize}

\subsection{Organizations}
\begin{itemize}
    \item Open Web Application Security Project (OWASP): International non-profit organization focused on web application security.
    \item Federal Risk and Authorization Management Program (FedRAMP): Cloud computing regulatory effort, government-wide, delivers systemized approach to security assessment, authorization, and continuous monitoring of cloud products and services.
\end{itemize}

\subsection{Cloud computing}
\begin{itemize}
    \item Platform as a service (PaaS): Third-party provider delivers hardware and software tools to users over the internet. PaaS frees developers from having to install in-house hardware and software to develop or run a new application.
    \item Infrastructure as a Service (IaaS):
    \item Hardware as a Service (HaaS):
    \item Software as a Service (SaaS):
    \item Models:
    \begin{itemize}
        \item Private
        \item Public
        \item Community: Infrastructure is shared by several organizations, usually with the same policy and compliance considerations.
        \item Hybrid
    \end{itemize}
\end{itemize}

\subsection{Cryptography}
\begin{itemize}
    \item Ciphers
    \begin{itemize}
        \item Symmetric Ciphers: Single key is used for encryption and decryption
        \begin{itemize}
            \item Data Encryption Standard (DES): Symmetric-key block cipher with key size of 56-bits
            \item Triple Data Encryption Algorithm (3DES, TDES, TDEA): Applies the DES algorithm 3 times to each data block. Key length of \(56 \times 3 = 168\) bits when 3 independent keys are used, or 112 when two keys are independent.
        \end{itemize}
        \item Asymmetric Ciphers (Public key cryptography): One key can encrypt and one key can decrypt.
        \begin{itemize}
            \item
        \end{itemize}
    \end{itemize}
\end{itemize}

\subsection{Registers}
\begin{itemize}
    \item EIP - Extended Instruction Pointer stores the address of the next instruction to be executed.
    \item ESP - Stack pointer, contains the address of the next element to be stored onto the stack.
    \item EBP - Extended Base pointer (StackBase), contains the address of the bottom (first element) of the stack frame.
    \item EDI - Destination Index, used with string instruction.
    \item ESI - Source Index, used with string instruction.
\end{itemize}
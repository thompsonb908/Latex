\section{NTP Information}
Network Time Protocol
\subsection{NTP Commands}
\begin{itemize}
    \item \verb|ntpdate| sets the local date and time by polling the NTP server(s).
    \begin{itemize}
        \item \verb|-B|: Forces the time to always be slewed using the adjtime() system call.
        \item \verb|-d|: Enables the debugging mode, \verb|ntpdate| will go through all the steps but not adjust the local clock.
        \item \verb|-b|: Force time to be stepped using the settimeofday() system call, rather than slewed (\verb|-B|).
        \item \verb|-q|: Query only - don't set the clock.
    \end{itemize}
    \item \verb|ntptrace| Determines from where the NTP server gets time and follows the chain of NTP servers back to its prime time source.
    \begin{itemize}
        \item 
    \end{itemize}
    \item \verb|ntpdc| Queries the ntpd daemon about its current state and requests changes in that state.
    \begin{itemize}
        \item
    \end{itemize}
    \item \verb|ntpq| Monitors NTP daemon ntpd operations and determines performance.
\end{itemize}
\section{Introduction to Ethical Hacking}
\subsection{Module Objectives}
\begin{itemize}
    \item Understand elements of information security.
    \item Understand information security attacks and information warfare.
    \item Overview of cyber kill chain methodology, TTps, and IoCs.
    \item Overview of hacking concepts, types, and phases.
    \item Understanding ethical hacking concepts and its scope.
    \item Overview of information security controls.
    \item Overview of information security acts and laws.
\end{itemize}

\subsection{Information Security Overview}

\subsubsection{Elements of Information Security}
\begin{itemize}
    \item \textbf{Confidentiality}\\
    Confidentiality is the assurance that the information is accessible only to authorized users.
    Control methods are data classification, data encryption, and proper disposal of equipment.
    \item \textbf{Integrity}\\
    Integrity is the trustworthiness of data or resources in the prevention of improper and unauthorized changes--the assurance that data is accurate. 
    Control methods are checksums and access control.
    \item \textbf{Availability}\\
    Availability is the assurance that systems are accessible when required by authorized users.
    Methods to maintain data availability can include disk arrays for redundant systems and clustered machines, antivirus software to combat malware, and distributed denial-of-service (DDoS) prevention systems.
    \item \textbf{Authenticity}\\
    Authenticity refers to the characteristics of communication, documents, or any data that ensures the quality of being genuine or uncorrupted.
    The major role of authentication is to confirm that the user is genuine.
    Control methods include biometrics, smart cards, and digital certificates.
    \item \textbf{Non-Repudiation}\\
    Non-Repudiation is a way to guarantee that the sender of a message cannot later deny having sent the message and that the recipient cannot deny having received the message.
    Individuals and organizations use digital signatures to ensure non-repudiation.
\end{itemize}

\subsubsection{Motives, Goals and Objectives of Information Security Attacks}

\begin{equation*}
    \text{Attack} = \text{Motive (Goal)} + \text{Method} + \text{Vulnerability}
\end{equation*}
A motive originates from the notion that the target system stores or processes something valuable, this leads to the threat of an attack on the system.

\subsubsection{Motives}
\begin{itemize}
    \item Disrupt business continuity
    \item Perform information theft
    \item Manipulate data
    \item Create fear and chaos by disrupting critical infrastructures
    \item Bring financial loss to the target
    \item Propagate religious or political beliefs
    \item Achieve a state's military Objectives
    \item Damage the reputation of the target
    \item Take revenge
    \item Demand ransom
\end{itemize} 

\subsubsection{Classification of Attacks}
\begin{itemize}
    \item Passive Attacks: Monitor network traffic for reconnaissance on network activities using sniffers.
    used for gathering data useful in active attacks.
    \item Active Attacks: Tamper with data in transit or disrupt communication or services between systems to bypass or break into secured systems.
    Attackers launch an attack on the target system by sending traffic actively that can be detected.
    \item Close-in Attacks: Attacker is in close proximity to the target.
    Used to gather or modify information or disrupt its access.
    \item Insider Attacks: Performed by trusted persons who have physical access to critical assets of the target.
    \item Disruption Attacks: Attackers tamper with hardware or software prior to installation.
\end{itemize}

\subsubsection{Information warfare}
Refers to the use of information and communication technologies (ICT) for competitive advantages over an opponent.
\begin{itemize}
    \item Command and control warfare (C2 warfare)
    \item Intelligence-based warfare
    \item Electronic warfare
    \item Psychological warfare
    \item Hacker warfare
    \item Economic warfare
    \item Cyberwarfare
\end{itemize}

\subsection{Cyber Kill Chain Concepts}
The cyber kill chain is a way to illustrate how attacks occur and possible threats at different stages of an attack as well as countermeasures.

\subsubsection{Cyber Kill Chain Methodology}
A component of intelligence based defense for the identification and prevention of malicious intrusion activities.

Attacks can happen in seven phases:
\begin{itemize}
    \item Reconnaissance: collection of information about target to probe for weaknesses.
    \begin{itemize}
        \item Public information on the internet
        \item Network information
        \item system information
        \item organizational information
    \end{itemize}
    \item Weaponization: identification of vulnerabilities based on data collected.
    \begin{itemize}
        \item Identify appropriate malware
        \item create payload
        \item deliver to target
        \item leverage exploits
    \end{itemize}
    \item Delivery: Measures the effectiveness of security controls implemented by the target based on whether or not the intrusion attempt succeeds.
    \begin{itemize}
        \item Phishing emails
        \item USB drives
        \item Website attacks
        \item Hacking tools against operating systems, applications,\dots
    \end{itemize}
    \item Exploitation: Trigger of the malicious code to exploit the vulnerability
    \item Installation: adversary downloads and installs more malicious software on the target system to maintain access to the network for an extended period.
    \item Command and Control: adversary creates a command and control channel that establishes two-way communication.
    \item Actions on Objectives: Adversary controls the victim system and gains access to confidential data, disrupts services or network, or destroys operational capability of the target.
    May use this as a launching point for new attacks.
\end{itemize}

\subsubsection{Tactics, Techniques, and Procedures (TTPs)}
TTPs refer to the patterns of activities and methods associated with specific threat actors.
\begin{itemize}
    \item Tactics: The way a threat actor operates during the different phases of the attack.
    \item Techniques: Technical methods used by an attacker to achieve intermediate results during the attack.
    \item Procedures: Organizational approaches that threat actors follow to launch an attack.
\end{itemize}
TTP helps identify and profile attackers or APTs and learn more about how attacks occur.

\subsubsection{Adversary Behavioral identification}
Identification of the common methods or techniques followed by an adversary to launch attacks on or to penetrate an organization's network.
Gives security professionals insight into upcoming threats and exploits.
\begin{itemize}
    \item Internal Reconnaissance: Methods used once inside a target network for enumeration.
    Monitor activity by checking for unusual commands and packet capturing tools.
    \item Use of PowerShell: Automating data exfiltration.
    Can check the PowerShell logs or Windows Event logs.
    \item Unspecified Proxy activities
    \item Use of command-line interface
    \item HTTP user agent
    \item Command and Control server
    \item Use of DNS tunneling
    \item Use of web shell
    \item Data staging
\end{itemize}

\subsubsection{Indicators of compromise (IoCs)}
Clues, artifacts, and pieces of forensic data found on the network or operating system of an organization that indicate a potential intrusion.
\begin{itemize}
    \item Email Indicators
    \begin{itemize}
        \item Sender's email address
        \item Email subject
        \item attachments or links
    \end{itemize}
    \item Network Indicators
    \begin{itemize}
        \item URLs
        \item Domain names
        \item IP addresses
    \end{itemize}
    \item Host-Based Indicators
    \begin{itemize}
        \item Filenames
        \item File Hashes
        \item Registry keys
        \item DLLs
        \item Mutex
    \end{itemize}
    \item Behavioral Indicators
    \begin{itemize}
        \item Document executing PowerShell script
        \item Remote command execution
    \end{itemize}
\end{itemize}

\subsubsection{Key Indicators of Compromize (IoCs)}
\begin{itemize}
    \item Unusual outbound network traffic
    \item Unusual activity through a privileged user account
    \item Geographical anomalies
    \item Multiple login failures
    \item Increased database read volume 
    \item Large HTML response size
    \item Multiple requests for the same file
    \item Mismatched port-application traffic
    \item Suspicious registry or system file changes
    \item Unusual DNS requests
    \item Unexpected patching of systems
    \item Signs of Distributed Denial-of-Service (DDoS) activity
    \item Bundles of data in the wrong place
    \item Web traffic win superhuman behavior
\end{itemize}

\subsection{Hacking Concepts}
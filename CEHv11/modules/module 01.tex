\section{Introduction to Ethical Hacking}
\subsection{Module Objectives}
\begin{itemize}
    \item Understand elements of information security.
    \item Understand information security attacks and information warfare.
    \item Overview of cyber kill chain methodology, TTps, and IoCs.
    \item Overview of hacking concepts, types, and phases.
    \item Understanding ethical hacking concepts and its scope.
    \item Overview of information security controls.
    \item Overview of information security acts and laws.
\end{itemize}

\subsection{Information Security Overview}

\subsubsection{Elements of Information Security}
\begin{itemize}
    \item \textbf{Confidentiality}\\
    Confidentiality is the assurance that the information is accessible only to authorized users.
    Control methods are data classification, data encryption, and proper disposal of equipment.
    \item \textbf{Integrity}\\
    Integrity is the trustworthiness of data or resources in the prevention of improper and unauthorized changes--the assurance that data is accurate. 
    Control methods are checksums and access control.
    \item \textbf{Availability}\\
    Availability is the assurance that systems are accessible when required by authorized users.
    Methods to maintain data availability can include disk arrays for redundant systems and clustered machines, antivirus software to combat malware, and distributed denial-of-service (DDoS) prevention systems.
    \item \textbf{Authenticity}\\
    Authenticity refers to the characteristics of communication, documents, or any data that ensures the quality of being genuine or uncorrupted.
    The major role of authentication is to confirm that the user is genuine.
    Control methods include biometrics, smart cards, and digital certificates.
    \item \textbf{Non-Repudiation}\\
    Non-Repudiation is a way to guarantee that the sender of a message cannot later deny having sent the message and that the recipient cannot deny having received the message.
    Individuals and organizations use digital signatures to ensure non-repudiation.
\end{itemize}

\subsubsection{Motives, Goals and Objectives of Information Security Attacks}

\begin{equation*}
    \text{Attack} = \text{Motive (Goal)} + \text{Method} + \text{Vulnerability}
\end{equation*}
A motive originates from the notion that the target system stores or processes something valuable, this leads to the threat of an attack on the system.

\subsubsection{Motives}
\begin{itemize}
    \item Disrupt business continuity
    \item Perform information theft
    \item Manipulate data
    \item Create fear and chaos by disrupting critical infrastructures
    \item Bring financial loss to the target
    \item Propagate religious or political beliefs
    \item Achieve a state's military Objectives
    \item Damage the reputation of the target
    \item Take revenge
    \item Demand ransom
\end{itemize} 

\subsubsection{Classification of Attacks}
\begin{itemize}
    \item Passive Attacks: Monitor network traffic for reconnaissance on network activities using sniffers.
    used for gathering data useful in active attacks.
    \item Active Attacks: Tamper with data in transit or disrupt communication or services between systems to bypass or break into secured systems.
    Attackers launch an attack on the target system by sending traffic actively that can be detected.
    \item Close-in Attacks: Attacker is in close proximity to the target.
    Used to gather or modify information or disrupt its access.
    \item Insider Attacks: Performed by trusted persons who have physical access to critical assets of the target.
    \item Disruption Attacks: Attackers tamper with hardware or software prior to installation.
\end{itemize}

\subsubsection{Information warfare}
Refers to the use of information and communication technologies (ICT) for competitive advantages over an opponent.
\begin{itemize}
    \item Command and control warfare (C2 warfare)
    \item Intelligence-based warfare
    \item Electronic warfare
    \item Psychological warfare
    \item Hacker warfare
    \item Economic warfare
    \item Cyberwarfare
\end{itemize}

\subsection{Cyber Kill Chain Concepts}
The cyber kill chain is a way to illustrate how attacks occur and possible threats at different stages of an attack as well as countermeasures.

\subsubsection{Cyber Kill Chain Methodology}
A component of intelligence based defense for the identification and prevention of malicious intrusion activities.

Attacks can happen in seven phases:
\begin{itemize}
    \item Reconnaissance: collection of information about target to probe for weaknesses.
    \begin{itemize}
        \item Public information on the internet
        \item Network information
        \item system information
        \item organizational information
    \end{itemize}
    \item Weaponization: identification of vulnerabilities based on data collected.
    \begin{itemize}
        \item Identify appropriate malware
        \item create payload
        \item deliver to target
        \item leverage exploits
    \end{itemize}
    \item Delivery: Measures the effectiveness of security controls implemented by the target based on whether or not the intrusion attempt succeeds.
    \begin{itemize}
        \item Phishing emails
        \item USB drives
        \item Website attacks
        \item Hacking tools against operating systems, applications,\dots
    \end{itemize}
    \item Exploitation: Trigger of the malicious code to exploit the vulnerability
    \item Installation: adversary downloads and installs more malicious software on the target system to maintain access to the network for an extended period.
    \item Command and Control: adversary creates a command and control channel that establishes two-way communication.
    \item Actions on Objectives: Adversary controls the victim system and gains access to confidential data, disrupts services or network, or destroys operational capability of the target.
    May use this as a launching point for new attacks.
\end{itemize}

\subsubsection{Tactics, Techniques, and Procedures (TTPs)}
TTPs refer to the patterns of activities and methods associated with specific threat actors.
\begin{itemize}
    \item Tactics: The way a threat actor operates during the different phases of the attack.
    \item Techniques: Technical methods used by an attacker to achieve intermediate results during the attack.
    \item Procedures: Organizational approaches that threat actors follow to launch an attack.
\end{itemize}
TTP helps identify and profile attackers or APTs and learn more about how attacks occur.

\subsubsection{Adversary Behavioral identification}
Identification of the common methods or techniques followed by an adversary to launch attacks on or to penetrate an organization's network.
Gives security professionals insight into upcoming threats and exploits.
\begin{itemize}
    \item Internal Reconnaissance: Methods used once inside a target network for enumeration.
    Monitor activity by checking for unusual commands and packet capturing tools.
    \item Use of PowerShell: Automating data exfiltration.
    Can check the PowerShell logs or Windows Event logs.
    \item Unspecified Proxy activities
    \item Use of command-line interface
    \item HTTP user agent
    \item Command and Control server
    \item Use of DNS tunneling
    \item Use of web shell
    \item Data staging
\end{itemize}

\subsubsection{Indicators of compromise (IoCs)}
Clues, artifacts, and pieces of forensic data found on the network or operating system of an organization that indicate a potential intrusion.
\begin{itemize}
    \item Email Indicators
    \begin{itemize}
        \item Sender's email address
        \item Email subject
        \item attachments or links
    \end{itemize}
    \item Network Indicators
    \begin{itemize}
        \item URLs
        \item Domain names
        \item IP addresses
    \end{itemize}
    \item Host-Based Indicators
    \begin{itemize}
        \item Filenames
        \item File Hashes
        \item Registry keys
        \item DLLs
        \item Mutex
    \end{itemize}
    \item Behavioral Indicators
    \begin{itemize}
        \item Document executing PowerShell script
        \item Remote command execution
    \end{itemize}
\end{itemize}

\subsubsection{Key Indicators of Compromize (IoCs)}
\begin{itemize}
    \item Unusual outbound network traffic
    \item Unusual activity through a privileged user account
    \item Geographical anomalies
    \item Multiple login failures
    \item Increased database read volume 
    \item Large HTML response size
    \item Multiple requests for the same file
    \item Mismatched port-application traffic
    \item Suspicious registry or system file changes
    \item Unusual DNS requests
    \item Unexpected patching of systems
    \item Signs of Distributed Denial-of-Service (DDoS) activity
    \item Bundles of data in the wrong place
    \item Web traffic win superhuman behavior
\end{itemize}

\subsection{Hacking Concepts}
Hacking refers to exploiting system vulnerabilities and compromising security controls to gain unauthorized or inappropriate access to a system's resources.
A hacker is a person who breaks into a system or network without authorization for malicious intent.

\subsubsection{Hacker Classes}
\begin{itemize}
    \item Black Hats: Bad guys
    \item White Hats: Good guys
    \item Gray Hats: In between good/bad
    \item Suicide Hackers: Do not care if they get caught
    \item Script Kiddies: Unskilled, uses prebuilt tools
    \item Cyber Terrorists: Religious or political, cause fear.
    \item State-sponsored hackers: Employed by the government.
    \item Hacktivist: Promote political agenda
\end{itemize}

\subsubsection{Hacking Phases}
There are 5 general phases of hacking:
\begin{enumerate}
    \item Reconnaissance\\
    Preparation phase to gain as much information about the target as possible prior to launching an attack.
    \begin{itemize}
        \item Passive: Attacker does not interact with the target directly, relies on publicly available information.
        \item Active: Direct interaction with the target. Using tools to scan for open ports, hosts, router locations, network mapping, operating system details, and applications.
    \end{itemize}
    \item Scanning (enumeration)\\
    Uses details from reconnaissance to scan the network for specific information.
    Scanning is a logical extension of active reconnaissance and often lumped with the reconnaissance phase.
    \item Gaining access\\
    Attacker gains access to the operating system or applications on the network.
    Examples are password cracking, buffer overflows, denial of service, and session hijacking.
    \item Maintaining access\\
    Attacker tries to retain ownership (root level access) of the system by installing backdoors, rootkits and trojans.
    \item Clearing tracks\\
    Erasing evidence of malicious activities, remain unnoticed and uncaught.
    Overwrite server, system and application logs to avoid suspicion.
\end{enumerate}

\subsection{Ethical Hacking Concepts}
Ethical hackers follow similar processes as malicious hackers.
Ethical hackers are employed to assist organizations in testing network security for possible loopholes and vulnerabilities.
The noun hacker refers to a person who enjoys learning the details of computer systems and stretching their capabilities.
The verb to hack describes the rapid development of new programs or the reverse engineering of existing software to make it better or more efficient in new and innovative ways.
The terms cracker and attacker refer to the persons who employ their hacking skills for offensive purposes.
The distinction between ethical hackers and crackers is consent.

\subsubsection{Why is ethial hacking neccessary}
\begin{itemize}
    \item Prevent hackers from gaining access.
    \item Uncover vulnerabilities.
    \item Strengthen security.
    \item Preventative measures.
    \item Safeguard data.
    \item Enhance security awareness.
    \item What can an intruder see?
    \item what can an intruder do?
    \item Does anyone notice?
    \item Are all components accounted for and patched?
    \item How much effort would it take to protect the system?
    \item Are the security measures compliant with legal standards?
\end{itemize}

\subsubsection{Scope and limitations}
\begin{itemize}
    \item Scope
    \begin{itemize}
        \item Risk assessment
        \item auditing
        \item counter fraud
        \item security best practices.
        \item highlight remedial actions.
    \end{itemize}
    \item Limitations
    \begin{itemize}
        \item Not much to gain without cause.
        \item Can only help the organization understand its security system. (Up to organization to implement remediation's).
    \end{itemize}
\end{itemize}

\subsubsection{Skills of an Ethical Hacker}
\begin{itemize}
    \item Technical Skills
    \begin{itemize}
        \item operating systems.
        \item networking concepts
        \item Computer expert
        \item Security areas and related issues
        \item How to launch sophisticated attacks.
    \end{itemize}
    \item Non-technical skills
    \begin{itemize}
        \item Quickly learn and adapt
        \item String work ethic
        \item problem solving skills
        \item Commitment to organizations security policies
        \item Awareness of local standards and laws.
    \end{itemize}
\end{itemize}

\subsection{Information security controls}
Security controls prevent the occurrence of unwanted events and reduce risk to assets.
Basic security concepts are Confidentiality, Integrity, Availability (CIA).
Concepts related to users accessing information are authentication, authorization, and non-repudiation.
This section covers Information Assurance (IA), defence in depth, risk management, cyber threat intelligence, threat modeling, incident management and AI and ML concepts. 

\subsubsection{Information Assurance (IA)}
Assurance of Confidentiality, availability, integrity and authenticity of information and information systems is protected during the usage , processing, storage and transmission of information.
\begin{itemize}
    \item Develop local policies, processes and guidance.
    \item Designing network and user authentication strategies.
    \item Identifying network vulnerabilities and threats.
    \item Identifying problem and resource requirements.
    \item Creating plans for identified resource requirements.
    \item Applying appropriate information assurance controls.
    \item Performing certification and accreditation
    \item Providing information assurance training.
\end{itemize}

\subsubsection{What is Risk?}
Degree of uncertainty or expectation that an adverse event may cause damage to the system.
Risks are categorized into  different levels according to their estimated impact on the system.
A risk matrix is used to scale risk according to the probability, likelihood, and consequence or impact of the risk. 
\begin{itemize}
    \item Probability of the occurrence of a threat or and event that will damage the organization.
    \item Possibility of a threat acting upon internal or external vulnerability and causing harm to a resource.
    \item The product of the likelihood that an event will occur and the impact that the event might have on an information technology asset.
\end{itemize}
\begin{equation*}
    \text{RISK} = \text{Threats} \times \text{vulnerabilities} \times \text{Impact}
\end{equation*}
The impact of and event on an information asset is the product of vulnerability in the asset and the asset's value to it's stakeholders.
IT risk can be expanded to:
\begin{equation*}
    \text{RISK} = \text{Threat} \times \text{Vulnerability} \times \text{Asset Value }
\end{equation*}

\begin{equation*}
    \text{Level of Risk} = \text{Consequence} \times \text{Likelihood}
\end{equation*}
\begin{table*}
    %\centering
    \begin{tabular}{|l|l|l|}
        \hline
        \textbf{Risk Level} & \textbf{Consequence} & \textbf{Action}\\
        \hline
        \textbf{Extreme or High} & Serious or imminent danger & \begin{minipage}[t]{0.4\textwidth}
            \begin{itemize}
                \item Immediate measures are required to combat the risk
                \item Identify and impose controls to reduce the risk to a reasonably low level
            \end{itemize}
        \end{minipage} \\
        \hline
        \textbf{Medium} & Moderate danger & \begin{minipage}[t]{0.4\textwidth}
            \begin{itemize}
                \item Immediate action is not required, but action should be implemented Quickly
                \item Implement controls as soon as possible to reduce the risk to a reasonably low level  
            \end{itemize}
        \end{minipage}\\
        \hline
        \textbf{Low} & Negligible danger & \begin{minipage}[t]{0.4\textwidth}
            \begin{itemize}
                \item Take preventative steps to mitigate the effects of risk
            \end{itemize}
        \end{minipage}\\
        \hline
    \end{tabular}
\end{table*}

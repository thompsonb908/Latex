The Atlantic Charter outlined the idealized vision for political and economic order of the postwar world.
The document outlined the principles that the United States and the United Kingdom hoped would be the future for the world after the war.
Signed by Roosevelt and Churchill on August 14, 1941, this document was a major shift the in the ideology of the United States.

The first two points of the Atlantic Charter cement that neither the United States or the United Kingdom are participating with the anticipation of increasing their territories or receiving compensation.
The fourth and fifth points argue that no nation should be denied access to the world's raw materials and that every nation should strive to advance its economy.
The sixth, seventh and eighth points can be summarized as every nation should feel safe within their own boundaries, be able to travel the seas without hinderance, and abandon the use of force against other nations.

The most important point in the Atlantic Charter in relation to ideas of isolationism is the third point.
The third point states that \textquote{all people should be able to choose the form of government under which they will live; and they wish to see sovereign rights and self-government restored to those who have been forcibly deprived of them.}
With the signing of this document Roosevelt has dedicated the United States to seeking out these issues and correcting them, moving America from an isolationist country to a interventionist country.
\section{Overview}
\subsection{Terms and Evolution of Hacking}
\textbf{Hackers / Crackers}
\begin{itemize}
    \item Access to computer system or network without authorization.
    \begin{itemize}
        \item Breaks the law; can go to prison.
    \end{itemize}
\end{itemize}

\textbf{Ethical Hacker}
\begin{itemize}
    \item Performs most of the same activities as a hacker with the owners / company's permission.
\end{itemize}

Hackers became more prolific and dangerous after the availability of the Internet expanded to include the general public.
Originally hacking was the skillful modification of systems.
As the Internet began to include more people and technology, there was more of a following for hacking.
First for fun or curiosity, then for maliciousness and financial reasons.

\subsection{What is Hacking}
\begin{itemize}
    \item Stealing passwords and usernames - ''Theft of access'' 
    \item Network intrusions - Form of digital trespassing
    \item Social Engineering - Humans are the weakest point of a computer system.
    \item Posting / transmitting illegal material - Illegal material can spread very quickly with the use of social media.
    \item Financial Fraud - Deception of one party to elicit information or system access typically for financial gain to to cause damage.
    \item Software / data piracy - the possession, duplication / distribution of software in violation of license agreement / act of removing copy protection.
    \item Dumpster Diving - gathering of material that has been discarded or left in unsecured or unguarded receptacles.
    \item Creating and Planting Malicious Code - refers to items such as viruses, worms, spyware, adware, rootkits, and other types of malware. (any type of software written to wreak havoc, destruction or disruption.)
    \item Unauthorized Destruction or Alteration - Modifying, destroying, or tampering of information without permission.
    \item Data diddling - Unauthorized modification of information to cover up other malicious activity.
    \item Denial of Service (DoS/DDoS) - Overload a systems resources so it cannot provide the required services to its users.
    \item Ransomware - encryption of key files on a system for the purpose of extracting payment from the victim.
\end{itemize}

\subsection{Types of Hackers}
\begin{itemize}
    \item Black-Hat Hackers: Bad guys, may or may not have an agenda (Criminal)
    \item White-Hat Hackers: Good guys, have a code of ethics.
    \item Gray-Hat Hackers: Could be good or bad (do not trust)
    \item Suicide Hackers: Trying to prove a point, are not stealthy because they do not care about repercussions.
    \item Script Kiddies: No or limited training, use prebuilt tools, mainly curious.
\end{itemize}

\subsection{Motives of Hackers}
\begin{itemize}
    \item Monetary - financial gain
    \item Status - Increased reputation within their communities
    \item Terrorism - To scare, intimidate, or cause panic to the target group or the victims
    \item Revenge / Hatred - Disgruntled employees
    \item Hacktivism - Bring attention to a cause, group, or political ideology.
    \item Fun - Attacks with no specific goal
\end{itemize}

\subsection{Tasks of Ethical Hackers}
\begin{itemize}
    \item \textbf{Penetration Test}\\ Perform attacks / exploits on system and network
    \begin{enumerate}
        \item Attempt to break into a company's network or system or application to find the weakest link.
        \item Report on the attack to company
        \item Company decides how to use the information given by the attack.
    \end{enumerate}
    \item \textbf{Vulnerability Assessment / Research}\\
    \begin{enumerate}
        \item Tester enumerates all vulnerabilities found in an application or system.
        \item Passively uncovering vulnerabilities or weaknesses.
        \item Correct found vulnerabilities.
    \end{enumerate}
    \item \textbf{Security Test}\\
    \begin{enumerate}
        \item Analyze company's security policy and procedures.
        \item Security tester must examine best practices, legal issues, and industry regulations.
    \end{enumerate}
\end{itemize}

\subsection{Goals of Penetration Tester}
Every security minded organization enforces the CIA triad (confidentiality, integrity and availability).
Penetration testers work to find the weaknesses in the client's environment that would disrupt the CIA triad.
The anti-CIA triad is the DDA (Disclosure, Disruption, Alteration).
Blue teams try to maintain CIA and red teams try to do DDA.

\subsection{Penetration-Testing Methodologies}
\begin{itemize}
    \item White box model\\ ''Full information''
    \begin{itemize}
        \item Tester is told about network topology.
        \item Floor plan / network plan
        \item Interviews with IT personnel and employees.
        \item Information for routers, switches, firewalls, and IDS's.
        \item List of OS running on systems.
    \end{itemize}
    \item Black box model\\ ''No information''
    \begin{itemize}
        \item Given to details about technologies used.
        \item Tester has to find details themselves.
        \item Tests security personnel's ability to detect an attack.
        \item Staff does not know about the test.
    \end{itemize}
    \item Gray box model\\ ''Some information''
    \begin{itemize}
        \item Hybrid of white and black box model.
    \end{itemize}
\end{itemize}

\subsection{Pen Testing Process}
\begin{enumerate}
    \item Foot-Printing
    \item Scanning
    \item Enumeration
    \item Hacking (4 phases)
    \begin{enumerate}
        \item System Hacking
        \item Covering Tracks
        \item Planting Backdoors
        \item Privilege Escalation
    \end{enumerate}
\end{enumerate}
After following this 7-step process the tester should be prepared to present a detailed report of their findings.
Depending on the statement of work there could be a presentation of the report, a presentation and recommendations, or a presentation and recommendations with remediation.

\subsection{Code of Conduct and Ethics}
\begin{itemize}
    \item Keep private and confidential information gained in your professional work.
    \item Protect intellectual property
    \item Provide service in your areas of competence, being forthright about limitations.
    \item Disclosure to appropriate persons or authorities only.
    \item Never knowingly use a software or process that is obtained illegally or unethically.
    \item Ensure good management for any project you lead.
    \item Conduct yourself in the most ethical and competent manner.
    \item Do not associate with malicious hackers or engage in malicious activity.
\end{itemize}

\subsection{Before a Pen-Test Answer these questions:}
\begin{itemize}
    \item Why did the client request a pen test?
    \item 
\end{itemize}
\section{Chapter 2: System Fundamentals}
\subsection{Knowing Operating Systems}
Operating systems offer common vulnerabilities if not configured properly by the administrator.
Quite a few organizations are using OS without configuration or are running old versions that could be susceptible to attackers.

\subsubsection{Windows}
Windows is a large target because of its large user base.
An endless stream of updates are hard for users to keep track of.
Default configurations are left in place by many users.
Old legacy versions are still in use.

Microsoft Server dominates industry server rooms / farms.
Many organizations still use outdated servers that are not secure against modern attacks.

\subsubsection{Linux}
Linux is an advanced operating system for tech enthusiasts. Also popular with pen testers and a widely used Server OS. Linux distros are open source meaning that the responsibility of maintaining the security of the system is on the administrator.

\subsubsection{Mac OS X}
Security solutions are lacking compared to other platforms, many of its users do not think they are vulnerable.
Features are typically all enabled even if the user is not using them, such as 802.11 wireless and bluetooth connectivity that creates a much larger attack surface.

Apple devices do not work well on Windows domains.
Some features work well but others will not.

\subsubsection{Android}
Android is estimated to be on 80\% of smartphones in use today.
Designed for touchscreen devices.
Android has seen rapid growth because of its flexibility and customization and because it is free to use.

Counterfeit devices are cheap and often contain malware.
Has similar issues to Linux were there are many different versions.

\subsection{Networks}
Networks come in four different sizes, based on the scale of the network.
\begin{itemize}
    \item Personal Area Network (PAN)
    \item Local Area Network (LAN)
    \item Municipal Area Network (MAN)
    \item Wide Area Network (WAN)
\end{itemize}
\subsubsection{Network Types}
\begin{itemize}
    \item Bus
    \begin{itemize}
        \item All connecting nodes in a single run.
        \item Bus is a common link to all machines.
        \item All connectivity is lost if the bus is damaged.
        \item No true bus-topology is used today.
    \end{itemize}
    \item Ring
    \begin{itemize}
        \item Use the common connector in a loop style.
        \item Some layouts use concentric rings for redundancy if one fails.
        \item Each node attaches to the ring and delivers packets according to its designated turn or availability of the \textbf{token}.
    \end{itemize}
    \item Star
    \begin{itemize}
        \item Common because of ease of setup and isolation of connectivity problems.
        \item Multiple nodes connect to a central network device.
        \item Popular because of its resistance to outages.
    \end{itemize}
    \item Hybrid
    \begin{itemize}
        \item Most common layout in use today.
        \item Current networks are the offspring of many additions and alterations over many years of expansion or changes. 
    \end{itemize}
    \item Mesh
    \begin{itemize}
        \item Web of cabling that attaches a group of nodes to each other.
        \item Often used for mission-critical services because of its high level of redundancy and resistance to outages.
        \item The Internet, is one growing complex mesh network.
    \end{itemize}
\end{itemize}

\subsubsection{OSI Model}
Operating System Interconnect (OSI).
Defines a common set of rules for vendors of hardware and software.
\begin{itemize}
    \item Layer 1: Physical\\
    Physical media and dumb devices (cabling, Category 5e, RJ-45 connectors and hubs).
    \item Layer 2: Data Link\\
    Ensures that data transfer is free of errors (Media access control MAC, and link establishment).
    802.3 Ethernet and 802.11 Wi-Fi protocols.
    \item LAyer 3: Network\\
    Determines the path of data packets based on different factors defined by the protocol used.
    IP addressing for routing data packets (ICMP is here).
    \item Layer 4: Transport\\
    Layer ensures the transport re sending of data is successful, includes error checking operations as well as working to keep data messages in sequence.
    \item Layer 5: Session\\
    Identifies session between different network entities.
    Monitors and controls connections, allowing multiple, separate connections to different resources.
    \item Layer 6: Presentation\\
    Translates data that is understandable by the next receiving layer.
    Traffic flow is presented in a format that can be consumed bt the receiver and can optionally be encrypted vis SSL.
    \item Layer 7: Application\\
    User platform in whic the user and the software processes within the system can operate and access network resources.
    Applications and software suites that we use on a daily basis are in this layer.
\end{itemize}
\subsection{4 Phases of Ethical Hacking}

\subsubsection{Footprinting}


\subsubsection{Scanning}
\subsubsection{Enumeration}
\subsubsection{System Hacking}

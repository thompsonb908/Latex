\documentclass[12pt]{article}
\usepackage[top=1in, bottom=1in, left=1in, right=1in]{geometry}
%\usepackage[margin=1in]{geometry}
\usepackage[onehalfspacing]{setspace}
%\usepackage[doublespacing]{setspace}
%\usepackage{amsmath, amssymb, amsthm}
%\usepackage{enumerate, enumitem}
%\usepackage{fancyhdr, graphicx, proof, comment, multicol}
\usepackage[none]{hyphenat} % This command prevents hyphenation of words
%\binoppenalty=\maxdimen % This command and the next prevent in-line equation breaks
%\relpenalty=\maxdimen
%    Good website with common symbols
% http://www.artofproblemsolving.com/wiki/index.php/LaTeX%3ASymbols
%    How to change enumeration using enumitem package
% http://tex.stackexchange.com/questions/129951/enumerate-tag-using-the-alphabet-instead-of-numbers
%    Quick post on headers
% http://timmurphy.org/2010/08/07/headers-and-footers-in-latex-using-fancyhdr/
%    Info on alignat
% http://tex.stackexchange.com/questions/229799/align-words-next-to-the-numbering
% http://tex.stackexchange.com/questions/43102/how-to-subtract-two-equations
%    Text align left-center-right
% http://tex.stackexchange.com/questions/55472/how-to-make-text-aligned-left-center-right-in-the-same-line
%\usepackage{microtype} % Modifies spacing between letters and words
%\usepackage{mathpazo} % Modifies font. Optional package.
%\usepackage{mdframed} % Required for boxed problems.
%\usepackage{parskip} % Left justifies new paragraphs.
\linespread{1.1} 

\usepackage{cite}

%figure support
\usepackage{import}
\usepackage{xifthen}
\pdfminorversion=7
\usepackage{pdfpages}
\usepackage{transparent}
\newcommand{\incfig}[1]{%
	\def\svgwidth{\columnwidth}
	\import{./figures/}{#1.pdf_tex}
}
\graphicspath{ {./figures/} }
\pdfsuppresswarningpagegroup=1

%\newenvironment{problem}[1]
%{\begin{mdframed}[linewidth=0.8pt]
%        \textsc{Problem #1:}
%
%}
%    {\end{mdframed}}

%\newenvironment{solution}
%    {\textsc{Solution:}\\}
%    {\newpage}% puts a new page after the solution
%    
%\newenvironment{statement}[1]
%{\begin{mdframed}[linewidth=0.6pt]
%        \textsc{Statement #1:}
%
%}
%    {\end{mdframed}}

%\newenvironment{prf}
 %   {\textsc{Proof:}\\}
 %   {\newpage}% puts a new page after the solution

\begin{document}
% This is the Header
% Make sure you update this information!!!!
\noindent
\textbf{CIS 4367.01} \hfill \textbf{Brandon Thompson} \\
\normalsize Prof. Elibol \hfill Due Date: 4/25/2020 \\

% This is where you name your homework
\begin{center}
\textbf{Homework 17}
\end{center}

	\doublespacing

	The internet of things is an architecture of devices that communicate between 
	each other and other services. This results in the generation of enormous amounts
	of data which has to be stored, processed and presented in a seamless, efficient,
	and easily interpreted form. \cite{gubbi} A simple IoT architecture is composed
	of sensors that collect data and send it to a database, the database retrieves
	and compiles necessary information and the user interface and displays this information
	to the user in as close to real time as possible.

	In 2016 Ronen et al \cite{ronen} classified attacks on IoT architectures into four groups,
	ignoring functionality, reducing functionality, misusing functionality and extending
	functionality. The main focus of the paper was to show an extended functionality attack
	on smart lights to leak data from secure locations or cause epileptic people to have
	seizures. This assumes that there is already an infected computer and the attacker is
	using the lights to retrieve the compromised data.

	Williams et al \cite{williams} conducted a large scale Nessus vulnerability scan on 156,680
	consumer IoT devices and found that 12\% of these devices contained a vulnerability of
	'critical', 'high', 'medium' or 'low' risk. The devices scanned were webcams, smart TV's
	and printers that were considered part of an IoT infrastructure. This means that 12\% of
	consumer IoT devices could be susceptible to attacks. Considering the number of devices
	is expected to grow by 21 percent annually, rising to 18 billion between 2016 and 2022
	\cite{ericson}
	that are being added each year, this percentage is not acceptable and could cause major
	security issues in the future if companies do not improve the security on IoT devices.

	\newpage
	\bibliography{homework17.bib}{}
	\bibliographystyle{plain}
\end{document}

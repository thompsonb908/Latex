\documentclass[a4paper]{article}

\usepackage[utf8]{inputenc}
\usepackage[T1]{fontenc}
\usepackage{textcomp}
\usepackage[english]{babel}
\usepackage{amsmath, amssymb, amsthm}


%figure support
\usepackage{import}
\usepackage{xifthen}
\pdfminorversion=7
\usepackage{pdfpages}
\usepackage{transparent}
\newcommand{\incfig}[1]{%
	\def\svgwidth{\columnwidth}
	\import{./figures/}{#1.pdf_tex}
}
\graphicspath{ {./figures/} }
\pdfsuppresswarningpagegroup=1

\theoremstyle{definition}
\newtheorem{definition}{Definition}[section]

\theoremstyle{remark}
\newtheorem*{remark}{Remark}

\begin{document}
	\title{Chapter 1}
	\author{Brandon Thompson 5517}
	\maketitle

	\section{Computer Security}
	The NIST Internal/Interagency Report defines \textit{computer security} as:
	
	\begin{definition}{Computer Security:}
		Measures and controls that ensure confidentiality, integrity, and availability of
        	information system assists including hardware, software, firmware, and information
        	being processed, stored, and communicated.
	\end{definition}

	\subsection{CIA Triad}
	The CIA triad is the combination of \textbf{Confidentiality}, \textbf{Integrity}, and \textbf{Availability}.
	These concepts are the fundamental security of both data and for information and computing services.
	%figure 1.1
	The use of the CIA triad to define security objectives is well established, some in the security
	field feel that additional concepts are needed to present a complete picture.
	\begin{itemize}
		\item Confidentiality:
			\begin{itemize}
				\item Unauthorized disclosure of information.
			\end{itemize}
		\item Integrity:
			\begin{itemize}
				\item Unauthorized modification or destruction of information.
			\end{itemize}
		\item Availability:
			\begin{itemize}
				\item Disruption of access to or use of information or system.
			\end{itemize}
		\item Accountability:
			\begin{itemize}
				\item The requirement for actions of an entity to be traced uniquely to
					that entity.
				\item This supports non-repudiation, deterrence, fault isolation, intrusion
					detection and prevention, and after-action recovery and legal action.
				\item Must be able to trace a security breach to a responsible party.
				\item Systems must keep records of their activities to permit later forensic
					analysis to trace security breaches to aid in transaction disputes.
			\end{itemize}
		\item Authenticity:
			\begin{itemize}
				\item Being genuine and being able to be verified and trusted; confidence
					in the validity of a transmission, message, or message originator.
			\end{itemize}
	\end{itemize}
	
	\subsection{Levels of Impact}
	Three levels of impact on organizations or individuals should there be a breach of security.
	\begin{itemize}
		\item Low:
			\begin{itemize}
				\item Minor damage to assets.
				\item Minor financial loss.
				\item Degradation in capability such that primary functions are still
					operational, but noticeably reduced.
				\item Minor harm to individuals.
			\end{itemize}
		\item Moderate:
			\begin{itemize}
				\item Significant damage to assets.
				\item Significant financial loss.
				\item Degradation in capability such that primary functions are still
					operational, but significantly reduced.
				\item Harm to individuals that does not involve loss of life or serious,
					life-threatening injury.
			\end{itemize}
		\item High:
			\begin{itemize}
				\item Major damage to assets.
				\item Major financial loss.
				\item Organization is not able to perform one or more of its primary
					functions.
				\item Severe or catastrophic harm to individuals involving loss of life
					or serious life-threatening injuries.
			\end{itemize}
	\end{itemize}
	
	\subsection{Computer Security Challenges}
	\begin{enumerate}
		\item Computer security is not as simple as it first appears.
		\item Consider potential attacks on security features.
		\item Procedures used to provide particular services are often counterintuitive.
		\item Physical and logical placement needs to be considered.
		\item Security mechanisms typically involve secret information that leads to
			questions about how this secret information is created, distributed,
			and protected.
		\item Attackers only need to find a single weakness, while designer needs to find
			and eliminate all weaknesses.
		\item Security is often an afterthought, rather than an integral part of a system.
		\item Security requires regular can constant monitoring.
		\item Natural tendency on part of mangers to see little benefit from security
			investment until security fails.
		\item Many users view strong security as an impediment to efficient and
			user-friendly operation.
	\end{enumerate}

	\subsection{Common Terms}
	\begin{definition}{Adversary (Threat Agent)}
		Individual, group, organization or government that conducts or has intent to
		conduct detrimental activities.
	\end{definition}
	\begin{definition}{Attack}
		Any kind of malicious activity that attempts to collect, disrupt, deny, degrade,
		or destroy information system resources or the information itself.
	\end{definition}
	\begin{definition}{Countermeasure}
		A device or techniques thats objective is the impairment of adversarial activity, or
		the prevention of espionage, sabotage, theft, or unauthorized access to or use of
		sensitive information or information systems.
	\end{definition}
	\begin{definition}{Risk}
		A measure of the extent to which an entity is threatened by a potential circumstance
		or event, and typically a function of:
		\begin{enumerate}
			\item The adverse impacts that would arise if the circumstance or event occurs.
			\item The likelihood of occurrence.
		\end{enumerate}
	\end{definition}
	\begin{definition}{Security Policy}
		Set of criteria for the provision of security services. Defines and constrains the
		the activities of a data processing facility in order to maintain a condition of security
		for systems and data.
	\end{definition}
	\begin{definition}{System Resource (Asset)}
		A major application, general support system, high impact program, physical plant,
		mission critical system, personnel, equipment, or a logically related group of systems.
	\end{definition}
	\begin{definition}{Threat}
		Any circumstance with potentially adverse impacts to organizational operations.
	\end{definition}
	\begin{definition}{Vulnerability}
		Weakness in an information system, system security procedures, internal controls
		or implementation that could be exploited by a threat source.
	\end{definition}
	
	%figure 1.2
	
	\subsection{Assets of a Computer System}
	\begin{itemize}
		\item Communication facilities and networks:
		\item Data:
		\item Software:
		\item Hardware:
	\end{itemize}
	
	%figure 1.3
	
	\subsection{Vulnerabilities, Threats and Attacks}
	\begin{itemize}
		\item Categories of vulnerabilities
	\end{itemize}
\end{document}
